
Ennen laskuvarjon pakkausta on hypyn jälkeen palautettava muut hypyllä tarvitut varusteet omille paikoilleen. Näin ne ovat muiden käytössä, eivätkä ne rikkoonnu pakkaustilan lattialla. Laskuvarjon pakkausta edeltää myös varjon selvittäminen ja puhdistaminen hiekasta ja irtoliasta. Jos varjossa havaitaan vikaa ennen pakkausta tai sen aikana, on asiasta välittömästi ilmoitettava hyppymestarille tai kalustopäällikölle. Viallista tai märkää varjoa ei saa pakata! Pakkauskirja on esitäytettävä valmiiksi ja pakkauksessa noudatetaan aina huolellisuutta. Kun varjo pakataan oikein, se toimii ja säilyy ehjänä ja käyttökelpoisena pidempään. Vastuu pakkauksesta on sekä pakkaajalla että pakkaustarkastajalla. Tarkastuksen saa suorittaa A-D-lisenssihyppääjä. Pakkaustarkastuksesta on muistettava, että se on aina opetus- ja oppimistapahtuma. 


Pystypakkaus I tarkastusvaiheeseen suoritetaan seuraavasti: 


Tarkistetaan ohjauspunosten selvyys koko matkalta ohjauslenkistä kuvun takahelmaan, poistetaan kierteet ohjauspunoksista ja kiinnitetään puolijarrut. Nostetaan kupu olkapäälle, laitetaan 9 tunnelin suuta polvien väliin, viikataan A\mbox{-,} B\mbox{-,} C- ja D-punosryhmien välistä kankaat sivulle stabilisaattoriväleihin. Taitellaan kuvun takahelma ohjauspunosten välistä sivulle, muiden punosryhmien päälle, lasketaan slider tähden muotoon stabilisaattorien kovikkeita vasten. 

\section{ I tarkastusvaihe }
\label{pakkaustarkastus-i-tarkastusvaihe}


I tarkastusvaiheen tarkastaja vastaa siitä, että 

\begin{itemize}
\item  Repun, valjaiden ja varjon yleiskunto sekä siisteys on hyvä. 
\item  Repun luuppi ja avausjärjestelmän varusteet ovat kunnossa, slider ja apuvarjo on viritetty. 
\item  PL-hihna, lukko ja sokka/IA-kahva ovat kunnossa, eikä niissä ole kulumia. 
\item  Olkalukot ovat maata kohti eivätkä kantohihnat ole kierteellä. 
\item  Olkalukot ovat kiinni ehjillä luupeilla ja connectorilenkkien suojat ovat kiinni ja ehyet. 
\item  Puolijarrut on kiinnitetty, ohjauslenkkien päät on työnnetty käärmeensilmien läpi ja lenkit ovat kiinni tarroissaan ja taskuissaan. 
\item  Ylimääräinen ohjauspunos on lenkitetty siististi ja taiteltu kiinnitystarrojen (tai vastaavien) alle. 
\item  Ohjauspunokset ovat koko pituudeltaan erillään muista punoksista. 
\item  Kupu on laskostettu molemmilta puolilta stabilisaattoriväleihin siististi. 
\item  Takahelma on laskostettu ohjauspunosten välistä sivulle. 
\item  Punokset ovat tiukalla kuvun keskellä: 4+4 ohjaus\mbox{-,} 5+5 D\mbox{-,} 5+5 C\mbox{-,} 5+5 B- ja 5+5 A-punosta. 
\item  Slider on taitettuna kuvun sisälle ristiin, purjerenkaat nipussa. 
\item  Etuhelman 9 tunneliparia on tasattu. 
\end{itemize}

Tarkastuksen on oltava yksityiskohtainen, jotta pakkausta ei tarvitse keskeyttää myöhemmin puutteen tai vian takia. Tämän jälkeen tarkastaja kuittaa ensimmäisen tarkastusvaiheen pakkauskirjanpitoon todeten samalla esitäytettyjen pakkaustietojen paikkansapitävyyden. 


Pakkaus II tarkastusvaiheeseen suoritetaan seuraavasti: 

\begin{itemize}
\item  Asetetaan takahelman keskiosa ylös (warning label / keskimerkki) sliderin purjerenkaiden päälle ja kiedotaan niiden ympärille. 
\item  Kierretään takahelman reunat kuvun ympäri, vapautetaan tunnelinsuut jalkojen välistä, tasataan tunnelin suut kuvun reunan tasalle sekä kiristetään takahelman reunat ja rullataan 5–6 kierrosta. 
\item  Asetetaan kupu maahan punokset kireällä, poistetaan ilma kuvusta ja muotoillaan se hieman sisäpussia leveämmäksi pötköksi, taitetaan kuvun alaosa kuvun s-mutkalle ja kuvun yläosa edellisen taitoksen päälle sekä sujautetaan kupu sisäpussiin.  
\item  Vedetään ylimääräinen yhdyspunos pois sisäpussista ja tarkistetaan, ettei varjokangasta jää sisäpussin purjerenkaan ja yhdyspunoksen väliin.  
\item  Suljetaan sisäpussi purjerenkaiden kautta kulkevilla yksinkertaisilla kumilenkeillä ja punostetaan loput punokset kumilenkeillä kireälle. 
\end{itemize}
\section{ II tarkastusvaihe }
\label{pakkaustarkastus-ii-tarkastusvaihe}


II tarkastusvaiheen tarkastaja vastaa seuraavista asioista: 

\begin{itemize}
\item  Kupu on sisäpussissa tasaisesti ja pursuilematta. 
\item  Yhdyspunos on vedetty ulos sisäpussista kiinnitysrenkaaseen saakka eikä varjokangasta ole purjerenkaan ja yhdyspunoksen välissä. 
\item  Yhdyspunoksessa/pakkolaukaisuhihnassa ei ole kierrettä. 
\item  Punostus on tasainen ja kireä. 
\item  Toisen vaiheen tarkastaja vastaa myös (pakkaajan ohella) repun oikeasta sulkemisesta: 
	\begin{itemize}
	\item  Sisäpussi asetetaan reppuun oikein ja pyörittämättä. 
	\item  Kantohihnat tulevat taskuihinsa suoraan ja riittävän syvälle eikä niitä ole taitettu varavarjon repun pohjaa vasten. 
	\item  Ohjauslenkit ovat pysyneet kiinni tarroissaan ja taskuissaan. 
	\item  Apuvarjo ja yhdyspunos taitellaan ja pakataan oikein. 
	\item  Reppu suljetaan käsikirjan mukaisesti ja paranaru poistetaan. 
	\item  Viimeistellään pakkaus siistiksi. PL-hihna taitellaan lenkkeihinsä ja viedään varjo paikoilleen. 
	\end{itemize}
\end{itemize}

Tämän jälkeen tarkastaja kuittaa toisen tarkastusvaiheen pakkauskirjanpitoon. 

\section{ Harjoitus }
\label{pakkaustarkastus-harjoitus}

\begin{enumerate}[label=\bfseries \arabic*)]
\item  Harjoitellaan oppilaspäävarjon pakkaustarkastuksen tekemistä jokaisella pakkauskerralla tämän koulutuksen jälkeen. 
\item  Huomioidaan tarkastuksissa olevat erot eri avausjärjestelmillä olevissa varjoissa: pakkolaukaisu, jousi- ja HD-apuvarjo. 
\end{enumerate}
