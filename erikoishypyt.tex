
A-lisenssin saamisen jälkeen on mahdollista laajentaa hyppykokemustaan alkamalla harjoitella valitsemiaan hyppylajeja. Kokemuksen karttuessa voi hypätä myös erilaisia hyppyjä. Näitä ovat esimerkiksi yö\mbox{-,} vesi- tai näytöshypyt. Tällaiset erikoisuudet tuovat vaihtelua normaaliin hyppytoimintaan, mutta edellyttävät huolellisen ja asiantuntevan valmistautumisen. SIL:n ohjeet asettavat osalle erikoishypyistä kokemus- ja välinerajoituksia. Oman perushyppytaidon täytyy olla riittävällä tasolla, jotta vaativat haasteet eivät muodostu riskitekijäksi.  

\section{ Yöhypyt }
\label{erikoishypyt-yohypyt}


Sen lisäksi mitä laskuvarjohyppytoiminnalta vaaditaan, ovat SIL ry:n ohjeen \textit{Laskuvarjohyppääjien toiminnalliset ohjeet ja kelpoisuusvaatimukset} mukaan yöhypyt sallittuja seuraavin ehdoin: 

\begin{itemize}
\item  hyppääjällä on vähintään C-lisenssi  
\item  hyppääjällä on valaistu tai itsevalaiseva korkeusmittari 
\item  hyppääjä on varustettu kiinteällä, ympäristöön näkyvällä valolaitteella 
\item  hyppääjällä on suunnattava valolaite laskuvarjon tarkastamiseksi 
\item  maalialue on valaistu 
\end{itemize}

Yö määritellään ilmailumääräysten mukaan (OPS M1-1) seuraavasti: ''Auringon laskun ja nousun välinen aika silloin, kun valaisematonta kohdetta (savupiippua, mastoa, tms.) ei selvästi voida erottaa 8 kilometrin etäisyydeltä. Epäselvissä tapauksissa katsotaan yön vallitsevan''. 


Myös lentäminen yöllä on rajoitetumpaa kuin päivällä. Lentäjällä täytyy olla yölentokelpuutus ja koneen mittarivarustus täytyy olla riittävä. 


Maahenkilön pitää olla tietoinen tehtävistään ja heidän on pystyttävä toimimaan pimeässä (riittävät taskulamput). Koneeseen yhteydenpitoa varten on suotavaa olla maaradio. Laskeutumisalue valaistaan esimerkiksi käyttämällä turvalliseen paikkaan pysäköityjä autoja. Valot kannattaa suunnata niin, että ne eivät häikäise laskeutuvaa hyppääjää.  


Käytettävien valojen täytyy olla luotettavia, ja ne on kiinnitettävä ottaen huomioon vapaapudotus. Kemialliset kiilut ovat toimintavarmoja ja suositeltavia korkeusmittarin valaisemiseen. Ne toimivat myös kiinteänä, ympäristöön näkyvänä valolaitteena esimerkiksi päähineeseen teipattuna. On myös tärkeää muistaa asentaa valot niin, etteivät ne häikäise itseä tai muita hyppääjiä. Pieni käsivarteen teipattu taskulamppu soveltuu hyvin laskuvarjon tarkastukseen. Käytettävät lisävarusteet on kiinnitettävä siten, että ne eivät häiritse normaaliin hyppytoimintaan kuuluvia liikeratoja (esimerkiksi avaus, varavarjotoimenpiteet). 


Olosuhteiden on oltava yöhyppyjä hypättäessä hyvät. Taivaan pitää olla pilvetön hyppykorkeuden alapuolelta. Lisäksi on huomioitava ylätuulien voimakkuus ja suunta. Kovilla ylätuulilla yöhyppyjä ei kannata hypätä. 


Erikoishypyt ovat aina voimakkaasti stressiä lisääviä suorituksia. Yöllä kuluu huomattavasti pitempi aika tilanteiden havaitsemiseen ja niihin reagoimiseen kuin päivällä. Laskeutuminen on valaistuksesta huolimatta hankalampaa, eikä uloshyppypaikka aina ole oikea, mikä täytyy ottaa huomioon varjoa valittaessa. 


Hyppyyn valmistautuminen alkaa huolellisella suunnittelulla. Hyppääjien on tutustuttava hyppyalueeseen päiväsaikaan. Varjon aukaisua ja VV-toimenpiteitä on syytä harjoitella.  


Sääolosuhteet ja ennusteet täytyy tutkia huolella. Tuulen suunta muuttuu usein auringon laskun jälkeen. Valojen toiminta ja kiinnitys on varmistettava. Ennen hyppyä käydään hyppysuunnitelma läpi kokeneemman yöhyppääjän johdolla. Avauskorkeudet ja porrastukset sovitaan tarkasti ja niistä pidetään kiinni.  


Ensimmäisen yöhypyn ei kannata olla ryhmähyppy eikä hyppyä tule muutenkaan suunnitella liian vaikeaksi. Ennen hyppyä kannattaa antaa silmien tottua hämärään. Hämäränäkö kehittyy noin tunnissa. Tupakointi heikentää hämäränäköä huomattavasti. 


Koneessa on varottava häikäisemästä muita. Uloshyppypaikka on pystyttävä määrittämään tarkasti. Itse hyppy tehdään suunnitelman mukaisesti. Jos kyseessä on FS-hyppy, purkukorkeuden on oltava riittävän korkea. Liu’un lisäksi on syytä käyttää avausporrastuksia. Varjon varassa lennetään rauhallisesti. 

\section{ Vesihypyt }
\label{erikoishypyt-vesihypyt}


Hyppääjältä edellytetään vesihypyllä vähintään B-lisenssiä. Jos hypyt hypätään vesialueella, jossa on ilmeinen hukkumisvaara, kannattaa varustautua seuraavasti: hyppääjällä on oltava tarkoituksenmukainen kelluntaväline ja vedessä on oltava pelastustarkoitukseen soveltuva vesikulkuneuvo ja avustaja kutakin varjon varassa olevaa hyppääjää kohti. 


Uloshyppypaikka on määritettävä huolella. Virtaavaan veteen ja kauas rannasta hyppäämistä on vältettävä. Maahenkilöt huolehtivat alastulopaikalle tuulen suuntaa osoittavan laitteen, ensiapuvälineet, ja varmistavat, että alastulopaikaksi valittu alue on vapaa esteistä. Maahenkilöiden on oltava tietoisia tehtävistään ja heidän on pystyttävä toimimaan niin, että hyppääjä voidaan tarvittaessa pelastaa vedestä. Käytettävän veneen on oltava riittävän tukeva. Se ei saa kaatua, vaikka veneessä oleva henkilö auttaa vedessä olevan hyppääjän veneeseen.  


Kalustoa valittaessa on otettava huomioon, että kaikki automaattilaukaisimet ja korkeusmittarit eivät kestä vettä (jos hyppykorkeus vesihypyllä ei ylitä 1500 metriä, visuaalinen korkeusmittari ei ole pakollinen varuste). ZP-kangas on sinänsä melko kutistumatonta, mutta varjon valmistuksessa käytetyt vahvikenauhat eivät välttämättä säilytä mittojaan kuivuttuaan, jolloin varjon ominaisuudet huonontuvat.  


Varjon aukaisua ja varavarjotoimenpiteitä kannattaa harjoitella pelastusliivit päällä, sillä kahvojen paikat saattavat muuttua. Kelluntavälinettä valittaessa on otettava huomioon hyppyvarustuksen tuoma painonlisäys. Ennen hyppyä käydään hyppysuunnitelma läpi kokeneen hyppääjän johdolla. 


Hyppyä toteutettaessa on pidettävä huolta, että varjon varassa ei ole enemmän hyppääjiä kuin on pelastusveneitä. Päävarjon irtipäästö veden yläpuolella voi olla kohtalokasta, sillä vesi on yllättävän kova elementti ja korkeutta on vaikea arvioida.  


Hypyn jälkeen on varjot ja muu kalusto kuivattava huolella auringon valolta suojassa. Merivedessä kastuneet varjot pitää huuhtoa makeassa vedessä ennen kuivaamista. Myös varavarjo on aukaistava ja kuivattava ennen uutta pakkausta. Kaikki metalliosat ja hihnat kuivataan mahdollisimmat hyvin esimerkiksi pyyhkeellä. Myös vesipelastusvälineet on huollettava ja palautettava omille paikoilleen. 

\section{ Näytöshypyt }
\label{erikoishypyt-naytoshypyt}


Näytöshypyt ovat julkisille paikoille hypättyjä hyppyjä - yleensä suuren tai pienen yleisön eteen. Näytöshypyt voivat olla tilattuja (esimerkiksi perhetapahtumat, yleisötapahtumat jne.) tai hyppyorganisaation esittelyä muulle yleisölle. 


Näytöshyppytoiminnassa on noudatettava SIL:n \textit{Toiminnallisia ohjeita}. Tässä ohjeessa on käsitelty mm. näytöshyppyorganisaatio, luvat, ilmoitukset, ehdot, näytöshyppyalue ja -olosuhteet.  


Parhaimmillaan hyvä näytös on erinomaista PR-toimintaa. Näytöshyppy on kuitenkin hyvin vaativa suoritus. Sen suunnittelussa ja toteutuksessa on oltava erittäin huolellinen, ja riskien ottamista pitää välttää. Kokeneetkin ja yleensä arviointikykyiset hyppääjät voivat tehdä virheitä, kun suorituspaineet kasvavat liikaa. Näytöshyppypaikkaan on tutustuttava aina etukäteen. 

\section{ Hyppytapahtumat (boogiet) }
\label{erikoishypyt-hyppytapahtumat-boogiet}


Laskuvarjohyppääjien kokoontumisia kutsutaan boogieiksi. Osa hyppytapahtumista keskittyy enemmän hyppäämiseen ja sen harjoitteluun, osa myös juhlimiseen. Boogieihin kokoontuu hyppääjiä jopa ympäri maailmaa ja lentolaitteet ovat usein isompia ja/tai niitä on enemmän kuin hyppypaikan normaalihyppytoiminnassa. Joskus boogiepaikkakin on sellainen, missä ei normaalisti hyppytoimintaa järjestetä. 


Suomessa boogieita järjestetään vuosittain useita, viime vuosina suurimpia tapahtumia ovat olleet Pimp My Fly (Utti), Freefly Playdate (Oripää) ja Embassy of Freedom (Pudasjärvi). Virossa järjestetään vuosittain Parasummer, johon osallistuu myös paljon suomalaisia. 


Ruotsissa järjestettiin 1982 - 2004 yksitoista kertaa Hercules-boogie, jossa oli käytössä Ruotsin Ilmavoimien C-130 Hercules kuljetuskoneita. Osallistujia oli satoja hyppääjiä ympäri maailmaa. Näiden koneiden saatavuus hyppykäyttöön on vähentynyt merkittävästi eikä koneita Ruotsissa ole enää siviilihyppykäytössä. Suomessa järjestettiin Hercules-leiri 1997. 


Kun kiinnostava tapahtuma on löytynyt, kannattaa aluksi selvittää mahdolliset hyppykokemusrajat ja noudattaa niitä. Tapahtumissa hypätään yleensä isoista koneista, jolloin myös hyppyryhmät ovat suurempia. Vähäinen kokemus voi tällöin johtaa vaaratilanteisiin.  


Yleensä ilmoittautumalla ennakkoon säästää ilmoittautumismaksuissa. Järjestäjien tarjoama tapahtumaan liittyvä lisäinformaatio kannattaa lukea ja sisäistää sekä toimia ohjeiden mukaan. Joissain tapahtumissa järjestäjät edellyttävät määrätyn summan kattavaa kolmannen osapuolen vahinkovakuutusta. On syytä varmistaa oman vakuutuksen korvaussummien riittävyys sekä sen kattavuus ja käytännön järjestelyt ulkomailla. Normaali matkavakuutus harvoin kattaa laskuvarjohyppäämistä. Lajiliitolta saadun vakuutuksen korvaussummat saattavat olla pieniä hyppymaan, esimerkiksi Yhdysvaltojen hintatasoon tai vaatimuksiin nähden. Ilmailuliiton vakuutus on voimassa maissa, jotka noudattavat kansainvälisiä ilmailumääräyksiä. 


Boogieihin kannattaa saapua ajoissa ja tutustua paikallisiin olosuhteisiin. Ennen toiminnan alkua (joko päivittäin tai tapahtuman alussa) pidetään yleensä informaatiotilaisuus, jossa on syytä olla mukana. Jos joku asia on jäänyt epäselväksi, se on selvitettävä ennen hyppäämisen aloittamista. Kyseessä on paitsi oma, myös muiden hyppääjien turvallisuus. 


Boogieissa on usein isoja koneita tai niitä on paljon, joten ilmassa on runsaasti laskuvarjoja yhtä aikaa. Laskeutumiskuvio on yleensä tiukasti määrätty ja sitä tulee noudattaa. Väärään paikkaan laskeutumisesta saattaa aiheutua järjestäjän määräämiä sanktioita. 


Hyppytapahtumat ovat oivallisia paikkoja saada koulutusta kokeneemmilta hyppääjiltä. Niin kansainvälisissä kuin kotimaisissakin tapahtumissa on usein hyppyorganisaattoreita, jotka kokoavat hyppyryhmät halukkaille. Omat ryhmät kootaan FF- ja FS-hyppääjille erikseen ja hyppääjien erilainen kokemus huomioidaan hyppyjen suunnitteluvaiheessa. Oma hyppykokemus kannattaa tuoda rehellisesti esille, jotta saa hypystä mahdollisimman paljon irti vaarantamatta muita.  


Boogieihin liittyy yhtenä osana myös juhliminen. Hauskaa kannattaa kyllä pitää, mutta seuraavana aamuna pitää myös itsekritiikin olla riittävää. 


Boogieissa tarkastetaan asiakirjat ja välineet huolellisesti sisäänkirjoittautumisen yhteydessä. Ennen hyppytapahtumaan lähtöä on huolehdittava, että tarvittavat asiapaperit ovat mukana: lisenssi, hyppypäiväkirja, varjokirjat ja mahdolliset vakuutuspaperit. On myös tarkastettava, että lisenssi, kaikki kaluston tarkastukset ja pakkausjaksot (huom. varavarjon pakkausjaksot saattavat olla boogieissa tiukemmat kuin kotimaassa) ovat voimassa ja kalusto on hyvässä kunnossa. Erityisesti avausjärjestelmään kiinnitetään boogieissa huomiota. Luuppien kunto ja tiukkuus sekä apuvarjon taskun tiukkuus tarkistetaan. Boogieissa voi olla mahdotonta korjata puutteita tai se voi olla kallista. 

\section{ Kilpailut }
\label{erikoishypyt-kilpailut}


Laskuvarjourheilussa järjestetään kilpailuja monella eri tasolla (SM, PM, MM). Lisäksi mm. World Cup ja World Games ovat arvostettuja kilpailuja. 


Suomenmestaruuskilpailuja järjestetään eri puolilla maata. Joinakin vuosina järjestetään yhteiskilpailuja, jolloin kaikki lajit hypätään samassa tapahtumassa. Usein kuitenkin eri lajit (tai lajiryhmät: konventionaaliset, CF- ja vapaapudotuslajit) kilpaillaan omissa kilpailuissaan. Kerhot hakevat kisojen järjestämisoikeutta Suomen Ilmailuliitto ry:ltä. 


Virallisia SM-lajeja ovat tällä hetkellä konventionaalisissa lajeissa taitohyppy, tarkkuushyppy, joukkuetarkkuus ja yleismestaruus. FS-hypyissä kilpaillaan 4 ja 8 henkilön joukkueissa. FS:ssä hypätään myös Intermediate-sarjassa, jossa hyppääjien kokemustaso on rajoitettu. CF-puolella kilpaillaan neljän hengen kierrossa. Artistic-lajeista Freestyle (kahden hengen joukkue) ja Freefly (kolmen hengen joukkue) ovat SM-lajeja.  


SM-tasolla on järjestetty myös Para-Ski-kilpailuja. Kilpailuissa on ollut kaksi lajia, joissa on yhdistetty tarkkuushyppy kaltevaan rinteeseen sekä suurpujottelu tai murtomaahiihto (Nordic Para-Ski). 


Epävirallisia lajeja ovat mm. canopy piloting (pituus\mbox{-,} tarkkuus- ja nopeuslaskeutumiset), blade running (varjolla lennetään alas jyrkkää rinnettä pujotellen korkeiden viirien välistä) ja FS-hypyissä esimerkiksi 16-way, 20-way ja nopeustähti. CF:ssä kilpaillaan myös 8 hengen nopeusmuodostelma sekä 2 ja 4 hyppääjän sekvenssissä. Artistic-lajien puolella voidaan hypätä skysurfingia laudan kanssa. Lisäksi kilpailuja on järjestetty nopeushyppäämisessä. Uusin laji on speed skydiving, jossa tavoite on yksinkertainen: saavuttaa mahdollisimman suuri nopeus vapaapudotuksen aikana. 


SM-kilpailuihin vaaditaan A-lisenssin kelpoisuus ja SIL:stä haettu kilpailulisenssi (FAI-lisenssi). Joissakin sarjoissa pitää pystyä osoittamaan hyppymääränsä (ettei ylitä minimiä) sekä riittävät taitonsa, jotta kilpaileminen olisi turvallista. 


Kilpailujen järjestäjät lähettävät kerhoihin kilpailukutsut, joissa kerrotaan mm. kilpailupaikka ja -aika, hyppykonetyyppi, viimeinen ilmoittautumispäivä, hinta, hygieniapalvelut jne. Kilpailupaikalla ilmoittaudutaan järjestäjille, jotka tarkastavat asiakirjat sekä hyppyvälineiden kunnon. 


Kilpailukutsujen mukana tulee yleensä suomenkielinen tiivistelmä säännöistä, mutta viime kädessä Sporting Code -säännöstö on se, joka määrää FAI:n alaisissa kilpailuissa. 

\section{ Eri hyppylajit}
\label{erikoishypyt-eri-hyppylajit}


Seuraavassa esitellään lyhyesti eri hyppylajeja. Lisätietoja lajeihin löytyy lajikohtaisista jatkokoulutusoppaista. 

\subsection{ Speed skydiving }
\label{erikoishypyt-speed-skydiving}


Speed skydiving on uusimpia kilpailulajeja. Nimi voidaan kääntää vaikka nopeushypyksi. Siinä tavoite on yksinkertainen: saavuttaa mahdollisimman suuri nopeus vapaapudotuksen aikana. 


Tällä hetkellä nopeimmat hyppääjät pystyvät putoamaan yli 500 kilometrin tuntinopeudella. 


Lajin on tehnyt mahdolliseksi mikroprosessoripohjaisten korkeusmittarien käyttöönotto laskuvarjourheilussa. Niiden avulla pystytään mittaamaan ihmisen putoamisnopeus riittävän luotettavasti kilpailemista varten. 

\subsection{ Canopy piloting }
\label{erikoishypyt-canopy-piloting}


Canopy piloting (CP) eli kotoisammin swooppaaminen on yksi vauhdikkaimmista laskuvarjourheilun lajeista. Swooppaamisessa hyppääjä pyrkii saavuttamaan kuvullaan mahdollisimman pitkän ja tarkan loppuliidon, joka tapahtuu aivan maan pinnassa. 


Swooppi aloitetaan vauhdinotolla, jossa hyppääjä kääntää varjonsa syöksyyn. Syöksy oikaistaan sopivasti ennen maata, jotta saavutetaan optimaalinen nopeus swooppiin. Hyppääjät voivat saavuttaa jopa yli 100 km/h vauhdin. 


Swooppaaminen on varsin vaativa laskuvarjourheilun laji. Canopy pilotingissa suoritukset tapahtuvat aivan katsojien silmien edessä, joten yleisön on helppo seurata lajin tapahtumia. 

\subsection{ Freefly }
\label{erikoishypyt-freefly}


Freeflyssä on tavoitteena lentää mahdollisimman monipuolisesti, kaikissa mahdollisissa lentoasennoissa. Freeflyn perusasentoja ovat pää alaspäin lentäminen (head down), istualtaan lentäminen (head up) sekä liukuminen (tracking). Toisin kuin kuviohyppäämisessä, jossa kuviot tehdään pääosin tasossa, freeflyssa lennetään kolmiulotteisesti. Nimensä mukaisesti kyse on vapaasta lentämisestä, joten vain mielikuvitus on rajana hyppyjä suunniteltaessa. 

\subsection{ Freestyle }
\label{erikoishypyt-freestyle}


Freestyle on vapaapudotuslaji, jossa tehdään erilaisista lentoasennoista, pyörähdyksistä ja asentojen muutoksista koostuvia liikesarjoja. Monet liikkeet on lainattu suoraan uimahypyistä, tanssista tai voimisteluliikkeistä, ja niihin on lisätty vapaapudotuksen mukanaan tuomia uusia elementtejä. 


Kilpailuissa freestyleä hypätään kahden hengen joukkueissa, joissa toinen hyppääjä kuvaa freestylehyppääjän suorituksen. Suorituksessa arvosteltavia ominaisuuksia ovat ohjelman taiteellisuus, vaikeusaste, liikkeiden puhtaus ja kuvaajan työskentely. Kilpailuhypyt hypätään 4000 metrin korkeudesta ja työskentelyaika on 45 sekuntia. 


Kilpailuissa freefly:ta hypätään kolmen hengen joukkueilla, joissa yksi hyppääjä kuvaa kahden muun suorituksia osallistuen samalla myös itse suorituksiin. Hypyllä hyppääjät ottavat erilaisia otteita toisistaan ja lentelevät erilaisia kuvioita toisiinsa nähden. 

\subsection{ Kupukuviohyppääminen }
\label{erikoishypyt-kupukuviohyppaaminen}


Kupukuviohyppäämisessä (Canopy Formation, CF, entinen CRW) hyppääjät muodostavat kuvioita laskuvarjojen varassa tarttumalla kiinni toisten hyppääjien kupuihin. Lajin harrastamiseen on nykyään kehitetty aivan omanlaisia laskuvarjoja, jotka pysyvät hyvin lentokunnossa vaikeissakin tilanteissa. Kupuja on myös vahvistettu, jotta ne kestäisivät repeytymättä kovaakin vetoa. Kuvuissa on myös useampia ohjauslenkkejä kuin laskuvarjoissa yleensä. 


Kupukuviohypyissä on kaksi kilpailulajia: sekvenssi ja rotaatio. Sekvenssissä hyppääjät telakoituvat sekä päällekkäin että rinnakkain ja pyrkivät tekemään ennalta arvottuja kuvioita. Rotaatiossa kaikki hyppääjät telakoituvat päällekkäin, jonka jälkeen aina tornin ylin hyppääjä kiertää alimmaksi. Kupukuvioissa kilpaillaan sekä 2- että 4-miehisin joukkuein. Hypyt arvostellaan hyppyjoukkueeseen kuuluvan kuvaajan nauhoituksen perusteella. 

\subsection{ Kuvaaminen  }
\label{erikoishypyt-kuvaaminen}


Vaikka kilpailuissa kuvaajat ovatkin kuvio\mbox{-,} freestyle\mbox{-,} freeflying- ja kupumuodostelmajoukkueen jäseniä, kuvaaminen on kuitenkin oma taiteen lajinsa. 


Hyppysuoritusten kuvaaminen ilmassa on muuttunut viime vuosien aikana harvojen erityisosaajien lajista lähes kaikkien vähänkin kokeneempien hyppääjien harrastukseksi. Tämän kehityksen ovat mahdollistaneet lähinnä pienentyneet kamerat ja niiden hintojen reilu lasku. Nykyään vapaapudotuskuvaamisessa käytetään usein pieniä digitaalivideokameroita, joiden kuvanlaatu on hyvä. 


Kuvaaminen tarjoaa erilaisia haasteita riippuen siitä, mitä kuvaa. Siinä, missä kupumuodostelmahyppyjä kuvataan varjon varassa lentämällä, vaatii freestylehyppääjän kuvaaminen välillä hyvin aggressiivista liikehdintää vapaapudotuksessa, jotta etäisyys kuvattavaan pysyisi sopivana. 


Kuvauskäyttöön on suunniteltu myös omia haalareita. Muun muassa kuviohypyn kuvaamisessa käytetään haalareita, joissa käsivarren ja vartalon välissä on isot siivet, joilla kuvaaja pystyy säätelemään putoamisnopeuttaan ja liikehtimään tehokkaasti pystysuunnassa. 


Ilmakuvaaminen on yksi tapa kehittää itseään hyppääjänä. Se vaatii kuitenkin tietoa ja taitoja sekä paljon keskittymistä. Sen vuoksi kokemusrajaksi on asetettu C-lisenssi. Jos oma lentotaito ei ole hallinnassa, voi seurauksena olla katastrofi. Ennen kuvaamisen aloittamista pitää keskustella turvallisuuspäällikön sekä kokeneempien kuvaajien kanssa välineistä ja kuvaamiseen liittyvistä asioista. Äänikorkeusmittarin käyttäminen kuvaushypyillä on erittäin suositeltavaa. 

\subsection{ Kuviohyppääminen }
\label{erikoishypyt-kuviohyppaaminen}


Kuviohypyissä (Formation Skydiving, FS) hyppääjät tekevät kuvioita vapaapudotuksessa ottamalla otteita toisista hyppääjistä. Kuviohyppy on oivallista harjoitusta, kun halutaan oppia lentämään vapaapudotuksessa sekä hallitsemaan omaa vartaloa ja ilmavirtaa. Kuviohyppy tarjoaa kuitenkin erinomaisia haasteita kokeneemmillekin hyppääjille. 


Kilpailuissa kuviohyppyä hypätään sekä 4 että 8 hyppääjän joukkueissa. Joukkueeseen kuuluu lisäksi kuvaaja, jonka tehtävä on videoida joukkueen suoritus tuomarien arvostelua varten. Lisäksi esimerkiksi eri hyppytapahtumissa hypätään isoja kuvia, joissa mukana saattaa olla 10–50 hyppääjää. 

\subsection{ Liitohyppääminen }
\label{erikoishypyt-liitohyppaaminen}


Liitohyppäämiseen käytetään liitopukua, jonka avulla hyppääjät pystyvät yli kaksinkertaistamaan vapaapudotusajan sekä myös lentämään vapaapudotuksen aikana pitkiäkin matkoja. Hyvä liitohyppääjä pystyy lentämään 3 kilometrin vapaapudotuksen aikana 7–8 kilometrin vaakamatkan ja saavuttamaan yli 150 km/h vaakanopeuden.  


Liitohyppäämisessä ei tällä hetkellä maailmassa kilpailla eikä siihen ole kehitetty mitään sääntöjä. Liitopuvulla lentäminen vaikuttaa huomattavasti vapaapudotusnopeuteen ja vaatii hyvää kehonhallintaa vapaapudotuksen aikana. Liitohyppäämisen aloittaminen Suomessa vaatii vähintään C-lisenssin sekä lajioppaassa määritellyt vähimmäistaidot ja -kokemuksen. 

\subsection{ Taitohyppy }
\label{erikoishypyt-taitohyppy}


Taitohypyssä pyritään tekemään mahdollisimman nopeasti sarja, joka muodostuu neljästä 360 asteen käännöksestä ja kahdesta takavoltista. Hyppy kuvataan maasta, ja taitohyppysarjan tekemiseen kuluneeseen aikaan lisätään virhesekunteja sen mukaan, kuinka paljon vajaiksi tai liian pitkiksi käännökset ja takavoltit menevät. Taitohyppy hypätään 2200 metrin korkeudesta. Hypyn alussa hyppääjät pyrkivät saavuttamaan mahdollisimman suuren alkuvauhdin taitohyppysarjaa varten. Tämä tapahtuu yleensä pää alaspäin tapahtuvan pystysyöksyn eli tikkauksen avulla. Jotkut taitohyppäävät käyttävät jopa nopeuslaskupukujen tapaisia kumipukuja maksiminopeuden hakemiseen. 

\subsection{ Tarkkuushyppy }
\label{erikoishypyt-tarkkuushyppy}


Tarkkuushyppy on taitohypyn ohella vanhimpia laskuvarjourheilun muotoja. Tarkkuushypyssä hyppääjä pyrkii laskeutumaan maaliin, jonka halkaisija on vain 2 cm. Varjon avauksen jälkeen hyppääjien täytyy arvioida tarkasti vallitsevat tuulet, jotta he pystyisivät laskeutumaan haluttuun maaliin. Viimeiset metrit hyppääjä lähestyy maalia suoraan yläpuolelta ja pyrkii asettamaan jalkansa tarkasti maalin keskelle. Virhetulokset mitataan 16 senttimetriin asti. Tarkkuushypyssä käytetyt varjot ovat nimenomaan tarkkuushyppyyn suunniteltuja ja eroavat muista, niin suurelta kooltaan kuin siipiprofiililtaan ja lento-ominaisuuksiltaankin. 


Taito- ja tarkkuushyppyä yhdessä kutsutaan konventionaalisiksi lajeiksi. Usein taito- ja tarkkuushyppy kulkevatkin käsi kädessä, ja kilpailuihin tähtäävä hyppääjä harjoittelee molempia lajeja. Kilpailuissa jaetaan myös yleismestaruusmitalit, jonka tulokset lasketaan konventionaalisten lajien sijoitusten mukaan. 

