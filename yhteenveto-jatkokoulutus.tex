
Tässä vaiheessa olet jo itsenäinen oppilas etkä enää välttämättä tarvitse kouluttajaa koneeseen. Koulutuspäällikön luvalla voit käyttää hypätessä omia varusteitasi. Tärkein tavoitteesi vapaapudotuksessa on oppia turvallisen ryhmähyppäämisen perusteet. Voit myös kokeilla erilaisia hyppylajeja, kuten freehypyt. Varjon varassa keskityt kuvunkäsittelyyn ja tarkkuuslaskeutumisiin. 

\section{ Hyppysuoritukset }
\label{yhteenveto-jatkokoulutus-hyppysuoritukset}


Hyppysuoritukset voidaan hypätä kouluttajan valitsemassa järjestyksessä. 

\begin{itemize}
\item  5 ryhmähyppyä (\ref{jatkokoulutuksen-suoritukset-ryhmahyppy} s.\pageref{jatkokoulutuksen-suoritukset-ryhmahyppy}) 
\item  3 kuvunkäsittelyhyppyä (\ref{jatkokoulutuksen-suoritukset-kuvunkasittelyhyppy} s.\pageref{jatkokoulutuksen-suoritukset-kuvunkasittelyhyppy}) Jos oppilas ottaa käyttöön omat varusteet tai käyttää varjoa, jota ei ole tarkoitettu alkeis- tai peruskoulutukseen, suoritetaan kuvunkäsittelyhypyt ennen muita suorituksia. 
\item  Vapaavalintaisia hyppyjä siten, että A-lisenssiin vaadittavat 25 itseavaushyppyä saadaan täyteen. Kokeillaan freehyppäämistä (\ref{free-hyppaaminen} s.\pageref{free-hyppaaminen}) tai kouluttaja valitsee oppilaan osaamiseen ja kehityskohteisiin sopivimman suorituksen. Hypyillä on oltava oppimistavoite, pelkkiä täytehyppyjä ei tule hypätä. 
	\begin{itemize}
	\item  NOVA 2 kpl 
	\item  PL 5 kpl 
	\end{itemize}
\item  Koehyppy (\ref{jatkokoulutuksen-suoritukset-koehyppy} s.\pageref{jatkokoulutuksen-suoritukset-koehyppy}) 
\end{itemize}
\section{ Muut suoritukset }
\label{yhteenveto-jatkokoulutus-muut-suoritukset}

\begin{itemize}
\item  5 tarkkuuslaskeutumista hyppysuoritusten yhteydessä, laskeutuminen 25 metrin säteelle kouluttajan osoittamasta paikasta. 
\item  Pakkaustaidonnäyte (Voidaan suorittaa koulutusorganisaatiosta riippuen myös aikaisemmassa vaiheessa.) 
\item  Oppilasvarjon pakkaustarkastusnäyte 
\item  A-lisenssin teoriakoe. Kokeessa on tulee osata kaikki tähän mennessä luettu ja harjoiteltu sekä laskuvarjohyppäämistä koskevat lait, määräykset ja ohjeet (\ref{maaraykset-lait-ja-ohjeet} s.\pageref{maaraykset-lait-ja-ohjeet}). 
\end{itemize}
\section{ Suoritusten aikarajat }
\label{yhteenveto-jatkokoulutus-suoritusten-aikarajat}


Jos jatkokoulutuksen aikana oppilaalle tulee 30 vrk tai sitä pidempi hyppytauko, hänen on hypättävä totutteluhyppynä 15'' ennen seuraavaa ohjelman mukaista hyppyä. Hyppymestari voi määrätä muitakin suorituksia. 

