
Nyt sinulla pitäisi olla käsitys siitä, miten laskuvarjo lentää ja mihin sen ominaisuudet perustuvat. Tämä opas ei kuitenkaan tee sinusta hyvää kuvunkäsittelijää tai osaavaa swooppaajaa. Opettele varjosi lentämistä käytännössä ja vertaa lukemaasi teoriaa omiin havaintoihisi. Mieti harjoitushyppyjä hypätessä oppaan teoriaosuuksia. Huomaat varmasti havaitsevasi varjossasi uusia ominaisuuksia tai löytäväsi varjostasi täysin uusia kontrollipintoja ja lentotiloja. Toivottavasti myös ymmärrät paremmin eri tilanteissa, miksi varjosi tekee juuri tietynlaisen liikkeen. Muista aina, että hypätessä oppiminen ei lopu koskaan. Keskustele kokeneempien kanssa, kysele heidän mielipiteitään ja lue monipuolisesti erilaista tietoa, mitä tästä lajista voit löytää. Vertaa näin hankkimaasi tietoa tämän oppaan asioihin sekä käytännön kokemuksiisi. Pidä hauskaa varjon varassa. 


Maailmalla on nykyään tarjolla erittäin paljon tasokasta kuvunkäsittelykoulutusta. KTK suosittelee lämpimästi tällaiseen koulutukseen osallistumista. Koulutuksien vetäjillä on useimmiten tuhansien kuvunkäsittelyhyppyjen tausta ja he kouluttavat ammattimaisesti useita kursseja vuodessa. Näiltä koulutuksilta saa arvokasta käytännön tietoa sekä palautetta suoraan alan ammattilaisilta ja kursseja löytyy kaikentasoisille hyppääjille. 


\textbf{Koulutus- ja turvallisuuskomitea toivottaa turvallisia hyppyjä ja pehmeitä laskeutumisia!} 


\textbf{Lisäluettavaa:} 
\raggedright
\begin{itemize}
\item  Germain, Brian: The Parachute and Its Pilot. \url{http://www.bigairsportz.com/} 
\item  Sobieski, Jerry. The Aerodynamics and Piloting of High Performance Ram-Air Parachutes. \url{http://skysurfer.com.au/hosted/highperf.pdf} 
\item  Performance Designs: \url{http://www.performancedesigns.com/support.asp} 
	\begin{itemize}
	\item  Flying and Landing High Performance Parachutes Safely 
	\item  The Lowdown on Low Turns 
	\item  Choosing the right canopy (1 ja 2) 
	\item  Survival Skills for Canopy Control 
	\item  Wing loading and Its Effects 
	\end{itemize}
\end{itemize}
