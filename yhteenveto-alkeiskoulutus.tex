
Alkeisoppilaan kelpoisuus on voimassa vuoden myöntämispäivämäärästä. Jos kelpoisuus vanhenee, oppilaan tulee uusia teoriakoe ja käytännön näytteet ennen hyppäämistä. Oppilaan on saavutettava jokaisella hyppysuorituksella välttämättömät oppimistavoitteet hyväksytysti, jotta suoritus voidaan hyväksyä. 

\section{ Hyppysuoritukset, NOVA }
\label{yhteenveto-alkeiskoulutus-hyppysuoritukset-nova}

\begin{itemize}
\item  Taso 1 (\ref{nova-alkeiskoulutuksen-suoritukset-taso-1-tottuminen-vapaapudotukseen} s.\pageref{nova-alkeiskoulutuksen-suoritukset-taso-1-tottuminen-vapaapudotukseen}) 
\item  Taso 2 (\ref{nova-alkeiskoulutuksen-suoritukset-taso-2-asennon-hallinta} s.\pageref{nova-alkeiskoulutuksen-suoritukset-taso-2-asennon-hallinta}) 
\item  Taso 3 (\ref{nova-alkeiskoulutuksen-suoritukset-taso-3-stabiili-vapaapudotus} s.\pageref{nova-alkeiskoulutuksen-suoritukset-taso-3-stabiili-vapaapudotus}) 
\item  Taso 4 (\ref{nova-alkeiskoulutuksen-suoritukset-taso-4-kaannokset-90deg} s.\pageref{nova-alkeiskoulutuksen-suoritukset-taso-4-kaannokset-90deg}) 
\item  Taso 5 (\ref{nova-alkeiskoulutuksen-suoritukset-taso-5-kaannokset-360deg} s.\pageref{nova-alkeiskoulutuksen-suoritukset-taso-5-kaannokset-360deg}) 
\item  Taso 6 (\ref{nova-alkeiskoulutuksen-suoritukset-taso-6-irtiuloshyppy} s.\pageref{nova-alkeiskoulutuksen-suoritukset-taso-6-irtiuloshyppy}) 
\item  Taso 7 (\ref{nova-alkeiskoulutuksen-suoritukset-taso-7-puolisarja} s.\pageref{nova-alkeiskoulutuksen-suoritukset-taso-7-puolisarja}) 
\item  15'' lyhyt vapaa (\ref{nova-alkeiskoulutuksen-suoritukset-15-lyhyt-vapaa} s.\pageref{nova-alkeiskoulutuksen-suoritukset-15-lyhyt-vapaa}) 
\end{itemize}
\section{ Hyppysuoritukset, PL }
\label{yhteenveto-alkeiskoulutus-hyppysuoritukset-pl}

\begin{itemize}
\item  3 pakkolaukaisua (\ref{pl-alkeiskoulutuksen-suoritukset-pakkolaukaisu} s.\pageref{pl-alkeiskoulutuksen-suoritukset-pakkolaukaisu}) 
\item  3 harjoitusvetoa (\ref{pl-alkeiskoulutuksen-suoritukset-harjoitusveto} s.\pageref{pl-alkeiskoulutuksen-suoritukset-harjoitusveto}) 
\item  3'' itseaukaisu (\ref{pl-alkeiskoulutuksen-suoritukset-itseaukaisu-3} s.\pageref{pl-alkeiskoulutuksen-suoritukset-itseaukaisu-3}) 
\item  5'' itseaukaisu (\ref{pl-alkeiskoulutuksen-suoritukset-itseaukaisu-5} s.\pageref{pl-alkeiskoulutuksen-suoritukset-itseaukaisu-5}) 
\item  10'' itseaukaisu (\ref{pl-alkeiskoulutuksen-suoritukset-10} s.\pageref{pl-alkeiskoulutuksen-suoritukset-10}) 
\end{itemize}
\section{ Muut suoritukset }
\label{yhteenveto-alkeiskoulutus-muut-suoritukset}

\begin{itemize}
\item  Peruskoulutuksen teoriakoe. Opiskeltava alue on luvusta \textit{Yhteenveto: Peruskoulutus} (\ref{yhteenveto-peruskoulutus} s.\pageref{yhteenveto-peruskoulutus}) lukuun \textit{Peruskoulutuksen muut suoritukset} (\ref{peruskoulutuksen-muut-suoritukset} s.\pageref{peruskoulutuksen-muut-suoritukset}). 
\end{itemize}
\section{ NOVA-suoritusten aikarajat }
\label{yhteenveto-alkeiskoulutus-nova-suoritusten-aikarajat}

\begin{itemize}
\item  Jos tasokoulutuksen aikana oppilaalle tulee 30 vrk tai sitä pitempi hyppytauko, hänen on uusittava edellinen taso, kuitenkin enintään 3. taso.  
\item  NOVA-oppilaan on hypättävä alkeiskoulutuksen viimeinen hyppy (15'') \textbf{neljäntoista} vuorokauden sisällä viimeisestä tasohypystä, muuten hän joutuu uusimaan tason 7. 
\end{itemize}
\section{ PL-suoritusten aikarajat }
\label{yhteenveto-alkeiskoulutus-pl-suoritusten-aikarajat}

\begin{itemize}
\item  Jos alkeiskoulutuksen aikana oppilaalle tulee 30 vrk tai sitä pidempi hyppytauko, hänen on hypättävä totuttelusuoritus ennen seuraavaa hyppyä riippuen edellisestä suorituksesta: 
	\begin{itemize}
	\item  PL tai HV ⇨ PL 
	\item  3'', 5'' tai 10'' ⇨ HV 
	\end{itemize}
\item  Ensimmäinen itseaukaisuhyppy on hypättävä viimeistään seuraavan vuorokauden aikana viimeisestä hyväksytystä harjoitusvetosuorituksesta. 
\end{itemize}
