
Freehyppyjen tarkoituksena on perehtyä vapaan lentämisen alkeisiin. Harjoitushyppyjä ei saa ottaa liian vakavasti, sillä tarkoitus on antaa joitakin vinkkejä siitä, miten jatkossa kannattaa harjoitella perusasennosta poistumista. Freekoulutushyppyjen tarkoitus ei niinkään ole opettaa jotain yksittäistä lentoasentoa, vaan tutustuttaa oppilas lentämiseen erilaisissa asennoissa, oppia hallitsemaan vartaloaan muussakin asennossa kuin perusasennossa sekä oppia palautumaan yllättävistä asennoista stabiiliin perusasentoon. Ensimmäinen vaihe freelentämisen harjoittelussa on vatsallaan pysyminen. Kun tämä hallitaan, voidaan ruveta harjoittelemaan muissa asennoissa lentämistä. Seuraavissa teoriaosuuksissa käydään läpi muutamia freeperusasentoja ja -liikkeitä sekä annetaan perusteet niiden harjoitteluun. 

\section{ Sittis }
\label{free-hyppaaminen-sittis}


Katso sittisasennon perusasiat Freeoppaan puolelta (\ref{freefly-lentoasennot-sittiksen-perusasento} s.\pageref{freefly-lentoasennot-sittiksen-perusasento}). Siellä myös vinkkejä asentoon siirtymiseen. 

\section{ Stand up }
\label{free-hyppaaminen-stand-up}


Kun pysytään istuallaan vapaapudotuksessa, on siitä helppo harjoitella seisovaa asentoa eli stand upia. Stand upissa hyppääjä putoaa erittäin suurella nopeudella (korkeuden tarkkailu) seisovassa asennossa kädet sivuille ojennettuina. 


Siirtyminen sittiksestä stand upiin tapahtuu seuraavasti: 

\begin{itemize}
\item  Tarkastetaan korkeus ja viedään jalat alas yhteen. 
\item  Pidetään katse edelleen horisontissa tai hivenen sen alapuolella. Ei katsota missään tapauksessa suoraan alaspäin jalkoihin, sillä se aiheuttaa muutoksen vartalolinjassa. 
\end{itemize}

Pidetään stand upia muutama sekunti ja siirrytään takaisin sittikseen. 

\section{ Palloasento }
\label{free-hyppaaminen-palloasento}


Palloasennolla (ball position / recovery position) tarkoitetaan asentoa, johon freehyppäämisessä pyritään palautumaan, mikäli menetetään asennon hallinta esimerkiksi sittiksessä tai stand upissa. Asennossa hyppääjä käpertyy osittain palloksi, eli vetää jalat sisään ja koukistaa käsiä kyynärvarsista 90 asteen kulmaan eteenpäin, jolloin asento kääntyy putoamaan hieman selälleen takapainoisesti takapuoli maata kohti. Tällä asennolla koetetaan pitää sama putoamisvauhti kuin useimmilla freekuvilla. Recovery positionin käyttäminen lisää turvallisuutta, kun ilmassa on useampia freehyppääjiä. Jos sittiksen kaatuessa hyppääjä stabiloi itsensä suoraan perusasentoon mahalleen, hänen vauhtinsa hidastuu huomattavasti, ja tämä lisää törmäysriskiä. Oppilassuorituksena recovery positionia on hyvä alkaa harjoitella heti alusta alkaen: sittiksen / muun freeliikkeen kaatuessa otetaan recovery position ja pyritään palautumaan siitä takaisin sittikseen painamalla jalkoja ilmavirtaan ja levittämällä asento sittikseen. Recovery positionia voidaan harjoitella esimerkiksi menemällä kyykkyyn ja painamalla rintakehä reisiä vasten ja samanaikaisesti ottamalla käsiä hieman sisään ja kääntämällä kyynärvarret 90 astetta kyynärpäistä eteenpäin. 

\section{ Turvallisuus }
\label{free-hyppaaminen-turvallisuus}


Freehyppyjen turvallisuuteen liittyviä asioita on lueteltu seuraavassa: 

\begin{itemize}
\item  Tarkkaillaan korkeutta, sillä monissa freeasennoissa putoamisvauhti on erittäin kova. 
\item  Lopetetaan harjoittelu viimeistään 1400 metrin korkeudessa ja palataan perusasentoon. 
\item  Selällään tai istualtaan lennettäessä saattaa rintahihnassa oleva korkeusmittari näyttää väärin. Käsimittarin käyttäminen on suositeltavaa. 
\item  Työskennellään aina poikittain lennettyyn hyppylinjaan nähden, sillä monet freeasennot saattavat väärin tehtyinä ruveta ajelehtimaan (liukumaan). Poikittain linjaan nähden työskenneltäessä ei ajelehdita muiden hyppääjien alle tai päälle. 
\item  Freehypyillä on ensiarvoisen tärkeää, että valjaat ovat sopivat, joten sovitetaan valjaat kunnolla ja kiristetään valjaat tiukalle, jotta ne eivät pääse liikkumaan päällä vapaan aikana. 
\item  Kun puetaan varusteita päälle, pidetään huoli siitä, että kaikki ylipitkät valjashihnat on niputettu kumilenkeillä tarpeeksi tiukasti. 
\item  Varmistetaan, etteivät vaatteet pääse kahvojen ja pampuloiden päälle. 
\item  Pidetään huoli, että apuvarjo on tiukasti taskussaan. 
\item  Joskus selällään tai istualtaan lennettäessä reisihihnat saattavat liukua polvitaipeisiin. Tämän estää reisihihnojen välissä haaran kohdalla oleva kumilenkki. Pyydetään kalustohenkilöä ompelemaan kuminauha jalkahihnoihin, jos sellaista ei ole. 
\item  Muistetaan pysyä hypyn etukäteissuunnitelmassa. 
\end{itemize}

Headdown eli pää alaspäin lentäminen on oppilasvarusteilla kielletty! Varusteita ei ole suunniteltu eikä mitoitettu siihen tarkoitukseen. 

\section{ Hyppyohjelma }
\label{free-hyppaaminen-hyppyohjelma}


Harjoitellaan kutakin suoritusta kouluttajan ohjeiden mukaisesti harjoitusvaljaissa tai maassa ennen varsinaista suoritusta. 

\subsection{ Hyppy 1: Puhdas sittisharjoittelu }
\label{free-hyppaaminen-hyppy-1-puhdas-sittisharjoittelu}


Opetellaan istuvaa perusasentoa ja sittisasennon menetyksen yhteydessä palloasennon käyttöä freehypyillä. 

\subsection{ Hyppy 2: Sittis ja raajojen kontrolli }
\label{free-hyppaaminen-hyppy-2-sittis-ja-raajojen-kontrolli}


Poljetaan jaloilla vuoron perään ja heilutellaan käsiä. Tavoitteena on oppia raajojen hallintaa sittisasennossa. 

\subsection{ Hyppy 3: Sittiskäännökset }
\label{free-hyppaaminen-hyppy-3-sittiskaannokset}


Opetellaan sittiskäännöksiä. 

