
NOVA-koulutusohjelmassa oppilaalle annetaan henkilökohtaista koulutusta. Tasokoulutus koostuu seitsemästä erilaisesta tasosta. Jokaisella tasohypyllä on mukana vähintään yksi novahyppymestari. Tasokoulutuksessa edetään omaa tahtia. Jokaisella hypyllä hyödynnetään edellisten tasojen osaamista, joten oppilaan on suoritettava kukin taso hyväksytysti ennen etenemistä. 


Oppilaan on saavutettava välttämättömät oppimistavoitteet hyväksytysti päästäkseen seuraavalle tasolle. Jos ensimmäisen hypyn koulutuksesta on kulunut jonkin aikaa eikä sitä jostain syystä ole hypätty, koulutus on annettava uudelleen.  

\section{ Vapaapudotus NOVA-hypyillä }
\label{nova-alkeiskoulutuksen-suoritukset-vapaapudotus-nova-hypyilla}


Ulos hypätessä taivuta lantiosta, siirrä kädet ja jalat oikeaan asentoon ja pidä katse koneessa niin kauan kuin mahdollista. Pyri rentoutumaan puhaltamalla ulos hengittäessä suun kautta ja pidä leuka ylhäällä. Kuvittele puskevasi lantiota ilman halki. 

\subsection{Tarkkailukehä}
\label{nova-alkeiskoulutuksen-suoritukset-tarkkailukeha}


Laskettuasi 105:een aloita ja suorita omatoimisesti tarkkailukehä: 

\begin{enumerate}[label=\bfseries \arabic*)]
\item  MAA 
	\begin{itemize}
	\item  Siirrä katseesi 45° kulmassa alaspäin ja valitse jokin iso ja helposti löydettävä kiintopiste 
	\end{itemize}
\item  MITTARI 
	\begin{itemize}
	\item  Pidä taivutus, siirrä katseesi mittariin ja lue lukema. 
	\end{itemize}
\item  KORKEUS (VV-HM, varavarjon puoleinen hyppymestari) 
	\begin{itemize}
	\item  Katso vasemmalla puolella olevaa hyppymestaria käsivarren alta ja sano hänelle korkeusmittarin lukema. Hyppymestari voi tässä vaiheessa korjata asentoasi käsimerkein. Jos asennossa ei ole mitään korjattavaa, saat OK-käsimerkin (peukalo ylös). Älä jatka tarkkailukehää ennen kuin olet saanut OK-käsimerkin! 
	\end{itemize}
\item  KORKEUS (PV-HM, päävarjon puoleinen hyppymestari) 
	\begin{itemize}
	\item  Katso oikealla puolella olevaa hyppymestaria käsivarren alta ja suorita sama toimenpide kuin edellä. Jos kaikki on kunnossa saat OK-käsimerkin. Älä jatka tarkkailukehää ennen kuin olet saanut OK-käsimerkin! 
	\end{itemize}
\end{enumerate}

Tarkkailukehän avulla säilytät tietoisuuden korkeudesta ja suunnasta.  

\subsection{Harjoitusveto}
\label{nova-alkeiskoulutuksen-suoritukset-harjoitusveto}


Aloita itsenäisesti, mutta jos näet harjoitusveto-käsimerkin, se merkitsee harjoitusvedon aloittamista. Toimi rytmikkäästi ja sano toimenpiteet ääneen. 

\begin{enumerate}[label=\bfseries \arabic*)]
\item  TAIVUTA 
	\begin{itemize}
	\item  Työnnä lantiota alaspäin / varmista taivutus. 
	\item  Taivutus säilyy ja leuka pysyy ylhäällä koko harjoitusvedon ajan. 
	\end{itemize}
\item  TARTU 
	\begin{itemize}
	\item  Pidä katse ylhäällä. Siirrä vasen käsi pääsi yläpuolelle siten, että peukalo koskettaa päätäsi (otsaa) kämmenen ollessa auki. Siirrä oikea kätesi päävarjon avauskahvalle kämmen auki. Käsien liikkeiden on tapahduttava symmetrisesti, samalla tasolla ja yhdenaikaisesti. Vain kädet liikkuvat. Pään tai muun vartalon asento ei muutu.  
	\item  Kosketa kahvaa ja paina mieleesi kahvan paikka. On tärkeää tuntea selvästi, missä se on.   
	\end{itemize}
\item  VEDÄ 
	\begin{itemize}
	\item  Palauta kädet alkuperäiseen asentoon symmetrisesti ja yhdenaikaisesti. Vain kädet liikkuvat. Pään tai muun vartalon asento ei muutu. Huom! Kyseessä on harjoitusveto, joten älä vedä oikeasti. 
	\end{itemize}
\end{enumerate}
\subsection{ Varjon avaus }
\label{nova-alkeiskoulutuksen-suoritukset-varjon-avaus}


Ensimmäisellä hypylläsi aloitat avaustoimenpiteet 1600 metrin korkeudessa. Näytä ensin avausmerkki, jolla ilmoitat muille avaavasi varjosi, eli heilauta käsiäsi ristiin pääsi edessä. Avausmerkin jälkeen aloitat rauhallisesti ja täsmällisesti avaustoimenpiteet. Avaus tehdään samalla tavalla kuin harjoitusveto. Erona on vain se, että apuvarjon kahvasta tartutaan kunnolla kiinni ja apuvarjo heitetään reippaasti ilmavirtaan. Varjon ei siis pidä olla auki heti 1600 metrissä.  


TAIVUTA - TARTU - VEDÄ 


Tämän jälkeen laske ääneen 101, 102…105, jonka jälkeen tarkistat varjon ja suoritat mahdolliset selvitystoimenpiteet. Mikäli varjo ei selviä tai ei lennä, tee varavarjotoimenpiteet. Tasoilla 1-2 VV-HM pitää sinusta kiinni varjon avautumiseen asti ja tasoilla 3-7 HM varmistaa avautumisen.  

\section{ Taso 1 - Tottuminen vapaapudotukseen }
\label{nova-alkeiskoulutuksen-suoritukset-taso-1-tottuminen-vapaapudotukseen}


Kaksi novahyppymestaria, kolmella ensimmäisellä hypyllä radio. 


Jos näet vain yhden hyppymestarin, seuraa hänen ohjeitaan ja jatka hyppysuoritusta normaalisti ohjelman mukaan. Tasoilla 1 ja 2 \textbf{et} saa olla yksin vapaapudotuksessa. Jos et näe kumpaakaan hyppymestaria, avaa varjo välittömästi eli TAIVUTA-TARTU-VEDÄ 101..105 ja tarkista varjo.  

\subsection{ Oppilaan toiminta. }
\label{nova-alkeiskoulutuksen-suoritukset-oppilaan-toiminta}


Harjoittele 1. tason kulku (ja myöhemmin kaikki muut tasot) mahdollisimman hyvin etukäteen, sillä se helpottaa ja nopeuttaa huomattavasti oppimistasi. 

\subsection{ Ensimmäiseen hyppyysi valmistautuminen }
\label{nova-alkeiskoulutuksen-suoritukset-ensimmaiseen-hyppyysi-valmistautuminen}

\begin{enumerate}[label=\bfseries \arabic*)]
\item  Näytä, miten koneen ovelle siirrytään hyppyasentoon. 
\item  Esitä oikea hallittu uloshyppy mukaan lukien laskeminen. 
\item  Käy läpi ja harjoittele suunniteltu hyppy kohta kohdalta. 
\item  Harjoittele harjoitusvetoa samalla tavalla kuin oikeaa avausliikettä. 
	\begin{itemize}
	\item  a. TAIVUTA taivutus säilyy. Katse eteenpäin leuka ylhäällä. 
	\item  b. TARTU säilytä taivutus, kun siirrät samanaikaisesti oikean käden kämmen auki päävarjon avauskahvalle ja siirrä vasen käsi pääsi yläpuolelle siten, että peukalo koskettaa päätäsi (otsaa) kämmen avattuna. Paina tarkasti mieleesi kahvan paikka. Molemmat kädet pysyvät samassa tasossa. Vain kädet liikkuvat, ei pää eikä muu vartalon osa. Katse eteenpäin leuka ylhäällä. 
	\item  c. VEDÄ palauta kädet alkuperäiseen asentoon symmetrisesti sekä yhdenaikaisesti, katse eteenpäin ja leuka ylhäällä. Vain kädet liikkuvat, pään tai muun vartalon asento ei muutu lainkaan. HUOM. Tämä on harjoitusveto, joten älä vedä oikeasti.  
	\end{itemize}
\item  Harjoittele toimenpiteitä lennon ja hypyn aikana sattuvien vaaratilanteiden varalta. 
\item  Katso maatuulen suunta tuulipussista ja kuvaile oikeat ohjauskuviot. 
\item  Näytä oikea alastulo- ja kaatumisasento. 
\item  Kertaa käsimerkit. 
\end{enumerate}
\subsection{ Oppimistavoitteet }
\label{nova-alkeiskoulutuksen-suoritukset-oppimistavoitteet}


Jokaisella tasolla on omat oppimistavoitteet, joiden mukaan arvioidaan oppimisesi, sekä se, pääsetkö seuraavalle tasolle. Tutustu oppimistavoitteisiin tarkasti HM:n kanssa. 

\begin{enumerate}[label=\bfseries \arabic*)]
\item  Stabiilin vapaapudotusasennon löytäminen viimeistään 10 sekuntia ennen avauskorkeutta 
\item  Korkeuden tarkkailu 
\item  Avausmerkki 1600 m (±300m) ja avaus 
\end{enumerate}
\subsection{ Hyppylennolla }
\label{nova-alkeiskoulutuksen-suoritukset-hyppylennolla}

\begin{itemize}
\item  Katso laskeutumisalue ja oikea laskeutumissuunta yli 600 m:n korkeudessa. Hyppymestari kysyy tärkeitä korkeuksia. 
\item  Hyppymestari kysyy hypyn avainkohdat. 
\end{itemize}
\subsection{ Hypyn kulku }
\label{nova-alkeiskoulutuksen-suoritukset-hypyn-kulku}

\begin{tabular}[]{|l|p{4.7cm}|}
\hline
 \textbf{2700-4000 m} &  Rento taivutettu UH
\\ \hline
  &  Tarkkailukehä
\\ \hline
  &  3 Harjoitusvetoa
\\ \hline
  &  Tarkkailukehä
\\ \hline
  &  MAA - MITTARI
\\ \hline
 \textbf{1600 m} &  Avausmerkki ja avaus
\\ \hline
\end{tabular}

\end{multicols}\pagebreak\begin{multicols}{2} 

\section{ Taso 2 - Asennon hallinta }
\label{nova-alkeiskoulutuksen-suoritukset-taso-2-asennon-hallinta}


Kaksi novahyppymestaria, kolmella ensimmäisellä hypyllä radio.   Oppilaan on saavutettava välttämättömät oppimistavoitteet hyväksytysti päästäkseen seuraavalle tasolle. 

\subsection{ Oppilaan toiminta. }
\label{nova-alkeiskoulutuksen-suoritukset-oppilaan-toiminta}

\begin{enumerate}[label=\bfseries \arabic*)]
\item  Käy läpi I-tasolla opitut tiedot ja taidot. 
\item  Harjoittele stabiilia asentoa esim. rullalaudoilla. 
\item  Harjoittele kaikkia vapaassa tehtäviä liikkeitä kuten aikaisemmilla tunneilla. 
\item  Harjoittele harjoitusvetoa kuten oikeaa avausta. 
\item  Näytä maalialue ja käy läpi ennalta suunniteltu ohjauskuvio. 
\item  Käytä mielikuvaharjoittelua valmistautuessasi hyppyyn. 
\item  Osallistu itse aikaisempaa enemmän varusteiden tarkistukseen ja sovittamiseen. 
\end{enumerate}
\subsection{ Oppimistavoitteet }
\label{nova-alkeiskoulutuksen-suoritukset-oppimistavoitteet}

\begin{enumerate}[label=\bfseries \arabic*)]
\item  Stabiili vapaapudotusasento viimeistään 10 sekuntia UH:n jälkeen 
\item  Stabiilin vapaapudotusasennon säilyttäminen koko hypyn ajan, sisältäen jalkojen hallinnan 
\item  Avausmerkki 1600 m (±150 m) ja itsenäinen avaus  
\item  Itsenäisempi, turvallinen varjon ohjailu 
\end{enumerate}
\subsection{ Hyppylennolla }
\label{nova-alkeiskoulutuksen-suoritukset-hyppylennolla}

\begin{enumerate}[label=\bfseries \arabic*)]
\item  Toista kaikki aikaisemmin opetetut toimenpiteet lennon aikana. 
\item  Käytä rentouttavaa hengitystekniikkaa (hengitä hitaasti sisään, pidätä hetkisen ja rentoudu uloshengittäessäsi). 
\item  Näytä maamerkkejä ja osoita avauskorkeus hyppymestareille. 
\end{enumerate}
\subsection{ Hypyn kulku }
\label{nova-alkeiskoulutuksen-suoritukset-hypyn-kulku}

\begin{tabular}[]{|l|p{4.7cm}|}
\hline
 \textbf{2700-4000 m} &  Rento, taivutettu uloshyppy
\\ \hline
  &  Tarkkailukehä
\\ \hline
  &  Harjoitusvetoja kunnes OK
\\ \hline
  &  MAA, MITTARI, TAIVUTA, JALAT, RENTOUTA
\\ \hline
  &  Liike eteenpäin
\\ \hline
  &  MAA, MITTARI, TAIVUTA, JALAT, RENTOUTA
\\ \hline
  &  90° vasen
\\ \hline
  &  MAA, MITTARI, TAIVUTA, JALAT, RENTOUTA
\\ \hline
  &  90° oikea
\\ \hline
  &  MAA, MITTARI, TAIVUTA, JALAT, RENTOUTA
\\ \hline
 \textbf{1600 m} &  Avausmerkki ja avaus
\\ \hline
\end{tabular}

\end{multicols}\pagebreak\begin{multicols}{2} 

\section{ Taso 3 - Stabiili vapaapudotus }
\label{nova-alkeiskoulutuksen-suoritukset-taso-3-stabiili-vapaapudotus}


Kaksi novahyppymestaria, kolmella ensimmäisellä hypyllä radio. 


Oppilaan on saavutettava välttämättömät oppimistavoitteet hyväksytysti päästäkseen seuraavalle tasolle. 


Jos tasoilla 3-7 et näe kumpaakaan hyppymestaria, voit jatkaa hyppyä, mikäli kaikki alla olevat ehdot täyttyvät: 

\begin{itemize}
\item  Tiedät korkeutesi  
\item  Et pyöri mihinkään suuntaan 
\item  Et ole selälläsi 
\item  Tilanne tuntuu kaiken kaikkiaan olevan hallinnassa 
\end{itemize}

Tarkkaile korkeutta koko ajan n. 5 s välein (MAA, MITTARI, TAIVUTA, JALAT, RENTOUTA) ja tason mukaisessa korkeudessa tee avausmerkki ja avaa päävarjo normaalisti. Älä tee muita suorituksia.  


Jos et avauksessa löydä päävarjon kahvaa, yritä toisen kerran, mutta jos et \textbf{heti} löydä kahvaa, tee varavarjotoimenpiteet. 


Jos et avauksessa jaksa vetää päävarjon kahvasta (tiukka kahva), yritä toisen kerran, mutta jos et silloinkaan \textbf{heti} saa varjoa auki, tee varavarjotoimenpiteet. 

\subsection{ Oppilaan toiminta. }
\label{nova-alkeiskoulutuksen-suoritukset-oppilaan-toiminta}

\begin{enumerate}[label=\bfseries \arabic*)]
\item  Kuvaile ja esitä käytännössä tiedot ja taidot, jotka on opittu aikaisempien hyppyjen yhteydessä. 
\item  Harjoitteleilmatilan tarkistusta HM:n avustuksella. 
\item  Harjoittele maassa tekniikkaa jolla käännyt tarvittaessa selältään oikein päin. 
\item  Näytä millä tekniikalla suunta säilytetään vapaapudotuksessa. 
\item  Keskity hyppyyn ja rentoudu. Mieti, millainen on hyvä stabiili asento ja harjoittele sitä. 
\item  Käy läpi hypyn kulku ja harjoittele sitä käytännössä. 
\end{enumerate}
\subsection{ Oppimistavoitteet }
\label{nova-alkeiskoulutuksen-suoritukset-oppimistavoitteet}

\begin{enumerate}[label=\bfseries \arabic*)]
\item  Ilmatilan tarkistus ennen uloshyppyä 
\item  Stabiili itsenäinen vapaapudotus viimeistään 5 sekuntia UH:n jälkeen 
\item  Tarkoituksettomien käännösten pysäyttäminen 
\item  Avausmerkki 1600 m (±150m), stabiili itsenäinen avustamaton avaus ilman NHM:n otetta oppilaasta 
\item  Itsenäinen, turvallinen varjon ohjailu 
\end{enumerate}
\subsection{ Hyppylennolla }
\label{nova-alkeiskoulutuksen-suoritukset-hyppylennolla}

\begin{enumerate}[label=\bfseries \arabic*)]
\item  Toista aiemmin opitut toimenpiteet. 
\item  Käytä rentouttavaa hengitystekniikkaa ja positiivisia mielikuvia. 
\item  Katso HM:n johdolla lentokenttää ja ilmatilaa ovella ennen uloshyppyä. 
\end{enumerate}
\subsection{ Ohjailu ja laskeutuminen }
\label{nova-alkeiskoulutuksen-suoritukset-ohjailu-ja-laskeutuminen}

\begin{enumerate}[label=\bfseries \arabic*)]
\item  Toimi kuten edellisillä hypyillä, mutta tee kaikki entistä tarkemmin. 
\item  Seisontalaskua saa yrittää (mutta pidä aina hyvä alastuloasento). 
\end{enumerate}
\subsection{ Hypyn kulku }
\label{nova-alkeiskoulutuksen-suoritukset-hypyn-kulku}

\begin{tabular}[]{|l|p{4.7cm}|}
\hline
 \textbf{2700-4000 m} &  Rento taivutettu UH
\\ \hline
  &  Tarkkailukehä
\\ \hline
  &  Harjoitusvetoja kunnes OK
\\ \hline
  &  Stabiili itsenäinen vapaapudotus
\\ \hline
  &  MAA, MITTARI, TAIVUTA, JALAT, RENTOUTA
\\ \hline
 \textbf{1600 m} &  Avausmerkki ja avaus
\\ \hline
\end{tabular}

\end{multicols}\pagebreak\begin{multicols}{2} 

\section{ Taso 4 - Käännökset 90° }
\label{nova-alkeiskoulutuksen-suoritukset-taso-4-kaannokset-90deg}


Yksi novahyppymestari. 


Oppilaan on saavutettava välttämättömät oppimistavoitteet hyväksytysti päästäkseen seuraavalle tasolle. 

\subsection{ Oppilaan toiminta. }
\label{nova-alkeiskoulutuksen-suoritukset-oppilaan-toiminta}

\begin{enumerate}[label=\bfseries \arabic*)]
\item  Osaat tehdä varusteiden tarkastuksen maassa ja ennen koneelle menoa. (\ref{laskuvarjokalusto-ja-hyppyvarusteet-3x3-tarkastus} s.\pageref{laskuvarjokalusto-ja-hyppyvarusteet-3x3-tarkastus}) 
\item  Harjoittele koko hyppytapahtuman kulku käytännössä. 
\item  Tarkista ilmatila ja pilvet ennen uloshyppyä ja huomioi riittävä exit-väli edelliseen ryhmään. 
\item  Tarkista varusteet ennen UH:ta (\ref{laskuvarjokalusto-ja-hyppyvarusteet-3x3-tarkastus} s.\pageref{laskuvarjokalusto-ja-hyppyvarusteet-3x3-tarkastus}) 
\end{enumerate}
\subsection{ Oppimistavoitteet }
\label{nova-alkeiskoulutuksen-suoritukset-oppimistavoitteet}

\begin{enumerate}[label=\bfseries \arabic*)]
\item  Ilmatilan ja pilvien tarkistus ja exit-väli  
\item  Omien varusteiden tarkastus (maassa ja koneessa) 
\item  Stabiili vapaapudotusasento 
\item  Vähintään kaksi hallittua 90° käännöstä (±20°) 
\item  Avausmerkki 1500 m (±150 m), stabiili itsenäinen avustamaton avaus  
\end{enumerate}
\subsection{ Hyppylennolla }
\label{nova-alkeiskoulutuksen-suoritukset-hyppylennolla}

\begin{enumerate}[label=\bfseries \arabic*)]
\item  Toimi lennon aikana aikaisemmin opetettujen toimenpiteiden mukaisesti. 
\item  Tarkista ilmatila ja pilvet sekä huomioi riittävä exit-väli edelliseen ryhmään. 
\end{enumerate}
\subsection{ Ohjailu ja laskeutuminen }
\label{nova-alkeiskoulutuksen-suoritukset-ohjailu-ja-laskeutuminen}

\begin{enumerate}[label=\bfseries \arabic*)]
\item  Toimi aiemmin opitulla tavalla. 
\item  Pyri laskeutumaan korkeintaan 50 m päähän kohteesta mahdollisimman vähin ohjein. 
\end{enumerate}
\subsection{ Hypyn kulku }
\label{nova-alkeiskoulutuksen-suoritukset-hypyn-kulku}

\begin{tabular}[]{|l|p{4.7cm}|}
\hline
 \textbf{2700-4000 m} &  Rento taivutettu UH
\\ \hline
  &  Tarkkailukehä (HM siirtyy oppilaan eteen)
\\ \hline
  &  Lupa käännökseen (oppilas nyökkää HM:lle, HM vastaa nyökkäämällä)
\\ \hline
  &  90° vasen
\\ \hline
  &  MAA, MITTARI, TAIVUTA, JALAT, RENTOUTA
\\ \hline
  &  90° oikea
\\ \hline
  &  MAA, MITTARI, TAIVUTA, JALAT, RENTOUTA
\\ \hline
 \textbf{2000 m} &  Ei käännöksiä
\\ \hline
  &  MAA, MITTARI, TAIVUTA, JALAT, RENTOUTA
\\ \hline
 \textbf{1500 m} &  Avausmerkki ja avaus
\\ \hline
\end{tabular}

\end{multicols}\pagebreak\begin{multicols}{2} 

\section{ Taso 5 - Käännökset 360° }
\label{nova-alkeiskoulutuksen-suoritukset-taso-5-kaannokset-360deg}


Yksi novahyppymestari. 


Oppilaan on saavutettava välttämättömät oppimistavoitteet hyväksytysti päästäkseen seuraavalle tasolle. 

\subsection{ Oppilaan toiminta. }
\label{nova-alkeiskoulutuksen-suoritukset-oppilaan-toiminta}

\begin{enumerate}[label=\bfseries \arabic*)]
\item  Näytä, että osaat oikeaoppisesti tarkastaa ja säätää varusteet. 
\item  Harjoittele koko hyppytapahtuman kulku käytännössä. 
\item  Tarkistat ilmatilan, pilvet ja huomioit exit-välin  
\item  Tarkastat automaattisesti itse varusteet ennen uloshyppyä. (\ref{laskuvarjokalusto-ja-hyppyvarusteet-3x3-tarkastus} s.\pageref{laskuvarjokalusto-ja-hyppyvarusteet-3x3-tarkastus}) 
\end{enumerate}
\subsection{ Oppimistavoitteet }
\label{nova-alkeiskoulutuksen-suoritukset-oppimistavoitteet}

\begin{enumerate}[label=\bfseries \arabic*)]
\item  Vähintään kaksi hallittua 360° käännöstä (±45°) 
\item  Yhä stabiilimpi vapaapudotus  
\item  Avausmerkki 1500 m (±150 m), stabiili suunnassa pysyvä itsenäinen avustamaton avaus   
\item  Tee varjon varassa 90° käännöksiä takimmaisista kantohihnoista puolijarrut kiinni sekä avattuina 
\end{enumerate}
\subsection{ Hyppylennolla }
\label{nova-alkeiskoulutuksen-suoritukset-hyppylennolla}

\begin{enumerate}[label=\bfseries \arabic*)]
\item  Tarkistat ilmatilan, pilvet ja huomioit exit-välin. 
\item  Tarkastat itse varusteet ennen uloshyppyä 
\end{enumerate}
\subsection{ Ohjailu ja laskeutuminen }
\label{nova-alkeiskoulutuksen-suoritukset-ohjailu-ja-laskeutuminen}

\begin{enumerate}[label=\bfseries \arabic*)]
\item  Tee 90° käännöksiä kantohihnoista puolijarrut kiinni sekä avattuina. 
\item  Pyri laskeutumaan enintään 50 metrin päähän kohteesta. 
\item  Muista oikea lähestymis- ja laskeutumistekniikka. 
\end{enumerate}
\subsection{ Hypyn kulku }
\label{nova-alkeiskoulutuksen-suoritukset-hypyn-kulku}

\begin{tabular}[]{|l|p{4.7cm}|}
\hline
 \textbf{2700-4000 m} &  Rento taivutettu UH
\\ \hline
  &  Tarkkailukehä (HM siirtyy oppilaan eteen)
\\ \hline
  &  Lupa käännökseen (oppilas nyökkää HM:lle, HM vastaa nyökkäämällä)
\\ \hline
  &  360° vasen
\\ \hline
  &  MAA, MITTARI, TAIVUTA, JALAT, RENTOUTA
\\ \hline
  &  360° oikea
\\ \hline
  &  MAA, MITTARI, TAIVUTA, JALAT, RENTOUTA
\\ \hline
 \textbf{2000 m} &  Ei käännöksiä
\\ \hline
  &  MAA, MITTARI, TAIVUTA, JALAT, RENTOUTA
\\ \hline
 \textbf{1500 m} &  Avausmerkki ja avaus
\\ \hline
\end{tabular}

\end{multicols}\pagebreak\begin{multicols}{2} 

\section{ Taso 6 - Irtiuloshyppy }
\label{nova-alkeiskoulutuksen-suoritukset-taso-6-irtiuloshyppy}


Yksi novahyppymestari. 


Oppilaan on saavutettava välttämättömät oppimistavoitteet hyväksytysti päästäkseen seuraavalle tasolle. 

\subsection{ Oppilaan toiminta. }
\label{nova-alkeiskoulutuksen-suoritukset-oppilaan-toiminta}

\begin{enumerate}[label=\bfseries \arabic*)]
\item  Toista aikaisemmilla tasoilla opitut asiat 
\item  Harjoittele suoran uloshypyn asentoa ja tekniikkaa 
\item  Harjoittele takavoltin tekniikkaa ja stabiloimista. 
\item  Delta-asento ja liukuminen. 
\end{enumerate}
\subsection{ Oppimistavoitteet }
\label{nova-alkeiskoulutuksen-suoritukset-oppimistavoitteet}

\begin{enumerate}[label=\bfseries \arabic*)]
\item  Itsenäinen, avustamaton uloshyppy 
\item  Asennon stabilointi epästabiilista viidessä sekunnissa 
\item  Liuku (suunta säilyttäen) 
\item  Avausmerkki 1300 m (±150 m) ja avaus  
\item  Oikeaoppiset lähestymis- ja laskeutumiskuviot 
\end{enumerate}
\subsection{ Hyppylennolla }
\label{nova-alkeiskoulutuksen-suoritukset-hyppylennolla}

\begin{enumerate}[label=\bfseries \arabic*)]
\item  Käy uudelleen lävitse toiminta kuten kaikilla aikaisemmilla tasoilla. 
\item  Ilmatilan tarkistus, pilvet, exit-välin huomioiminen  
\item  Tee stabiili, suora uloshyppy ilman HM:n apua 
\end{enumerate}
\subsection{ Ohjailu ja laskeutuminen }
\label{nova-alkeiskoulutuksen-suoritukset-ohjailu-ja-laskeutuminen}

\begin{enumerate}[label=\bfseries \arabic*)]
\item  Pyri laskeutumaan korkeintaan 25 m:n päähän määrätyn alastuloalueen keskustasta 
\item  Tee oikeaoppiset lähestymis- ja laskeutumiskuviot 
\end{enumerate}
\subsection{ Hypyn kulku }
\label{nova-alkeiskoulutuksen-suoritukset-hypyn-kulku}

\begin{tabular}[]{|l|p{4.7cm}|}
\hline
 \textbf{2700-4000 m} &  Suora uloshyppy 
\\ \hline
  &  MAA, MITTARI, TAIVUTA, JALAT, RENTOUTA
\\ \hline
  &  Tynnyri
\\ \hline
  &  MAA, MITTARI, TAIVUTA, JALAT, RENTOUTA
\\ \hline
  &  Takavoltti
\\ \hline
  &  MAA, MITTARI, TAIVUTA, JALAT, RENTOUTA
\\ \hline
 \textbf{Yli 2000 m} &  Lentokentän paikallistaminen ja liuku
\\ \hline
  &  MAA, MITTARI, TAIVUTA, JALAT, RENTOUTA
\\ \hline
 \textbf{1300 m} &  Avausmerkki ja avaus
\\ \hline
\end{tabular}

\end{multicols}\pagebreak\begin{multicols}{2} 

\section{ Taso 7 - Puolisarja }
\label{nova-alkeiskoulutuksen-suoritukset-taso-7-puolisarja}


Yksi novahyppymestari. 


Tämä on viimeinen tasohyppysi. Tällä tasolla sinun tulisi toimia mahdollisimman itsenäisesti ja osoittaa hallitsevasi hyppäämiseen liittyvät turvallisuustekijät, kuten stabiili vapaapudotus ja varjon avaus, sekä ohjaaminen. 

\subsection{ Oppilaan toiminta. }
\label{nova-alkeiskoulutuksen-suoritukset-oppilaan-toiminta}

\begin{enumerate}[label=\bfseries \arabic*)]
\item  Harjoittele sukellusuloshypyn asentoa ja tekniikkaa. 
\item  Harjoittele etuvoltin tekniikkaa. 
\item  Harjoittele vaakasuunnassa liikkuvaa liukua. 
\end{enumerate}
\subsection{ Oppimistavoitteet }
\label{nova-alkeiskoulutuksen-suoritukset-oppimistavoitteet}

\begin{enumerate}[label=\bfseries \arabic*)]
\item  Tarkista ilmatila, pilvet, huomioi exit-väli 
\item  Sukellusuloshyppy josta stabiili asento viidessä sekunnissa 
\item  Asennon stabilointi suoritusten aikana viidessä sekunnissa 
\item  Avausmerkki 1300 m (±150 m) ja avaus  
\item  Turvallinen varjon käsittely 
\end{enumerate}
\subsection{ Hyppylennolla }
\label{nova-alkeiskoulutuksen-suoritukset-hyppylennolla}

\begin{enumerate}[label=\bfseries \arabic*)]
\item  Tarkista ilmatila, pilvet ja huomioi exit-väli  
\item  Tarkista omat varusteesi 
\end{enumerate}
\subsection{ Ohjailu ja laskeutuminen }
\label{nova-alkeiskoulutuksen-suoritukset-ohjailu-ja-laskeutuminen}

\begin{enumerate}[label=\bfseries \arabic*)]
\item  Pyri laskeutumaan korkeintaan 25 metrin päähän määrätyn alastuloalueen keskustasta 
\item  Näytä, miten ohjaat varjoasi turvallisesti  
\end{enumerate}
\subsection{ Hypyn kulku }
\label{nova-alkeiskoulutuksen-suoritukset-hypyn-kulku}

\begin{tabular}[]{|l|p{4.7cm}|}
\hline
 \textbf{2700-4000 m} &  Sukellusuloshyppy 
\\ \hline
  &  MAA, MITTARI, TAIVUTA, JALAT, RENTOUTA
\\ \hline
  &  Etuvoltti
\\ \hline
  &  MAA, MITTARI, TAIVUTA, JALAT, RENTOUTA
\\ \hline
  &  Takavoltti
\\ \hline
  &  MAA, MITTARI, TAIVUTA, JALAT, RENTOUTA
\\ \hline
  &  360° Vasen 
/ Oikea 


\\ \hline
 \textbf{Yli 2000 m} &  Liuku
\\ \hline
  &  MAA, MITTARI, TAIVUTA, JALAT, RENTOUTA
\\ \hline
 \textbf{1300 m} &  Avausmerkki ja avaus
\\ \hline
\end{tabular}

\end{multicols}\pagebreak\begin{multicols}{2} 

\section{ 15'' - Lyhyt vapaa }
\label{nova-alkeiskoulutuksen-suoritukset-15-lyhyt-vapaa}


Yksi novahyppymestari koneessa. 


Tämä on alkeiskoulutuksen viimeinen  hyppy ja ensimmäinen hyppysi matalammasta hyppykorkeudesta. Hyppymestari ei enää hyppää mukanasi. 

\subsection{ Oppilaan toiminta. }
\label{nova-alkeiskoulutuksen-suoritukset-oppilaan-toiminta}


Toista tasokoulutuksessa oppimasi asiat: irtiuloshyppy eli suora uloshyppy, asennon stabilointi uloshypyn jälkeen, korkeuden tarkkailu ja stabiili avaus. 

\subsection{ Oppimistavoitteet }
\label{nova-alkeiskoulutuksen-suoritukset-oppimistavoitteet}

\begin{enumerate}[label=\bfseries \arabic*)]
\item  Tarkista ilmatila, pilvet ja huomioi exit-väli  
\item  Suora uloshyppy 
\item  15'' vapaapudotus koko ajan korkeutta tarkkaillen 
\item  Avausmerkki 1200 m ja avaus 
\end{enumerate}
\subsection{ Hyppylennolla }
\label{nova-alkeiskoulutuksen-suoritukset-hyppylennolla}

\begin{enumerate}[label=\bfseries \arabic*)]
\item  Tarkista omat varusteesi 
\item  Tarkista ilmatila, pilvet ja huomioi exit-väli  
\end{enumerate}
\subsection{ Ohjailu ja laskeutuminen }
\label{nova-alkeiskoulutuksen-suoritukset-ohjailu-ja-laskeutuminen}

\begin{enumerate}[label=\bfseries \arabic*)]
\item  Pyri laskeutumaan korkeintaan 25 metrin päähän määrätyn alastuloalueen keskustasta 
\item  Näytä, miten ohjaat varjoasi turvallisesti  
\end{enumerate}
\subsection{ Hypyn kulku }
\label{nova-alkeiskoulutuksen-suoritukset-hypyn-kulku}

\begin{tabular}[]{|l|p{4.7cm}|}
\hline
 \textbf{1800-2500 m} &  Suora uloshyppy
\\ \hline
  &  Mittari
\\ \hline
 \textbf{1200 m} &  Avausmerkki ja avaus
\\ \hline
\end{tabular}
