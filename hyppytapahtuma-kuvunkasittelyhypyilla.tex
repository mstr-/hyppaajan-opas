
Hyppytapahtumaan kuuluu oleellisena osana varjon varassa lentäminen ja laskeutuminen. Seuraavassa on esimerkkejä siitä, mitä pitää huomioida hypyllä erityisesti kuvunkäsittelyn kannalta. 

\section{ Olosuhteet }
\label{hyppytapahtuma-kuvunkasittelyhypyilla-olosuhteet}


Tarkista aina ennen hyppyä vallitsevat sääolosuhteet. Huomioi tuulen suunta ja mahdollinen turbulenttinen keli sekä niiden vaikutus laskeutumisalueeseen ja varalaskeutumispaikkoihin. Selvitä uudella hyppypaikalla paikalliset erityisolosuhteet ja laskeutumiseen vaikuttavat esteet, kuten rakennukset ja sähkölinjat. Älä lähde hyppäämään näytöshypylle tarkastamatta laskeutumisaluetta.  

\section{ Valmistautuminen }
\label{hyppytapahtuma-kuvunkasittelyhypyilla-valmistautuminen}


Älä koskaan kiirehdi hyppytoiminnassa, koska silloin unohtuu ja saattaa jäädä tekemättä tärkeitä asioita. Jo varjoa pakatessasi vaikutat sen avautumiseen ja sitä kautta hypyn onnistumiseen. Hyppyyn valmistautumiseen kuuluu myös varjolla lentämisen suunnittelu. Mieti muiden samassa pokassa olevien hyppääjien varjojen ominaisuuksia ja heidän tapaansa käsitellä varjoaan. Sopikaa yhdessä pokan vanhimman johdolla linjan hyppyjärjestys, avauskorkeudet, porrastukset sekä laskeutumisalueet ja -suunta. Näin voitte ennakoida mahdolliset tapahtumat varjon varassa. Pukiessasi varjoa tarkasta, että jalkahihnat tulevat yhtä kireälle, jotta asentosi valjaissa on symmetrinen eikä käännä varjoa avauksessa tai varjolla lentämisen aikana. 

\section{ Toiminta lentokoneessa ja uloshyppy }
\label{hyppytapahtuma-kuvunkasittelyhypyilla-toiminta-lentokoneessa-ja-uloshyppy}


Harjoittele nousun aikana ja mielikuvaharjoittelua tehdessäsi myös kuvunkäsittelyosuutta laskeutumisineen. Ylätuulien voimakkuus kannattaa vielä tarkastaa seuraamalla lentokoneen nopeutta maahan nähden. Varmista oikea uloshyppypaikka vaikka hyppäisitkin muiden perään. Älä hyppää, jos et varmasti tiedä sijaintiasi ja ole varma pääsystäsi laskeutumisalueelle. Ota tarvittaessa uusi hyppylinja. 

\section{ Vapaapudotus ja varjon avaus }
\label{hyppytapahtuma-kuvunkasittelyhypyilla-vapaapudotus-ja-varjon-avaus}


Tarkista vapaapudotuksen aikana sijaintisi laskeutumisalueeseen nähden. Avaa tarvittaessa varjosi hieman korkeammalla, jotta varmistat pääsysi oikeaan laskeutumispaikkaan. Huomioi muut hyppääjät pitämällä riittävät exit-välit, sillä se mahdollistaa tarvittaessa hypyn purkamisen ja varjon avaamisen turvallisesti hieman korkeammallakin, jos se on välttämätöntä laskeutumisalueelle pääsemiseksi. Tee hyvä liuku, jotta vaakaetäisyys muihin on riittävä. Näytä avausmerkki ja avaa varjo stabiilista avausasennosta. Apuvarjon katsominen kallistaa avausasentoa. Seurauksena on varjon kehittyminen kallistuneena, jolloin kupu kääntyy kallistuksen suuntaan tai voi tehdä kierteitä. Erityisesti luokkien 4–6 varjot ovat hyvin herkkiä vajaatoiminnoille epätasaisesta (kallistuneesta tai muuten epästabiilista) asennoista avattuina. Varjon avauduttua tarkasta ilmatila ja väistä tarvittaessa oikealle takakantohihnoista vetämällä. Tämän jälkeen tukahduta slider ja avaa puolijarrut. 

\section{ Toiminta varjon varassa }
\label{hyppytapahtuma-kuvunkasittelyhypyilla-toiminta-varjon-varassa}


Varjon täytyy olla täysin laskeutumiskuntoinen 600 metrin korkeudessa. Myös puolijarrujen tulee olla tuolloin avattuna. Seuraa muuta liikennettä suhteessa 90 \% ilmatilaa ja 10 \% maalialuetta. Huomioi muut hyppääjät ja porrasta laskeutumiset. Jos sinulla on pieni ja nopea varjo ja olet matalalla muihin nähden, nopeuta laskeutumistasi. Jos taas olet muiden yläpuolella, seuraa sopivaa rakoa liikenteessä. Jos sinulla on suuri ja hidas varjo ja olet matalalla, mieti missä haluat nopeitten varjojen ohittavan sinut. Tämän tulisi tapahtua mieluummin myötätuuliosalla kuin finaalissa. 

\section{ Tarkkuusniksi }
\label{hyppytapahtuma-kuvunkasittelyhypyilla-tarkkuusniksi}


\textbf{Tarkkuusniksi} on yksinkertainen tapa selvittää, pääsetkö varjollasi laskeutumisalueelle vai kannattaako tehdä päätös laskeutumisesta jonnekin muualle. Etsi ensin maastosta etusektorista se piste, joka ei näytä siirtyvän alapuolellesi eikä nousevan ylemmäksi. Tämä paikallaan pysyvä piste on se paikka, johon päädyt, jos pidät varjosi lentotilan samana. Testaa harjoitusmielessä omalla varjollasi mihin tarkkuuspiste siirtyy, kun myötä- tai vastatuulessa  

\begin{enumerate}[label=\bfseries \arabic*)]
\item  Asteittain jarruttaen hidastat vajoamista ja vaakanopeutta,  
\item  Vedät takakantohihnoista hieman alaspäin loiventaen varjon liitokulmaa tai  
\item  Vedät etukantohihnoista alaspäin jyrkentäen varjon liitokulmaa. Näin tiedät, kuinka sinun kannattaa lentää varjoasi esim. pitkältä spotilta maalialuetta kohti lentäessäsi tai joutuessasi alatuulen puolelle. 
\end{enumerate}

Jotkut varjot lentävät pidemmälle jarruja, jotkut taas takakantohihnoja, käyttäen. Kantohihnoilla toleranssi on vain muutama sentti. Kädet ovat korkealla, minkä vuoksi verenkierto käsissä on huono. Kantohihnoista vetäminen vaikuttaa C- ja D-punoksiin ja muuttaa koko kuvun muotoa. Siksi voimaa tarvitaan melko paljon. Älä purista kantohihnoista, vaan työnnä sormet punosten läpi connectorien yläpuolelta. Voit myös koittaa kantohihnojen levittämistä. 


Jarruilla toleranssi on suurempi. Kädet ovat sydämen korkeudella, joten verenkierto on käsille parempi. Käsien ollessa ohjauslenkeissä kannattaa voimien säästämiseksi ottaa ote lisäksi valjaista. Kuvusta riippuen jopa 60 \% jarrut saattavat olla paras asetus matkalentoa ajatellen. Jos on tiedossa pitkä matkalento, kannattaa puolijarrut jättää kiinni. Se helpottaa käsien lepuuttamista. Muista, että varjon pitää olla lentokuntoinen 600 metrin korkeudessa ja että myös puolijarrujen täytyy olla viimeistään tuolloin avattuina! 

\section{ Laskeutuminen }
\label{hyppytapahtuma-kuvunkasittelyhypyilla-laskeutuminen}


Tee päätös laskeutumispaikasta viimeistään 300 metrin korkeudessa ja keskity sen jälkeen vain laskeutumiseen. Jos uloshyppypaikka on huono, käytä tarkkuusniksiä ja valitse varalaskeutumispaikka ajoissa. Sovi ja suunnittele laskeutumiskuvio etukäteen esteet, muu liikenne ja sääolosuhteet huomioiden. Vältä laskeutumista samaan paikkaan ja yhtä aikaa muiden kanssa. On parempi kävellä pakkausalueelle 100 metriä omin jaloin kuin nousta ambulanssin kyytiin pakkausalueen reunalta. Jos tarkoituksenasi on tehdä vauhdikas laskeutuminen, niin tee se kaukana muista ja tarpeeksi korkealla tuntien varjosi ominaisuudet. 

\section{ Viimeiset metrit }
\label{hyppytapahtuma-kuvunkasittelyhypyilla-viimeiset-metrit}


Älä muuta laskeutumissuunnitelmaasi viime hetkellä, sillä siitä johtuu suuri osa onnettomuuksista. Opettele hyvällä kelillä eri laskeutumistekniikoita. (\ref{laskeutumistekniikat} s.\pageref{laskeutumistekniikat}) Etsi omat ja varjosi rajat, niin osaat tiukassakin paikassa laskeutua oikein. Vältä käyttämästä aina samaa laskeutumistekniikkaa tai samanpuolista laskeutumiskuviota (ellei kuvion suunta ole hyppypaikallasi määrätty). 

