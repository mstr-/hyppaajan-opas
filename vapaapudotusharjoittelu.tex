\section{ 2-way -hyppääminen }
\label{vapaapudotusharjoittelu-2-way-hyppaaminen}


Yhdistelemällä erilaisia 2-way kuvia voi suunnitella hyvin monipuolisia ja erilaisia hyppyjä. Vaihtoehtoisesti joko työskennellään vuorotellen, jolloin paikallaan oleva voi harjoitella asentoa ja toimia samalla kiintopisteenä toiselle tai molemmat työskentelevät yhtä aikaa. Varsinkin aloiteltaessa FS-hyppäämistä kannattaa hypyt suunnitella niin, että työskentely tapahtuu vuorotellen. Taitojen kehittyessä hyppyjä voi vaikeuttaa lisäämällä liikkeiden ja muodostelmien määrää, käännösten astelukua ja suuntia.  


Oppilaan FS-hyppyohjelma (\ref{fs-kuviohyppaaminen-hyppyohjelma} s.\pageref{fs-kuviohyppaaminen-hyppyohjelma}) kannattaa hypätä läpi varsinkin, jos molemmat hyppääjät ovat kokemattomia FS-hyppääjiä, koska kyseisten hyppyjen harjoittelu antaa hyvät perustaidot, jotka pätevät kaikkeen FS-hyppäämiseen. Luvusta \textit{Tuoreen lisenssihyppääjän FS-ohjelma} (\ref{tuoreen-lisenssihyppaajan-fs-ohjelma} s.\pageref{tuoreen-lisenssihyppaajan-fs-ohjelma}) löydät lisää harjoituksia. 

\section{ Purkaminen }
\label{vapaapudotusharjoittelu-purkaminen}


Purkukorkeus sovitaan aina ennen hyppyä, ja sen tulee olla riittävä hyppääjien kokemukseen ja hypylle osallistuvien lukumäärään nähden. Purkukorkeuden tulee antaa jokaiselle aikaa liukua, jotta saadaan hyppääjien välille riittävästi etäisyyttä ja tilaa turvalliseen avaukseen. Aluksi kokemattomien kannattaa asettaa purkukorkeus ylemmäs (1400 - 1500 metriä), mutta kokemuksen myötä sen voi laskea 1200 metriin. 


Vastuu korkeuden tarkkailusta ja purkamisesta kuuluu kaikille ja sovitussa korkeudessa näytetään purkumerkki. Purkumerkki näytetään heilauttamalla kädet ristiin. Äänikorkeusmittariin voi asentaa hälytyksen muistuttamaan purkukorkeudesta. 

\section{ Liuku }
\label{vapaapudotusharjoittelu-liuku}


Purkumerkin jälkeen käännytään poispäin muista hyppääjistä ja liu'utaan vapaaseen suuntaan. Liuku pyritään suorittamaan mahdollisimman pienessä kulmassa vaakatasoon nähden, jotta saadaan aikaiseksi riittävän suuret etäisyydet muihin hyppääjiin pienimmällä mahdollisella korkeuden menetyksellä. 


Kantava liukuasento: 

\begin{itemize}
\item  oikaistaan jalat suoriksi ja tuodaan ne lähemmäksi toisiaan  
\item  viedään kädet sivuille lähelle vartaloa kämmenet alaspäin  
\item  suoristetaan vartalo ja painetaan olkapäitä alas  
\item  leuka painetaan rintaan  
\item  asento on jäykkä  
\item  kämmenillä ohjataan liu'un suuntaa.  
\end{itemize}

Liukua jatketaan kunnes ilmatila on vapaa turvalliseen avaukseen. Liuku pysäytetään palaamalla perusasentoon. Pysäytyksen jälkeen tarkastetaan ilmatila sekä ylhäältä että alhaalta ja näytetään selkeä avausmerkki. Avausmerkki näytetään heilauttamalla käsiä ristiin pari kertaa. Alempana olevalla on aina oikeus avata ensin. 

