\section{ Asiakirjojen tarkastus }
\label{kaluston-tarkastus-ja-huolto-asiakirjojen-tarkastus}


Ostettavan varjokokonaisuuden laitekortit on aina tarkastettava huolellisesti. Apua laitekorttien tarkastukseen kannattaa pyytää kerhon kalustohenkilöiltä tai hyppymestareilta. Laitekortit voivat olla reppu-valjasyhdistelmässä tai varjoissa uudelleen avattuja. Tällaisissa tapauksissa laitekorteista ei selviä todellinen käyttö- ja huoltohistoria. Käyttö- ja huoltohistorian ollessa epäselvä on syytä suhtautua ostamiseen terveellä epäluulolla. 


Laitekortista on tarkastettava aina seuraavat kohdat ja niitä on verrattava kalustoon: 

\begin{itemize}
\item  Varjon ja valjaiden tyypin, sarjanumeroiden ja valmistusmerkintöjen eli tunnistetietojen on täsmättävä laitekortissa oleviin. Kupujen tunnistetiedot löytyvät kuvun takahelmassa olevasta varoituslapusta. Valjaissa tunnistetiedot on yleensä merkitty joko varavarjon kantohihnassa tai repun niskassa ommeltuna olevaan merkkiin. Merkin sijainti vaihtelee repun valmistajan mukaan. Tunnistetietojen vertailuun ja merkin löytämiseen on hyvä pyytää apua kerhon kalustohenkilöiltä tai hyppymestareilta. 
\item  Tunniste- ja rekisteröintitiedot ovat luettavissa laitekortista. 
\item  Varjon omistajan vaihdoksesta tekee vanha omistaja merkinnän laitekorttiin.   
\item  Varjoihin ja reppu-valjasyhdistelmään tehdyt tarkastukset, huollot, korjaukset sekä lentokelpoisuusmääräysten edellyttämät työt näkyvät laitekortista. Seuraavan tarkastuksen ajankohta kannattaa huomioida, ja jos mahdollista, käytetty varjo tai reppu-valjasyhdistelmä kannattaa tarkastuttaa ennen ostamista. 
\item  Varavarjo ja reppu-valjasyhdistelmä tarkastetaan aina yhdessä. Varavarjon ja valjaat voi tarkastaa kalustomestari C:n pätevyyden omaava henkilö. Päävarjo tarkastetaan aina kiinnitettynä reppu-valjasyhdistelmään. Päävarjon voi tarkastaa kalustomestari B tai C. Kalustomestari B:n tai C:n on tarkastettava päävarjo myös uuteen reppu-valjasyhdistelmään ennen kuin päävarjolla voidaan hypätä. 
\item  Varavarjon pakannut henkilö merkitsee aina laitekorttiin varavarjoon tehtävät toimenpiteet kuten pakkaukset ja käytönjälkeiset tarkastukset. 
\item  Päävarjon pakkausmerkinnäksi riittää hyppypäiväkirjamerkintä.  
\item  Automaattilaukaisimen laitekortista selviää laitteen tunnistetietojen lisäksi huoltotilanne ja mahdollinen pariston vaihdon ajankohta. 
\item  Ostettaessa pelkästään vara- tai päävarjoa tai valjaita on huomioitava, että pää- ja varavarjo sisältävät connectorilenkit, kantopunokset, sliderin ja itse varjon. Reppu-valjasyhdistelmään kuuluvat apuvarjo yhdyspunoksineen, kantohihnat sekä päävarjon ja varavarjon sisäpussi. 
\end{itemize}

Käytettyä tai uutta varjokokonaisuutta ostettaessa on harkittava tarkkaan, mitkä ovat omat lajimieltymykset ja mikä on todellinen hankintatarve. On myös hyvä miettiä, tarvitaanko kalliimpi, uusi varjo, vai riittääkö alkuun edullisempi, käytetty varjo. 

\section{ Kuntotarkastus }
\label{kaluston-tarkastus-ja-huolto-kuntotarkastus}


Käytetty varjo tai varjokokonaisuus kannattaa kuntotarkastuttaa kalustohenkilöllä säännöllisesti ja aina ennen ostopäätöstä. Koehyppäämällä ja tarkistuttamalla voidaan todeta varjokankaan mahdollinen ilmanläpäisyn lisääntyminen ja punosten venymisestä johtuvat ominaisuuksien heikkenemiset. Kuntotarkastuksessa selviää kaluston mahdollinen korjaustarve, mikä auttaa omalta osaltaan hinnan määrityksessä. Samalla on myös hyvä määrittää kokonaisuuden sopivuus ostajan päälle, ettei tule hankituksi liian isoa tai liian pientä reppu-valjasyhdistelmää. 


Kuntotarkastusta tehtäessä etsitään vikoja ja käytöstä johtuvia kulumajälkiä valjaista, repusta, pää- ja varavarjosta. Rispaantuneet paineentasaukkojen reunat, ohjauspunosten alapäiden ja reisihihnojen kulumiset ovat näkyvimpiä ja yleisimpiä käytöstä johtuvia kulumisen kohteita. Varjokokonaisuuden kulumiseen vaikuttavia asioita ja seurattavia kohteita: 

\begin{itemize}
\item  Punokset kuluvat nopeasti kuluneiden sliderin renkaiden takia. Punossetti kestää yleensä 500–700 hyppyä. Punosten epätasainen kutistuminen aiheutuu sliderin tuottamasta kitkalämmöstä ja huonontaa varjon lento- ja avautumisominaisuuksia. 
\item  Avausjärjestelmään sekä kupuun tulee palamisreikiä huolimattoman pakkauksen vuoksi. Valmistajan ohjeiden mukainen pakkaus säästää kupua. 
\item  Ompeleet ja varjokangas kuluvat lian ja roskien vaikutuksesta. 
\item  Valjaiden kulumista ja likaantumista lisää ilman pakkausalustaa pakkaaminen. 
\item  Kuluvia osia, kuten ohjauspunosten alapäitä, paineentasausaukkojen reunoja, avausjärjestelmän osia, sliderin renkaita sekä mahdollisia tarroja on tarkkailtava ja niiden vaihdattaminen tai korjauttaminen kannattaa suorittaa ajoissa muiden osien säilymiseksi. 
\end{itemize}

Määräaikaistarkastusten yhteydessä suoritettavat korjaukset ja osien vaihdot voivat tuntua kalliilta, mutta ne takaavat varjokokonaisuuden toimivuuden ja hyppyturvallisuuden. Korjaukset ja vaihdot saa suorittaa vain kalustomestari B tai C pätevyytensä mukaisesti. 

\section{ Huolto }
\label{kaluston-tarkastus-ja-huolto-huolto}


Päivittäiseen kalustohuoltoon kuuluu 

\begin{itemize}
\item  kierteiden poisto ohjauspunoksista ja yhdyspunoksesta 
\item  päävarjon luupin kunnon ja pituuden tarkastus ja tarvittaessa vaihto uuteen 
\item  irtohiekan ja muun mahdollisen lian poistaminen päävarjosta ravistamalla takahelmasta 
\item  sisäpussin kuminauhojen vaihtaminen ehjiin ja valjaskuminauhojen kunnon tarkastaminen 
\item  repun ja valjaiden puhdistus irtoliasta tarvittaessa esimerkiksi juuriharjalla 
\item  varusteiden kuivaus tarvittaessa sekä oikea säilytys suojassa valolta, lialta ja kosteudelta. 
\end{itemize}

Varusteiden huollossa ja säilyttämisessä on huomioitava myös seuraavat seikat: 

\begin{itemize}
\item  Kuvun ja valjaiden pesua kannattaa välttää. Pakottavissa tilanteissa pesu on mahdollinen, mutta asiassa kannattaa kääntyä kalustohenkilön puoleen turvallisen pesutavan selvittämiseksi. 
\item  Ei säilytetä varjoa auton takakontissa ilman varjokassia. 
\item  Luupin on oltava ehjä ja oikean pituinen. 
\item  Kerhon säilytystilat on pidettävä kuivina, eikä niissä saa säilyttää mitään syövyttäviä tai paloarkoja aineita. 
\item  Varjoa säilytetään talven yli kotona kuivassa ja pimeässä paikassa. Kuiva varjo voi olla sisäpussissaan, mutta varavarjon jousiapuvarjo on hyvä vapauttaa, mikäli säilytys on pitempiaikaista.  
\end{itemize}

Kun kalusto on kunnossa, voidaan hypätä turvallisesti. Kalusto huolletaan itse päivittäin ja toimitetaan kalustohenkilölle heti, jos siinä epäillään vähäistäkin kulumista tai vikaa. Tarkastamattomalla ja huoltamattomalla kalustolla ei saa hypätä. Kalustosta kannattaa pitää huolta, sillä ajoissa tehdyt huollot säästävät isommilta vaurioilta ja kalliilta korjauksilta. 

\section{ Harjoitus }
\label{kaluston-tarkastus-ja-huolto-harjoitus}

\begin{enumerate}[label=\bfseries \arabic*)]
\item  Tutustutaan varjokirjoihin ja laitekortteihin. 
\item  Suoritetaan varjon ja reppu-valjasyhdistelmän kuntotarkastus ja päivittäinen huolto pakkauksen yhteydessä. 
\end{enumerate}
