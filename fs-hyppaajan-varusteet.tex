\section{ Haalarit }
\label{fs-hyppaajan-varusteet-haalarit}


FS-hyppäämisessä haalareilla voidaan vaikuttaa putoamisnopeuteen, liikkumiseen ilmassa ja otteiden ottamiseen. Putoamisnopeutta voi säädellä haalareiden materiaalilla ja tiukkuudella. Hitaasti putoavien hyppääjien (kevyt ja/tai pitkäraajainen) haalareiden pitäisi olla liukaspintaisia. Vastaavasti nopeasti putoavien hyppääjien haalareiden pitäisi olla karkeampaa materiaalia. 


Haalareiden grippien tulisi olla tukevat ja riittävän paksut pitävän otteen saamiseksi. Grippien paikkaa voidaan korostaa muusta haalarista erottuvalla värillä. Monipuolisempien otteiden saamiseksi jaloissa on hyvä olla gripit sekä jalan ulko- että sisäsivuilla. Bootillisilla haalareilla saadaan jalkoihin enemmän pinta-alaa ja siten tehokkuutta liikkeisiin sekä kantavuutta liukuun. 

\section{ Painot }
\label{fs-hyppaajan-varusteet-painot}


Haalareiden lisäksi putoamisvauhdin eroja voidaan tasata painoilla. Painoja voidaan sijoittaa valjaisiin, painoliiveihin tai -vyöhön. Jos painoja tarvitsee vain pari kiloa, ovat valjaat hyvä vaihtoehto painojen sijoittamiseen, mikäli valjaissa on taskut niitä varten. Painoliiveissä painot ovat yleensä liivien etupuolella. Suurilla painomäärillä liivit rasittavat niska- ja hartiaseutua. Painovyössä painot sijaitsevat vyötäröllä, jolloin paino sijoittuu paremmin keskivartalolle/lantiolle. Painovyötä voi käyttää haalareiden alla tai päällä. Painojen materiaalina lyijyhaulit ovat lyijylaattoja parempia, koska ne ovat mukavampia käyttää. 


Painojen tarve täytyy huomioida kuvunvalinnassa. Esimerkiksi 6 kg painoja 50 kg painavalla hyppääjällä lisää exit-painoa noin 10 \% kasvattaen huomattavasti siipikuormaa. 

\section{ Korkeusmittari }
\label{fs-hyppaajan-varusteet-korkeusmittari}


Rintahihnaan sijoitettu mittari mahdollistaa korkeuden tarkkailun toisen mittarista ilman lentoasennon muuttumista tai katsekontaktin menettämistä. Rintamittari näkyy hyvin myös liu'un aikana verrattuna käsimittariin. Tavallisen korkeusmittarin ohella turvallisuutta lisää äänikorkeusmittari. Paras vaihtoehto on äänikorkeusmittari, johon voi asettaa hälytysäänen useampiin korkeuksiin, ainakin purku- ja avauskorkeuteen. Äänikorkeusmittari on vain muistuttava apuväline. Korkeuden tarkkailua mittarista ei saa unohtaa! 

\section{ Kypärä }
\label{fs-hyppaajan-varusteet-kypara}


FS-hypyillä tulee toisinaan hyppääjien välille korkeuseroja ja käännökset voivat olla hyvinkin nopeita. Korkeuserot, hallitsemattomat ja rajut liikkeet altistavat pään iskuille. Tästä syystä FS-hypyillä onkin aina syytä käyttää kovaa kypärää -- mieluiten integraalikypärää, joka suojaa myös kasvoja. 

\section{ Rullalaudat }
\label{fs-hyppaajan-varusteet-rullalaudat}


Lautakuivien mielekkyyttä lisäävät hyvin rullaavat ja mukavat laudat. Rullalautoja rakennettaessa kannattaa valita renkaat huolella siten, että ne ovat riittävän isot ja kestävät. Pienillä renkailla varustetun laudan liikuttaminen on raskaampaa. Lautaa voi myös pehmustaa esim. retkipatjan tyyppisellä materiaalilla. 

\section{ Muut varusteet }
\label{fs-hyppaajan-varusteet-muut-varusteet}


Sormikkaita kannattaa käyttää jokaisella hypyllä. Sormikkaat, joissa on luistamaton ja pitävä kämmenosa, takaavat otteen pidon silloinkin kun otteisiin tulee vetoa. Kilpailuissa sormikkaiden värin tulisi erottua haalareiden ja grippien väreistä, jotta tuomarit näkevät otteet ja purut hyvin. 


Automaattilaukaisin on suositeltava hyppylajista riippumatta! 

