
Laskuvarjohyppääjällä ei saa olla sellaista synnynnäistä tai hankittua vikaa, vammaa tai toiminnanvajausta, joka rajoittaa pysyvästi toimintakykyä tai saattaa äkillisesti/yllättäen vaikuttaa siihen siinä määrin, että tehtävän turvallinen suoritus vaarantuisi. Hyppääjällä ei saa olla vaikeaa psyykkistä poikkeavuutta, päihderiippuvuutta, epilepsiaa tai muuta neurologista sairautta, tuoretta (ei parantuneeksi todettua) aivo- tai kallovammaa, sydänvikaa tai sydämen toiminnan vajavaisuutta, keuhko- tai hengityselinsairauksia jne. Tuki- ja liikuntaelinten poikkeavuudet, kuten nivelten vaikeat liikerajoitteet, olkapäiden sijoiltaanmenotaipumus tai tuore luunmurtuma estävät myös hyppäämisen. 


Lääkäri voi harkintansa mukaan antaa luvan harrastaa laskuvarjourheilua, jos tauti, sairaus tai vamma ei harrastusta häiritse tai se on tilapäistä. Raskauden, loukkaantumisten ja sairauksien jälkeen on käytettävä myös tervettä järkeä hyppytoiminnan jälleen aloittamisessa.  

\section{ Fyysiset seikat }
\label{fysiologia-fyysiset-seikat}


Korvat ja nenän sivuontelot ovat kaikessa ilmailussa eniten ongelmia aiheuttavia elimiä. Kaasujen laajenemisen aiheuttamat paine-erot sekä noustessa ylöspäin koneen mukana että vapaapudotuksen aikana aiheuttavat vaikeuksia, jos ei osata toimia oikein. Paine-erot voivat aiheuttaa kipua, korvien soimista tai lukkiutumista ja jopa tärykalvon rikkoontumista. Korvien (korvatorvien) paineen tasaus on syytä opetella jo hyppyuran alussa. Nieleskely ja haukottelu ovat yleisimmin käytettyjä menetelmiä. Jos ne eivät tuota tulosta, niin puhalletaan kevyesti nenän kautta ilmaa pitäen samalla nenästä kiinni. Flunssaisena ja allergisen nuhan aikana tulee hyppäämistä välttää. Tupakanpoltto yhdistettynä tulehdukseen tai allergiaan heikentää korvien ilmastointia ja luo lisäriskin sairauksille.  


Lentokorkeuden kasvaessa ja ilmanpaineen laskiessa elimistölle välttämättömän hapen osapaine hengitysilmassa pienenee. Seurauksena elimistön happikyllästeisyys alenee. Kun kudosten verenkierron mukana saama happimäärä ei enää riitä ylläpitämään solujen normaalia toimintaa, syntyy hapenpuute eli hypoksia. 


Syntyvä hapenpuute vaikuttaa näkökykyyn, hermostoon, henkiseen sekä kaikkeen motoriseen toimintaan. Hapenpuutetta hypyn aikana lisäävät fyysinen aktiivisuus ja kylmyys. Hypoksiassa näkökyky ja näöntarkkuus heikkenevät sekä näkökenttä pienenee ja tummenee. Lisäksi saattaa ilmetä huimausta, päänsärkyä, kevyen olon tunnetta, ahdistusta, väsymystä, uneliaisuutta, tuntohäiriöitä, puutumista ja lihasnykäyksiä. Objektiivisia oireita ovat persoonallisuuden muutokset, arvostelukyvyn heikentyminen ja mm päätöksenteon hidastuminen. Myös reaktioaika pitenee ja koordinaatiokyky heikkenee.  


Hapenpuutteen oireiden ilmenemisnopeuteen ja vaikeusasteeseen vaikuttavat 

\begin{enumerate}[label=\bfseries \arabic*)]
\item  Aika, jonka kuluessa happivajaus on syntynyt 
\item  Happivajauksen aste 
\item  Aika, jonka hapenpuute kestää  
\item  Henkilökohtaiset ominaisuudet, kuten ruumiinrakenne, mahdolliset sairaudet. 
\end{enumerate}

Hapenpuutteen vaarallisuus piilee siinä, että oireita on vähän eikä niitä aina havaita ajoissa. 


Hapen saantiin on aina suhtauduttava vakavasti, kun hyppykorkeus on yli kolme kilometriä. Lisähappea tarvitaan määräysten mukaan kuitenkin vasta yli neljän kilometrin hyppykorkeuksissa. Happivajetta voidaan ehkäistä istumalla rauhalliseti paikallaan koneessa uloshyppyyn asti sekä pukeutumalla lämpimästi. 


Hyperventilaatio on tiedostamatonta hengityksen kiihtymistä ilman hapenpuutetta. Se liittyy useimmiten ahdistus- ja pelkotiloihin. Oireina ovat ilman loppumisen tunne, pistely sormenpäissä ja huulissa, sormien puutuminen, ahdistus, puristava tunne rinnassa, rintakipu, tajunnan häiriöt, lihaskouristukset ja lopulta tajuttomuus. Hyperventilaation hoito on ennaltaehkäisevää eli tietoista hengityksen rauhoittamista. Hyperventiloiva ihminen tuodaan koneella alas. Häntä rauhoitellaan ja hän voi lyhytaikaisesti hengittää pussiin, kunnes oireilu katoaa.  


Laitesukelluksen jälkeen on noudatettava annettuja varoaikoja ennen hyppäämisen aloittamista. Verenluovutuksen jälkeen ei hyppäämistä suositella viikkoon. 

\section{ Lääkkeet, alkoholi ja tupakka }
\label{fysiologia-laakkeet-alkoholi-ja-tupakka}


Säännöllinen lääkkeiden käyttö, vaikka kyse ei olisikaan kolmiolääkkeistä, saattaa olla riski laskuvarjourheilussa. Lääkkeitä ottavan kannatta kysyä itseltään, ovatko tauti tai sen oireet itsessään este hyppäämiselle? Jotkut lääkkeet eivät sovi hyppääjille, vaikkei lääkepakkauksen kyljessä olisikaan varoituskolmiota. Lääkkeiden sopivuudesta on syytä kysyä hoitavalta lääkäriltä tai, jos tämä ei tunne ilmailun vaatimuksia, ilmailulääkäriltä. Lisenssihyppääjä vastaa itse lääkkeiden käytöstään yhdessä lääkärin kanssa. On muistettava, että lääkeiden mahdolliset vaikutukset toimintakykyyn saattavat johtaa onnettomuuteen ja sitä kautta vaikuttaa myös muiden hyppääjien turvallisuuteen. 


Unilääkkeet ja rauhoittavat lääkkeet turruttavat aisteja ja hidastavat reaktioita. Niiden vaikutus voi jatkua useita vuorokausia käytön jälkeen. Vahvat kipulääkkeet ovat myös pääosin kiellettyjä, sillä ne heikentävät suorituskykyä ja aiheuttavat väsymystä. Huumausaineet saattavat aiheuttaa pitkälläkin aikavälillä vaikutuksia, joita ei pystytä ennakoimaan. Puudutusaineet hammashoidossa vaativat vuorokauden hyppäämättömyyden. 


Pelon voittamiseksi ei lääkkeitä saa käyttää! 


Tupakoinnin haittavaikutukset saattavat olla hyvin merkittäviä. Pahin tupakoinnin lyhytvaikutteisista haitoista johtuu tupakan savun sisältämästä hiilimonoksidista (CO) eli häkäkaasusta. CO vähentää veren hapenkuljetuskykyä, jolloin hapen tarjonta kudostasolla laskee. Esimerkiksi kolmen tupakan poltto juuri ennen taivaalle lähtöä nostaa veren hiilimonoksidipitoisuutta siinä määrin, että ko. henkilöillä 1500 metrin korkeus vastaa tupakoimattoman henkilön noin 3750 metrin korkeutta.  


Hyppääjän suoristuskyvyn kannalta jo hyvin pienilläkin alkoholimäärillä (esimerkiksi yksi pullollinen keskiolutta) on todettu olevan selvästi haitallisia vaikutuksia. Alkoholin vaikutus on selvästi suurempi lievän hapenpuutteen vallitessa: esimerkiksi 3000 metrin korkeudessa nautittu annos vastaa 3–4 annosta merenpinnan tasolla. Siksi hyppytoiminnassa rajana on ehdoton 0 ‰. 


Antidopingtoimikunnan (ADT) dopingsäännöt koskevat myös laskuvarjohyppääjien virallista kilpailutoimintaa. On kuitenkin syytä muistaa, että vaikka lääke olisi sallittu urheilussa, niin se ei ole välttämättä sallittu ilmailussa. 

\section{ Olosuhteet ja voimat }
\label{fysiologia-olosuhteet-ja-voimat}


Valvominen vaatii veronsa jo normaaliolosuhteissa, saatikka hypätessä. Fyysinen suorituskyky laskee, vaikka hetkellisesti (hypyn ajan) hyppääjä tuntisikin itsensä vahvaksi. Tarkkaavaisuus, ajan taju, muisti, keskittymis\mbox{-,} oppimis- ja hahmotuskyky sekä havainnointi heikkenevät. Reaktioaika pitenee, virheet yleistyvät, mutta yksinkertaiset automaatiotasolle opitut liikesarjat kyetään useimmiten tekemään vielä väsyneenäkin. Riski loukkaantua ja tehdä virhearviointeja ja -suorituksia lisääntyy. Säännöllinen lepo ja uni takaavat osaltaan onnistuneen hyppypäivän. 


Ihmisen toimintakyky alenee nopeasti lämpötilan laskiessa. Käden lämpötilan (ei ilman lämpötilan) laskiessa alle +30 °C lihasten voima alkaa vähentyä ja tuntoherkkyys heiketä. Alle +7 °C kipu- ja lämpötuntemus katoaa ja käsien käyttö ei enää onnistu normaalisti. Sormikkaiden käyttöä suositellaan aina laskuvarjohypyillä. 


Nopeuden määrän ja suunnan muutos eli kiihtyvyys (hidastuvuus) vaikuttaa laskuvarjohyppääjään jokaisella hypyllä uloshypyssä ja avauksessa. Pitkäkestoisen hidastuvuuden (yli 2 sekuntia) aikana elimistö pyrkii sopeutumaan tilanteeseen, mutta jo poikkeuksellisen nopean varjon avautumisen aikana saattaa esiintyä lyhytkestoista näkökentän heikkenemistä. Vapaapudotuksessa negatiivinen kiihtyvyys lattakierteessä voi aiheuttaa tajunnan menetyksen.  Varavarjotoimenpiteiden tekeminen vajaatoimisella, suurella siipikuormalla olevalla varjolla on tehtävä ennen kuin vauhti ehtii kiihtyä liikaa.  Törmäystilanteessa hidastuvuus on lyhytkestoista, joten merkittävimpiä vaurioita ovat elimistön rakennevammat. G-voimien sietokykyyn vaikuttavat mm. hyppääjän ruumiinrakenne, yleiskunto, sairaudet, tottumattomuus, nestevajaus, ja vireystila. Urheileva, terve, normaalipainoinen ja -kuntoinen laskuvarjohyppääjä kestää normaalit G-voimat useita kertoja saman päivän aikana. 

