
A\mbox{-,} B\mbox{-,} C- tai D-lisenssihyppääjä on vastuussa myös käytännön hyppytoiminnan pyörittämisestä. Aina ei paikalla ole kouluttajia tai edes kokeneita hyppääjiä huolehtimassa rutiineista, joita tarvitaan normaalin hyppytoiminnan suorittamiseksi määräysten mukaan. Tässä luvussa annetaan ohjeita tätä toimintaa varten. Osaa niistä ei välttämättä tarvita kaikissa kerhoissa, mutta toisaalta kerhon toiminta saattaa vaatia erikoisuuksia, joita tässä ei ole mainittu. Tästä syystä on keskusteltava kouluttajien kanssa omassa kerhossa vallitsevista käytännöistä ja selvitettävä ne myös mentäessä vierailemaan toiseen kerhoon. 

\section{ Hyppylennon pokanvanhin }
\label{hyppytoiminnan-jarjestaminen-hyppylennon-pokanvanhin}


Jokaisella hyppylennolla tulee olla vähintään A-lisenssin omaava tehtäväänsä nimetty pokanvanhin, joka vastaa pokansa toiminnasta kokonaisuudessaan. Oppilaita pudottava kouluttaja toimii pokanvanhimpana. Mikäli kouluttaja poistuu koneesta ennen viimeistä linjaa, tulee pokaan olla nimetty pokanvanhin myös hyppylennon loppuajaksi.  


Pokanvanhimman paikka lentokoneessa on esimerkiksi ennalta kerhossa sovittu tai siitä on sovittava ennen hyppylennolle lähtöä siten, että lentäjä ja kaikki pokan hyppääjät ovat tietoisia asiasta.  


Pokanvanhin vastaa sujuvasta ja turvallisesta hyppytoiminnasta oman pokansa osalta, ja hänen tehtävänsä ovat vähintään 

\begin{itemize}
\item  hyppyjärjestyksen määrittäminen 
\item  ohjeiden antaminen lentäjälle ennen hyppylentoa ja lennon aikana 
\item  vastaaminen pokasta koneen kuormauksen ja lennon aikana 
\item  poikkeus- tai vaaratilanteessa yhteydenpito lentäjän kanssa ja ohjeiden anto hyppääjille. 
\end{itemize}
\section{ Hyppytoiminnan aloitus }
\label{hyppytoiminnan-jarjestaminen-hyppytoiminnan-aloitus}


Lennonjohdolle/aluelennonjohdolle tehdään ilmoitus/varaus hyppytoiminnasta. Ilmoitus tehdään tietyn kaavan mukaan ja tarvittaessa siihen löytyy kerhoista ohjeet. Joissakin kerhoissa tämän hoitavat automaattisesti lentäjät. Jos mahdollista, hyppytoiminnan aloittamisen suunnittelussa kannattaa käyttää apuna METAR\mbox{-,} TAF- ja GAFOR-sanomia, joita verrataan sääminimeihin. 


Hyppykoneen lentäjällä täytyy olla vähintään 100 tunnin lentokokemus, joista vähintään 75 tuntia kyseisellä ilma-alustyypillä (esimerkiksi lentokone tai helikopteri). Lisäksi hänen on oltava yhteisöön kuuluva ja etukäteen perehtynyt kyseisen ilma-aluksen ominaisuuksiin laskuvarjohyppytoiminnan kannalta. Kerhon vakituiset lentäjät ovat luonnollisesti em. ehdot täyttäviä kerhon hyväksymiä lentäjiä. Jos ollaan lähdössä ennestään tuntemattoman lentäjän kyytiin, on asiat syytä tarkastaa oman turvallisuuden vuoksi. 


Lentokoneen tarkastuksen koneen käsikirjan ohjeiden mukaisesti tekee lentäjä, joka koneen päällikkönä on vastuussa koneesta ja matkustajien turvallisuudesta. 


Toimittaessa muulla kuin kerhon koneella on syytä huomioida seuraavaa: ilma-aluksen lentokäsikirjassa tai sen liitteessä on oltava  

\begin{itemize}
\item  hyväksyntä siitä, että koneella voidaan lentää ilman ovea tai ovi voidaan avata lennon aikana 
\item  laskuvarjohyppytoimintaan tarvittavat ohjeet 
\end{itemize}

Lisäksi koneessa on oltava puukko tai vastaava teräase lentäjän ja hyppääjien saatavilla. Jos koneessa on yli 10 hyppääjää, kaikkien on käytettävä istuinvöitä. Yhdessä lentäjän kanssa on myös syytä tarkastaa, ettei poka ylitä koneen painorajoja.  


Maalialueen pitää olla hyppytoimintaan soveltuvassa kunnossa. Lisäksi tuulen suuntaa ja voimakkuutta osoittavan välineistön toiminta ja kunto tarkastetaan. Samalla kannattaa huomioida tuulen suunta ja voimakkuus. 


Jos toiminta vaatii maahenkilön paikallaolon, pitää hänen pätevyytensä ja tehtävänsä selvittää ja selventää. Lisäksi on syytä tarkistaa maahenkilön tarvitsemat varusteet (ensiaputarvikkeet, pelastusvälineet, vene, radiot, hakuauto). Jos osa tarvittavista varusteista sijaitsee maastossa, on myös ne käytävä tarkastamassa. 


Vallitseva sää ja olosuhteet on selvitettävä. On tehtävä päätös siitä, soveltuuko keli aiottuun laskuvarjohyppytoimintaan. Sää\mbox{-,} tuuli- yms. tiedot merkitään niille kerholla varattuihin paikkoihin. 


Lisäksi on huolehdittava, että kaikilla hyppytoimintaan osallistuvilla on asianmukaiset kelpoisuudet ja paperit kunnossa. Tämä koskee erityisesti kerholla hyppääviä vierailijoita. 


On muistettava myös määrittää uloshyppypaikka ja antaa ohjeet hyppääjille ja lentäjälle (jolle myös mielellään kirjallisena). 

\section{ Toiminnan aikana }
\label{hyppytoiminnan-jarjestaminen-toiminnan-aikana}


Kentällä ollaan harvoin yksin, joten yhteistyö muiden kentällä toimivien ilmailijoiden (esimerkiksi lennokkiharrastajat ja purjelentäjät) kanssa on tärkeää. 


Maastokartalle voidaan merkitä esimerkiksi uloshyppypaikka, varattu alue ja laskeutumiskuvio sekä -suunta. 


Koneessa ja maassa on syytä olla pelastuskartta, johon on ruudutettu hyppytoimintaan käytettävä kenttä ja sen ympäristö. Tilanteessa, jossa hyppääjä ei pääse kenttäalueelle, voidaan koneesta tarkasti sanoa, mihin ruutuun hyppääjä on laskeutunut, jolloin haku- ja pelastustoimet helpottuvat. 


Hyppypäivän aikana voi uloshyppypaikka tai suurin sallittu hyppykorkeus muuttua, jolloin on huolehdittava niiden perusteella tehtävistä merkinnöistä ja ilmoituksista. 


Koneessa on huolehdittava asianmukaisesta hyppyluvasta tai -ilmoituksesta. On myös otettava huomioon muut lentokentän ilmatilan käyttäjät sekä mahdollisesti kentällä liikkuvat ulkopuoliset. 


Säätilan kehittymistä on seurattava koko ajan, ja toiminta on tarvittaessa keskeytettävä tai sitä on rajoitettava. 


On muistettava täyttää myös pokalista, varmistaa hypyn jälkeen, että kaikki pokalla olleet pääsivät laskeutumisalueelle tai käynnistää tarvittavat pelastustoimenpiteet, tehdä lennonjohtoon ilmoitus esimerkiksi varavarjon käytöstä tai muusta vastaavasta sovitusta asiasta sekä huolehtia, että lentäjä ehtii pitämään asianmukaiset tauot. 

\section{ Toiminnan lopetus }
\label{hyppytoiminnan-jarjestaminen-toiminnan-lopetus}


Hyppytoiminnan loputtua on tehtävä asianmukaiset ilmoitukset, jotta lennonjohto ja/tai muut kentän käyttäjät saavat siitä tiedon. Lentokone on pysäköitävä ja suojattava. Kaikki käytössä ollut kalusto siirretään takaisin säilytyspaikoilleen. 

\section{ Harjoitus }
\label{hyppytoiminnan-jarjestaminen-harjoitus}

\begin{enumerate}[label=\bfseries \arabic*)]
\item  Kirjoitetaan muistiin lennonjohdon ja muiden tarvittavien yhteyksien numerot. 
\item  Harjoitellaan hyppyalueen varausta normaaliin toimintaan liittyen (jos hyppääjien tehtävä). 
\item  Suoritetaan vaadittavat toimenpiteet hyppytoiminnan aloittamiseksi.  
\item  Suoritetaan vaadittavat toimenpiteet hyppytoiminnan päättyessä. 
\item  Tutustutaan kerhon omiin ohjeisiin (esimerkiksi Ohje hyppytoiminnasta ja Pokanvanhimman ohje). 
\item  Kirjataan omalla kerholla käytössä olevat tärkeimmät erityistoimenpiteet. 
\end{enumerate}
