\section{ Laskuvarjohyppäämiseen liittyvät riskitekijät }
\label{riskitekijat-toiminta-onnettomuustilanteessa-ja-ensiapu-laskuvarjohyppaamiseen-liittyvat-riskitekijat}


Vakavat laskuvarjohyppyonnettomuudet tutkii ensisijaisesti poliisi ja erityistapauksissa Onnettomuustutkintakeskus. Tutkimuksen tarkoituksena on selvittää kaikki onnettomuuteen vaikuttaneet seikat, ja sitä käytetään hyväksi uusien onnettomuuksien ehkäisemisessä. Onnettomuustutkintakeskus julkaisee omista tutkinnoistaan tutkintaselostuksen, jossa on kaikille hyppääjille hyödyllistä luettavaa. Myös poliisitutkimuksien selostukset pyritään mahdollisuuksien mukaan julkaisemaan. Lisäksi Laskuvarjotoimikunnan Koulutus- ja turvallisuuskomitea julkaisee Turvallisuustiedotteita, joissa käsitellään turvallisuuteen liittyviä asioita. 


Hyvin yleistä onnettomuuksissa on, että tapahtumalle ei ole olemassa yhtä syytä, vaan useita osasyitä, jotka yhdessä ovat aiheuttaneet onnettomuuden. Riskitekijöitä ja onnettomuuteen johtavia tekijöitä laskuvarjourheilussa ovat esimerkiksi 

\begin{itemize}
\item  väsymys, krapula, huumaavat aineet, lääkkeet ja niiden vaikutuksen alaisena hyppääminen 
\item  ongelmat ja ristiriidat töissä, kotona tai kerholla 
\item  huolimaton päävarjon pakkaus 
\item  väärä pukeutuminen säähän ja olosuhteisiin nähden 
\item  lainavarusteilla hyppääminen ja usein vaihtuvat varusteet 
\item  muuttuneet toimintamallit varavarjotoimenpiteissä, ohjaamisessa jne. 
\item  liian vaativa hyppy kokemukseen nähden ja epäsäännöllinen hyppytausta 
\item  liika yrittäminen kokemuksen ollessa vähäinen 
\item  välinpitämättömyys muista hyppääjistä ja näyttämisen tarve 
\item  väärä uloshyppypaikka. 
\item  matalat purut ja aukaisut sekä lyhyt liuku tai puutteellinen liukutekniikka 
\item  puutteellinen ilmatilan tarkkailu varjon varassa ja laskeuduttaessa. 
\item  rajut ohjausliikkeet ja käännökset matalalla sekä liian vaativa varjo taitoihin nähden. 
\end{itemize}

Riskitekijöitä minimoitaessa on huomioitava ainakin seuraavat seikat: 

\begin{itemize}
\item  Tunnetaan omat tiedot, taidot ja vireystila eikä yliarvioida niitä. 
\item  Harjoitellaan opittuja toimintamalleja useasti hyppykauden aikana. 
\item  Hypätään säännöllisesti ja osallistutaan koulutustilaisuuksiin. 
\item  Noudatetaan määräyksiä ja sääntöjä sekä tiedetään kerho- ja kenttäkohtaiset ohjeet. 
\item  Tunnetaan kentän nostavat, laskevat ja pyörteet sekä ohjataan niistä pois täydellä liidolla (ZP-kangas). 
\item  Tutustutaan varjoon sekä varusteisiin eri tilanteissa ja keleissä sekä vältetään lainavarusteita. 
\item  Käytetään omaan taito- ja kokemustasoon soveltuvaa varjokalustoa sekä muita varusteita (esimerkiksi kamerakypärä). 
\item  Käytetään kaikilla hypyillä reppu-valjasyhdistelmää, jossa on varavarjon automaattilaukaisin. 
\item  Seurataan sään kehittymistä koko hyppypäivän ajan eikä hypätä, jos ollaan epävarmoja. 
\item  Ei olla itsekkäitä, vaan huomioidaan muut hyppääjät. 
\item  Varataan aina riittävästi korkeutta hyppysuoritusta varten. 
\item  Ei epäröidä käyttää varavarjoa epäselvässä tilanteessa. 
\item  Pukeudutaan sään mukaisesti huomioiden myös hyppykorkeus sekä käytetään aina sormikkaita. 
\item  Ei pakata märkää tai roskaista varjoa. 
\item  Ei hypätä pilveen. 
\item  Muistetaan, että onnettomuus ei ole yksityisasia! 
\end{itemize}

Hyppyturvallisuus rakentuu toimintamalleille ja -menetelmille. Toimintamalli on esimerkiksi opittu varavarjon käyttötapa. Toimintamallin tarkoituksena on, että voidaan harjoitella toimintoja niin hyvin ja aina samalla tavalla, että kriisitilanteessa toiminta olisi automaattista. Usein on voitu onnettomuuksien yhteydessä todeta, että hyppääjä ei ollut harjoitellut toimintamallia riittävästi tai hänellä oli ollut pitkä tauko hyppäämisessä, jolloin toimintamalli ei ollut riittävässä terässä. On myös muistettava, että erityisen riskialttiita ollaan, kun siirrytään käyttämään uutta toimintamallia, kuten uusia varusteita. Useissa onnettomuuksissa on ainakin yhtenä syynä ollut tässä oppaassa opetettavien perusturvallisuusasioiden laiminlyönti. 

\section{ Toiminta onnettomuustilanteissa }
\label{riskitekijat-toiminta-onnettomuustilanteessa-ja-ensiapu-toiminta-onnettomuustilanteissa}


Onnettomuuden tapahduttua keskeytetään hyppytoiminta välittömästi ja toimitaan SIL ry:n ohjeen Toiminta onnettomuustilanteessa mukaan. Kuolemantapauksissa ja vakavissa loukkaantumisissa onnettomuutta tutkimaan voidaan asettaa tutkintalautakunta, joka saavuttuaan paikalle ottaa johdon. Jos tutkintalautakuntaa ei aseteta, hoitaa poliisi tutkinnan. Lievissä loukkaantumisissa, varavarjotilanteissa ja hyppääjän ajautuessa ulos kenttäalueelta, toiminnan aloittaa maahenkilö ja toimintaa johtaa kokenein paikalla oleva hyppääjä. 


Onnettomuustilanteiden toimintamalli on vakio, mutta lievissä tapauksissa pelastustyöt vaikuttavat vain hetkellisesti muuhun hyppytoimintaan. Kerhon oma yksilöity toimintaohje eri tilanteita varten on säilytettävä puhelimen läheisyydessä sekä ensiapuvälineiden mukana. Siinä on huomioitava kentän erikoisolosuhteet (pelastusvene, -kartta ja -reitit), varalaskupaikat ja yhteistoiminta muiden alueella toimivien yhdistysten kanssa. Kerhon turvallisuuspäällikkö vastaa siitä, että hyppytoimintaan ja hyppyturvallisuuteen liittyvät määräykset, ohjeet ja tiedotteet ovat kaikkien hyppääjien saatavilla. 


Toiminta onnettomuuspaikalla (katso lisätietoja ohjeesta Toiminta onnettomuustilanteessa) on seuraava: 

\begin{itemize}
\item  Hyppytoiminta keskeytetään 
\item  Ilmoitukset: 
	\begin{itemize}
	\item  Hätäkeskus 112 
	\item  Suomen Ilmailuliitto 
	\item  Lähin lennonjohto (paikallinen / alue) 
	\item  Omaiset (kuolintapauksessa \textbf{AINA} viranomaisen kautta) 
		\begin{itemize}
		\item  Mihin sairaalaan loukkaantunut on viety 
		\end{itemize}
	\item  Kerholaisten (ennen kaikkea paikalla olleiden) keskinäinen kriisihuolenpito on tärkeää 
	\item  Yhdistyksen puheenjohtaja, koulutuspäällikkö ja yleisestä viestinnästä vastaava henkilö 
	\item  Silminnäkijöiden nimet, osoitteet ja puhelinnumerot 
	\item  Paikalla olleiden kerholaisten briefaus. Ei puhuta toisten kanssa tapahtuneesta, jotta todistajainlausunnot eivät yhdenmukaistuisi. \textbf{Tärkeintä, että medialle ei puhu kuin kriisitiedotuksesta vastaava henkilö!!} 
	\item  Ilmoittaudutaan kotiin, jos joku kuollut (sana leviää nopeasti) 
	\item  Kerhon puhelimet (numerot julkisesti saatavilla) otetaan haltuun. Vain se vastaa, joka tietää, mitä saa kertoa 
	\item  Otetaan talteen uhrin omaisuus 
	\end{itemize}
\end{itemize}
\subsection{Onnettomuuspaikalla}
\label{riskitekijat-toiminta-onnettomuustilanteessa-ja-ensiapu-onnettomuuspaikalla}

\begin{itemize}
\item  Estetään lisäonnettomuuksien synty (tie tai lähiympäristö) 
\item  Annetaan ensiapu 
\item  Eristetään alue ja estetään turhien jälkien syntyminen 
\item  Loukkaantuneen varusteet pyritään pitämään mahdollisimman alkuperäisessä tilassa, jos mahdollista 
\item  Kuolemaan johtaneissa onnettomuuksissa uhria ja varusteita ei saa siirtää. Uhri peitetään tarvittaessa 
\item  Otetaan onnettomuuspaikasta maastoon sidottuja valokuvia ja/tai videokuvaa 
\item  Järjestetään pelastushenkilökunnan sekä viranomaisten opastus onnettomuuspaikalle 
\end{itemize}
\subsection{Tiedottaminen}
\label{riskitekijat-toiminta-onnettomuustilanteessa-ja-ensiapu-tiedottaminen}

\begin{itemize}
\item  Tiedottamisen on oltava niukkaa. Ei soiteta itse minnekään. Jos tiedotusvälineistä soitetaan, kerrotaan vain, että a) onnettomuus on tapahtunut ja b) tiedottamisesta vastaa nimetty henkilö 
\item  Onnettomuuden uhrin nimeä ei saa julkaista, ennen kuin uhrin omaisille on mennyt tieto tapahtuneesta ja he ovat antaneet siihen luvan. Kerholaiset eivät saa kertoa nimeä ulkopuolisille 
\item  Onnettomuuteen johtaneista syistä ei saa esittää minkäänlaisia arvioita 
\item  Onnettomuudesta ei saa keskustella missään julkisessa paikassa, ei ravintoloisissa, ei busseissa, ei saunassa eikä missään, missä on ulkopuolisia kuulijoita 
\item  Täytetään ja lähetetään vaaratilanneraportti (tätä voi täydentää myöhemmin) 
\end{itemize}
\section{ Ensiapu }
\label{riskitekijat-toiminta-onnettomuustilanteessa-ja-ensiapu-ensiapu}


Hätäensiavulla tarkoitetaan sitä välitöntä ensiapua, jolla pelastetaan potilaan henki. Hätäensiavun tarkoituksena on palauttaa ja ylläpitää elintärkeät toiminnot siihen asti, kunnes potilas saadaan hoitoon. Hätäensiapu aloitetaan onnettomuuspaikalla, ensimmäisenä paikalla olevan ensiaputaitoisen toimesta. Älä siirrä loukkaantunutta, ellei se ole välttämätöntä. Nykypäivän ensiavun periaatteena on se, että ensiapu tuodaan loukkaantuneen luo eikä päinvastoin. 

\subsection{ Hätäensiapu (elvytys) }
\label{riskitekijat-toiminta-onnettomuustilanteessa-ja-ensiapu-hataensiapu-elvytys}

\begin{enumerate}[label=\bfseries \arabic*)]
\item  Kartoita nopeasti yleistilanne. Ota selvää saatko elottomalta näyttävän hereille puhuttelemalla ja tarvittaessa varovaisesti ravistelemalla. Jos potilas ei herää… 
\item  …kutsu apua ja tee hätäilmoitus numeroon 112. Kerro lyhyesti kuka olet, mistä soitat ja mitä on tapahtunut ja anna tarkka osoite ja/tai ajo-ohjeet. Kuuntele ja vastaa sinulle esitettyihin kysymyksiin, ja toimi saamiesi ohjeiden mukaan. Varaudu kertomaan ainakin potilaan nimi, potilaan tila (tajunnan taso, hengitys, verenkierto, näkyvät vammat). Katkaise puhelu vasta luvan saatuasi. 
\item  Tarkista hengittääkö potilas. Avaa tarvittaessa kypärän hihna ym. hengitysteitä kiristävät asiat. Mikäli kaularankavamma on mahdollinen, kuten lähes aina vakavassa laskuvarjohyppytapaturmassa, pään taivutus tehdään suurta varovaisuutta noudattaen. Ojenna autettavan pää leuan kärjestä nostamalla ja toisella kädellä otsasta painamalla. Samalla katso, kuuntele ja tunnustele hengitystä. Liikkuuko rintakehä? Kuuluuko hengityksen ääni? Tuntuuko ilman virtaus poskellasi? Arvio onko hengitys normaalia, epänormaalia tai hengitys puuttuu. Käytä enintään 10 sekuntia hengityksen arviointiin. Mikäli epäröit, toimi kuin hengitys ei olisi normaalia. Jos potilas on reagoimaton, mutta hengittää normaalisti, hänet tulee kääntää kylkiasentoon hengityksen turvaamiseksi. Tue kaularankaa ja seuraa potilasta jatkuvasti ja varmista että hengitys jatkuu. 
\item  Potilaalle tulee aloittaa painelu-puhalluselvytys, jos hän ei herää eikä hengitä normaalisti. Aseta kämmenesi tyviosa keskelle autettavan rintalastaa ja toinen kätesi rintalastalla olevan käden päälle. Sormet ovat limittäin.  
\item  Paina 30 kertaa suorin käsivarsin kohtisuoraan alaspäin siten, että rintalasta painuu 5 - 6 cm. Anna rintakehän palautua paineluiden välissä. Keskimääräinen painelutiheys on 100 kertaa minuutissa, eikä ylitä 120 kertaa minuutissa. Laske painelut ääneen. 
\item  Puhalla 2 kertaa. Avaa hengitystie nostamalle leukaa, tarkista että potilaan suu on tyhjä. Aseta suusi tiiviisti autettavan suun päälle ja sulje sormillasi hänen sieraimensa. Puhalla rauhallisesti ilmaa autettavan keuhkoihin. Puhalluksen aikana katso, että autettavan rintakehä nousee (liikkuu). Toista puhallus. Kahden puhalluksen kesto on 5 sekuntia 
\item  Jatka elvytystä tauotta rytmillä 30:2, kunnes autettava herää: liikkuu, avaa silmänsä ja hengittää normaalisti tai ammattihenkilöt antavat luvan lopettaa tai voimasi loppuvat. 
\end{enumerate}
\subsection{ Suuret verenvuodot }
\label{riskitekijat-toiminta-onnettomuustilanteessa-ja-ensiapu-suuret-verenvuodot}


Näin tyrehdytät verenvuodon: 

\begin{enumerate}[label=\bfseries \arabic*)]
\item  Nosta vuotava raaja ylös ja tyrehdytä verenvuoto painamalla sormin tai kämmenellä suoraan vuotokohtaan. 
\item  Aseta runsaasti vuotava potilas välittömästi pitkälleen. 
\item  Kun sidetarvikkeita on käytettävissä, sido vuotokohtaan paineside. 
\item  Tue vuotava raaja kohoasentoon. 
\end{enumerate}

Jos verenvuoto tyrehdyttämistoimenpiteistä huolimatta jatkuu, paina raajan tyvestä suuria suonia voimakkaasti kämmenellä valtimoveren virtauksen estämiseksi. Jos vuoto ei vieläkään asetu, aseta vuotokohdan yläpuolelle kiristysside. (hätätilanteessa voit käyttää apuna esimerkiksi laskuvarjon punoksia). Runsas verenvuoto voi johtaa verenkierron vakavaan häiriötilaan eli sokkiin. Hälytä tarvittaessa apua. 

\subsection{ Sokki }
\label{riskitekijat-toiminta-onnettomuustilanteessa-ja-ensiapu-sokki}


Muistetaan aina sokin vaara ja annetaan sokin oireenmukainen ensiapu: 

\begin{enumerate}[label=\bfseries \arabic*)]
\item  Asetetaan potilas pitkälleen. 
\item  Tarkkaillaan ja varmistetaan hengitys. 
\item  Kohotetaan alaraajat tarvittaessa ylös. 
\item  Tyrehdytetään verenvuoto, minimoidaan kipu ja tuetaan murtumat. 
\item  Suojataan potilas kylmältä ja rauhoitellaan häntä. 
\item  Tarkkaillaan tajunnan tasoa ja sen muutoksia. 
\end{enumerate}
\subsection{ Muu ensiapu }
\label{riskitekijat-toiminta-onnettomuustilanteessa-ja-ensiapu-muu-ensiapu}


(Nivelnyrjähdys / sijoiltaan meno / revähdys / venähdys / murtuma) 

\begin{enumerate}[label=\bfseries \arabic*)]
\item  Annetaan paikallisesti kylmähoitoa kylmäpusseilla, -suihkeilla tms. (kylmä supistaa etenkin pieniä verisuonia ja vähentää sisäistä verenvuotoa). 
\item  Kohotetaan loukkaantunut raaja turvotuksen ja sisäisen verenvuodon vähentämiseksi. 
\item  Tarvittaessa sidotaan nyrjähtäneen nivelen ympärille joustoside. 
\item  Vältetään loukkaantuneen raajan liikuttamista (raajan liikuttelu lisää kivun lisäksi verenkiertoa ja pahentaa sisäistä verenvuotoa ja turvotusta). Mahdollisuuksien mukaan tuetaan murtuma pehmustetulla tukevalla lastalla liikkumattomaksi. Nivelen sijoiltaan menossa niveltä ei saa yrittää vetää. 
\item  Tarvittaessa toimitetaan loukkaantunut pikaisesti terveyskeskukseen tai sairaalaan kuvantamistutkimuksia ja hoitoa varten. Kuljetus määräytyy potilaan vammojen mukaan. (Myös vakuutusoikeudellisten seikkojen takia kannattaa käydä lääkärissä mahdollisimman pian tapaturman sattumisen jälkeen.) 
\end{enumerate}
\section{ Vaaratilanneilmoitus }
\label{riskitekijat-toiminta-onnettomuustilanteessa-ja-ensiapu-vaaratilanneilmoitus}


Jos laskuvarjohypyn yhteydessä sattuu sellainen vaaratilanne, jossa hyppääjän tai ilmaliikenteen turvallisuus on ollut uhattuna, on tapahtuneesta viipymättä tehtävä kirjallinen selonteko SIL:lle. Tapauksia, joista ilmoitus on aina tehtävä, ovat mm. seuraavat: 

\begin{itemize}
\item  kaikki hypyt, joissa on käytetty varavarjoa, 
\item  kaikki tapaukset, joissa on sattunut lääkäri- tai sairaalahoitoon johtaneita loukkaantumisia, 
\item  kaikki tapaukset, joissa varjon toiminta on ollut epänormaali (vajaatoiminta), 
\item  tapaukset, jotka muuten ovat turvallisuutta vaarantavia tai joista olisi voinut kehittyä vaaratilanne. 
\end{itemize}

SIL:n laskuvarjohyppääjiä koskeva \textit{Toiminnallinen ohje} antaa lisätietoja vaaratilanneilmoituksen tekemisestä.  

\section{ Harjoitus }
\label{riskitekijat-toiminta-onnettomuustilanteessa-ja-ensiapu-harjoitus}

\begin{enumerate}[label=\bfseries \arabic*)]
\item  Tutustutaan Vaaratilanneilmoitukseen ja sen täyttöön. Lomake löytyy mm. Koulutus- ja turvallisuuskomitean nettisivuilta. \url{http://www.ilmailuliitto.fi/fi/suomen_ilmailuliitto/materiaalipankki/vaaratilanneilmoitus_hyppytoiminnassa} 
\item  Selvitetään, missä kerholla säilytetään turvallisuuspäällikön kokoamaa hyppyturvallisuuteen liittyvää materiaalia. 
\item  Tutustutaan kerholla olevaan SIL ry:n ohjeeseen Toiminta onnettomuustilanteessa. 
\item  Tutustutaan kerhon omiin ohjeisiin toiminnasta vaara- ja onnettomuustilanteessa.  
\item  Tutustutaan julkaistuihin tutkintaselostuksiin laskuvarjohyppyonnettomuuksista. 
\item  Tutustutaan kerholta löytyviin ensiapuohjeisiin ja -välineisiin, pelastusvälineistöön sekä niiden käyttöön. 
\end{enumerate}
