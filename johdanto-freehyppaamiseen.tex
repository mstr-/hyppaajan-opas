\section{ Freefly-hyppääminen }
\label{johdanto-freehyppaamiseen-freefly-hyppaaminen}


Lajina freefly, vapaasti käännettynä vapaa lentäminen, antaa mahdollisuuden luoda oman tyylinsä lentää ja liikkua taivaalla. Ei siis ole olemassa oikeaa tai väärää tyyliä lentää freetä. 


Freeflyn määritelmä on hyvin laaja.  Alkujaan sillä tarkoitettiin erilaisia tyylejä lentää niin, että selkäranka on suorassa ilmavirran suuntaisesti. Nykyään käsite sisältää käytännössä kaiken mahdollisen:  

\begin{itemize}
\item  \textit{Headdown}, pää alaspäin lentämisen, 
\item  \textit{Headup}, pää ylöspäin lentämisen, 
\item  \textit{Angle}, pää alaspäin tai ylöspäin kulmassa ilmavirtaan nähden lentämisen. 
\end{itemize}

Perusajatuksena on pystyä hallitsemaan oma vartalo ilmavirrassa erilaisissa lentoasennoissa. 


Joitakin asioita on syytä tietää, ennen kuin aloittaa harjoittelun. Tämä opas antaa perusteet freefly -hyppäämisen turvallisuudesta, lentoasennoista sekä perusliikkeistä vapaassa. Oppaan avulla voit aloittaa itsenäisen harjoittelun helposti esimerkiksi sittiksestä, istualtaan lentämisestä. Neuvoja hyppäämiseen kannattaa kysellä myös kerhosi freefly -kouluttajalta tai kokeneelta freefly -hyppääjältä. He neuvovat mielellään ja osaavat ohjata oikeaan suuntaan. Lukemalla tämän oppaan et muutu päivässä mestariksi, mutta sinulla on varmasti paremmat valmiudet tulla sellaiseksi! 

