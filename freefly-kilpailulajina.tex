
Termi freefly alkoi yleistyä vasta 1990-luvun alkupuolella, kun freestylen MM-kilpailuissa 1993 Olav Zipser ja Mike Vail voittivat kultaa joukkueella Freefly Clowns. Tämän jälkeen termi on yhdistetty nykyisen freeflyn lentämistyyliin. Pian Freefly Clownsien voiton jälkeen freeflyssa alettiin kilpailla omissa kilpailuissaan kolmihenkisin joukkuein. 


Freeflyissa kilpaillaan kahdessa eri lajissa. Freefly artistilajina on niistä vanhempi. Vertical formation skydiving (VFS) on lajina uudempi, ja kuuluu itse asiassa lajina kuviohyppäämiseen. Tässä luvussa esitellään lyhyesti nämä molemmat lajit. 

\section{ Artistic freefly }
\label{freefly-kilpailulajina-artistic-freefly}


Freefly-joukkueessa on kolme jäsentä, joista kaksi on artisteja ja kolmas toimii kuvaajana. Virallisissa freefly-kilpailuissa hypätään 7 kierrosta, joista 2 on pakollisia ja 5 vapaita. Vapaaohjelmassa joukkue suorittaa erilaisista lentoasennoista koostuvan itse keksimänsä ohjelmakokonaisuuden, jonka tuomarit arvioivat asteikolla 0–10. Arvioinnissa tarkastellaan teknistä vaikeutta, liikkeiden taiteellisuutta, kameratyöskentelyä, kokonaiskuvaa, monipuolisuutta ja joukkueen yhteistyötä. Vapaaohjelman suunnittelussa joukkueen kannattaa kiinnittää huomiota muun muassa siihen, että hypyllä on selkeä alku ja loppu, suoritusaika on käytetty hyödyllisesti ja liikkeet ovat monipuolisia sekä luovasti käytettyjä. Hypyn tulee olla esteettistä ja nautinnollista katsottavaa. Pelkällä hyvällä artistisuorituksella ei kuitenkaan vielä saada pisteitä ellei kuvaus ole onnistunut. Hyvä kuvaaja pitää artistit hypyn alusta loppuun keskellä kuvaa, hyödyntää monipuolisia kuvakulmia sekä maiseman, pilvien ja auringon sijoittumista kuvaan. Hyppykorkeutena on 13000 jalkaa eli noin 4000 metriä. Suoritusaika on 45 sekuntia. Pakolliset kierrokset hypätään samasta korkeudesta. Pakollisilla kierroksilla hypätään kummallakin neljän eri pakollisen liikkeen yhdistelmä. Liikkeet ovat joukkueiden tiedossa etukäteen. Liikkeet tai liikesarjat pyritään suorittamaan mahdollisimman puhtaasti ja siirtymät liikesarjojen välillä arvioidaan myös, tarkoituksena että hypystä muodostuisi kokonaisuus, johon pakolliset liikkeet sijoittuvat. 

\section{ Vertical formation skydiving }
\label{freefly-kilpailulajina-vertical-formation-skydiving}


VFS-joukkueessa on viisi jäsentä, joista yksi toimii kuvaajana. Tarkoituksena on hypätä ennalta määrättyjä neljän hyppääjän muodostelmia, jotka rakentuvat headdown ja headup kombinaatioista (eli sittis- ja hetukka-asentojen oteyhdistelmistä). Kilpailuissa hypätään kahdeksan kierrosta, joihin arvotaan 3–5 muodostelmaa ennalta määrätyistä kuvioista. Näitä muodostelmia pyritään toistamaan mahdollisimman monta kertaa oikeassa järjestyksessä. Hyppykorkeus on 13000 jalkaa eli vajaa 4000 metriä. Suoritusaika on 35 sekuntia. Tuomarit arvostelevat hypyt videonauhalta näkemiensä pisteiden (otteiden) perusteella ja voittajajoukkue on se, joka saa suurimman yhteispistemäärän kahdeksalta kierrokselta. 

