
Kun valitset itsellesi laskuvarjoa, vertaa mielikuvaasi sinulle jo tutuista malleista ja kokoluokista. Mieti, mitä ominaisuuksia varjoltasi haluat. Haluatko päästä lujaa vai hiljaa? Hidas kupu antaa enemmän anteeksi kuin nopea. Vai haluatko pehmeän avauksen ja varmuuden siitä, että pystyt laskeutumaan turvallisesti pienellekin varalaskeutumisalueelle? 


Pelkkiä valmistajien siipikuormataulukoita ei kannata tuijottaa, koska oma kokemustasosi ratkaisee. Kannattaa myös pyrkiä hyppäämään koehyppyjä useilla erityyppisillä varjoilla ja keskustella esim. kerhon turvallisuuspäällikön, koulutuspäällikön, kalustopäällikön tai hyppymestarien kanssa ennen hankintapäätöksen tekemistä. Kannattaa kysyä kokeneemmilta hyppääjiltä heidän arviotaan siitä, ovatko laskeutumisesi ja kuvunkäsittelytaitosi sillä tasolla, että vaihtamista pienempään tai tehokkaampaan varjoon kannattaa harkita. Ota kuitenkin huomioon, että jokaisella on kokemukseensa perustuva näkemys eri varjoista. Se, mikä toiselle on hidas, saattaa olla toiselle nopea ja suorituskykyinen. 


Jos päätät vaihtaa varjosi pienempään, mieti miksi haluat sen tehdä. Haluatko lisää vauhtia, pidempiä fleerejä vai suurempaa käännösnopeutta? Vai haluatko ehkä useampaa ominaisuutta yhdessä? Eri varjoilla on erilaisia ominaisuuksia, joten kannattaa ottaa niistä selvää, eikä vain miettiä varjon kokoa ja siipikuormaa. Muista myös pienemmän varjon haittavaikutukset: suurempi sakkausnopeus, anteeksiantamattomuus virheen sattuessa, suurempi vauhti jos (kun) virhe tulee, ongelmat pienien varalaskeutumispaikkojen kanssa, sopivuus omaan hyppylajiin jne. 


Mieti siis pienempään varjoon vaihtamista harkitessasi seuraavia vaihtoehtoja:  

\begin{itemize}
\item  Ottaisitko sittenkin harjoittelemalla lisää tehoja ja vauhtia irti vanhasta varjostasi? 
\item  Vaihdatko samankokoiseen, mutta erimalliseen kupuun? 
\item  Haluatko vauhtia suuremman siipikuorman avulla? 
\end{itemize}

Liian pienellä ja suorituskykyisellä kuvulla ei voi eikä kannata yrittää kompensoida puutteellisia kuvunkäsittelytaitoja. Varsinkin ensimmäisellä omalla varjolla kannattaa hypätä useita satoja hyppyjä, jolloin oppii oikeasti hallitsemaan varjoaan erilaisissa paikoissa ja olosuhteissa. Se, että saa C-lisenssin ja saa hypätä periaatteessa millä tahansa varjolla, ei todellakaan tarkoita sitä, että pitää ostaa pienempi varjo. Vasta sitten, kun ensimmäisen varjon kaikki resurssit on saatu käyttöön, kannattaa pienempään vaihtamista harkita, jos todella haluaa lisää vauhtia. Tällöin valmiiksi kerätyt taidot auttavat uuden varjon lentämisen opettelussa, eikä niitä tarvitse opetella alusta alkaen nopeammalla varjolla.  

\section{ Lajien asettamat vaatimukset varjon valintaan }
\label{varjon-valinta-lajien-asettamat-vaatimukset-varjon-valintaan}


Eri hyppylajit vaativat myös varjoilta erilaisia ominaisuuksia, vaikka periaatteessa tavallisella yleisvarjolla voi hypätä lähes lajia kuin lajia. Seuraavista suosituksista huolimatta erityisen tärkeää on, että tunnet sen varjon, jolla hyppäät, ja tiedät, miten se käyttäytyy avauksissa. Tarkempia kyseisten lajien varjokalustolle asettamia vaatimuksia löydät kunkin lajin erityisoppaasta. Seuraavassa esimerkkejä: 


\textbf{Kuvaushypyillä} on tärkeää, että varjo avautuu mahdollisimman pehmeästi ja yllätyksettömästi joka avauksessa. Kamerakypärän kanssa hyppäävän kannattaakin valita konservatiivinen varjo, jolloin mahdolliset vajaatoiminnat eivät ole niin radikaaleja ja kamerakypärän kanssa toimimiseen jää enemmän pelivaraa. 


\textbf{Liitohypyillä} todennäköisin vajaatoiminta on kierteet. Tämän vuoksi varjon tulisi olla rauhallinen ja siipikuorman kohtuullinen, jolloin kierteet eivät välttämättä johda rajuun pyörimiseen ja varavarjon käyttöön. Avauksen aikana ei ole mahdollista estää kierteen syntymistä, koska käsisiivet rajoittavat käsien käyttöä (nostamista kantohihnoille). Lisäksi käsien ja jalkojen vapauttaminen siivistä avauksen jälkeen vaatii hieman voimistelua, jolloin painopisteen muutoksesta herkästi ohjautuva varjo ei ole hyvä vaihtoehto. Nykyisien erittäin tehokkaiden liitopukujen myötä ongelmia ovat alkaneet aiheuttaa myös erittäin suuren siipien koon ja pienen kuvun yhdistelmät. Liitopuvun suuri siipipinta-ala aiheuttaa suuren turbulenssikuplan hyppääjän taakse. Tämä yhdistettynä erittäin pieneen varjoon jossa on lyhyet punokset, voi aiheuttaa tilanteen jossa varjo imeytyy avauksessa turbulenssikuplaan. SIL suosittelee liitopuvulla hypättäessä käytettäväksi A- ja B-lisenssihyppääjille soveltuvaa varjoa. 


\textbf{Skysurf-hypyille} pätevät käytännössä samat suositukset kuin liitohypyillekin. Erityisesti aloittelevalle skysurf-hyppääjälle laudan kanssa hyppääminen voi aiheuttaa avausasentoon epäsymmetrisyyttä. Ongelman tullen on lisäksi enemmän tekemistä ylimääräisten toimenpiteiden vuoksi. 


\textbf{Näytöshypylle} kannattaa valita aina tuttu varjo, mieluummin liian suuri kuin pieni. Näytöshypyillä alastuloalueet ovat yleensä pieniä ja hyppääjän täytyy tarvittaessa osata tehdä hyvinkin lyhyt loppulähestyminen ja fleeri. Myös muualle kuin varsinaiselle alastuloalueelle laskeutuminen täytyy ottaa huomioon. 


Kaikenlaisilla \textbf{kisahypyillä} rajoituksia asettaa usein rajallinen laskeutumisalue. Varjo kannattaa valita niin, ettei se pakota ottamaan vauhtia laskeuduttaessa ja tarvittaessa sen kanssa voi jäädä hetkeksi korkealle odottamaan matalammalla olevien laskeutumista. Kisoissa kannattaa käyttää tuttua ja varmaa varjoa, jolloin itse kilpailusuoritukseen keskittymiseen jää paremmin resursseja. 


\textbf{Freeflyssa} avaushetkellä saattaa olla enemmän vauhtia kuin muissa lajeissa, joten avausmukavuuden vuoksi kannattaa välttää erityisen nopeasti aukeavia varjoja. 


\textbf{Isoilla FS/FF-kuvilla} taivaalla on paljon väkeä avaushetkellä, eikä tällöin avauksessa helposti kääntyvä tai kierteitä tekevä varjo ei ole hyvä valinta. Lisäksi kannattaa välttää erittäin nopeita varjoja, kun taivaalla on muutenkin ahdasta. 


\textbf{FS-hyppäämisessä} pitää varjoa valittaessa muistaa mahdollisten lisäpainojen käytön vaikutus siipikuormaan. Erityisesti tämä korostuu juuri kevyillä hyppääjillä, joilla siipikuorman suhteellinen kasvu on huomattava ja pienillä varjoilla joiden lento-ominaisuudet muuttuvat paljon pienelläkin siipikuorman lisäämisellä. Esim. jos hyppääjän normaali exit-paino on 60 kg ja hän hyppää 120 neliöjalkaisella varjolla, nousee 8 kg lisäpainoilla siipikuorma 1,1:stä 1,25:een. Tämä vaikuttaa jo huomattavasti varjon lento-ominaisuuksiin. Jos hyppäät FS:ää, keskustele kokeneiden FS-hyppääjien kanssa painojen tarpeestasi, ja varaudu tarvittaessa 5-10 kg painojen käyttöön varjoa valitessasi.  


\textbf{Canopy piloting} on oma lajinsa, ja kilpailuissa käytetään yleensä ristituettuja tai muita suhteellisen pieniä ja erittäin suorituskykyisiä varjoja. Huomioi oma taitotasosi, jos alat harjoitella swooppeja. Pienemmillä siipikuormilla ja yleisvarjoilla on turvallisempaa harjoitella. Harjoittele ensin tekniikka ja lisää vasta sitten varjon suorituskykyä. 


\textbf{Kupumuodostelmiin} käytettävät varjot ovat yleisvarjoihin verrattuina erilaisia. Niissä on muiden varjojen punoksia paksummat punokset, isot ohjauslenkit, mahdollisesti ylimääräisiä apuvälineitä varjon asetuskulman muuttamiseen ja apuvarjo, jonka yhdyspunos jää varjon avauduttua kuvun sisälle. Nämä varjot lentävät varsin jyrkässä kohtauskulmassa, ja laskeutumistekniikat saattavat olla vaativia. Avaukset ovat kovia, eikä näitä varjoja ole suunniteltu avattavaksi täydestä vapaapudotusnopeudesta. Käytä CF-hyppäämiseen vain kyseiseen lajiin suunniteltuja varjoja. 


\textbf{Tarkkuushyppyvarjot} ovat perinteisesti F-111-kankaasta valmistettuja isokokoisia 7-tunnelisia korkeasiipiprofiilisia varjoja, joilla pystyy lentämään hyvin hitaasti ja tarvittaessa laskeutumaan pystysuoraan alas. 

