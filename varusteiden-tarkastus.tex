
Laskuvarjohyppääjän varusteet on tarkastettava aina ennen hypylle lähtöä ja uloshyppyä. Oppilaiden varusteet tarkastaa kouluttaja maassa ja koneessa sillä poikkeuksella, että kouluttaja tarkastaa jatkokoulutuksessa olevan oppilaan varusteet ennen koneeseen nousua, mutta oppilas itse vastaa varusteidensa kunnosta ja tarkastuksesta koneessa ennen uloshyppyä. Jatkokoulutuksessa olevat oppilaat ja itsenäiset hyppääjät ovat velvollisia tarkastamaan toistensa varusteet aina ennen uloshyppyä. Varusteiden tarkastamisesta alkaa varsinainen hyppysuoritus ja se päättyy maassa suoritettavaan arviointiin. 


Lainavarusteilla hyppäämisessä on aina riskinsä. Hyppääjä ei pysty oppimaan kaikkia eri avausjärjestelmiä automaatiotasolle, vaan joutuu joka kerta varusteiden muuttuessa keskittymään erityisesti avaustoimenpiteisiin. Näin lainavarusteilla hyppäävän suoritustaso laskee hypyn muulta osalta. Lainavarusteilla hypättäessä tulee aina harjoitella huolellisesti uuden varusteen käyttöä, esimerkiksi apuvarjon tasku voi olla lainavarjossa tiukempi kuin omassa. Huono valmistautuminen lainavarusteen käyttöön on ollut myötävaikuttava tekijä osassa onnettomuuksia. Omat hyppyvarusteet luovat varmuutta hypyille. Lisäksi niistä huolehtiminen on helpompaa ja niiden kunto tiedetään aina. Koneessa on syytä tarkkailla myös muiden hyppääjien hyppyvarusteita. Ajoissa havaittu epäkohta voi estää vaaratilanteen syntymisen. Varusteet tarkistetaan kolme kertaa ennen uloshyppyä. 

\section{ Varusteiden valinta }
\label{varusteiden-tarkastus-varusteiden-valinta}


Varusteet jokainen hyppääjä valitsee itse painonsa, kokemuksensa sekä kouluttajan ohjeet huomioon ottaen. Varusteita valittaessa on huomioitava seuraavaa: 

\begin{itemize}
\item  Henkilökohtaisen vaatetuksen on oltava vuodenaikaan ja säähän sovelias. Hyppyvaatteiden on oltava lämpimiä, mutta samalla myös joustavia. Turha ja liika vaatetus vaikeuttaa suoritusta sekä jäykistää ja hiostuttaa. 
\item  Varjokokonaisuuden ja automaattilaukaisimen asiakirjat on tarkastettava. Pakkausten on oltava alle 12 kk ikäiset ja määräaikaistarkastusten on oltava voimassa. Oppilaspäävarjo sekä varavarjo ja reppu-valjasyhdistelmä tarkastetaan vuoden välein. Automaattilaukaisimet tarkastetaan ja huolletaan 1–4 vuoden välein laitteesta riippuen. 
\item  Varusteet ovat ulkoisesti ehjät ja siistit. Varavarjon vaijerista ja irtipäästökaapeleista tarkastetaan myös niiden vapaa kulku suojaputkissa. 
\item  Automaattilaukaisin säädetään tai se käynnistetään. 
\item  Muista hypyllä tarvittavista varusteista tarkastetaan korkeusmittarin, kypärän, suojalasien, koukkupuukon, haalarin, sormikkaiden, kenkien ja pelastusliivien(tarvittaessa) kunto, toimivuus ja soveltuvuus laskuvarjohyppykäyttöön. Varusteissa ei saa olla ulokkeita tai osia, jotka voivat tarttua kiinni. 
\item  Varusteet ovat sopivat ja säädöt kunnossa. Tarvittaessa valjaiden pituussäätöä muutetaan ja haalari jne. vaihdetaan sopivampiin huomioiden hyppysuoritus ja sääolosuhteet. 
\end{itemize}

Tarkastukseen on varattava riittävästi aikaa, sillä kiireessä valitut ja sovitetut varusteet voivat aiheuttaa hypyllä ongelmia. Esimerkiksi itseaukaisukahva ei ole muuttuneista valjassäädöistä johtuen totutulla paikalla tai hyppylasit jäävät löysälle ja tippuvat. Huolellinen ensimmäinen tarkastusvaihe nopeuttaa muita vaiheita. 

\section{ Maassa ennen koneen lastaamista }
\label{varusteiden-tarkastus-maassa-ennen-koneen-lastaamista}


Varusteiden toinen tarkastusvaihe suoritetaan, kun varusteet on puettu päälle. Tarkastus tehdään ennen koneeseen nousua. Varusteiden tarkastuksessa noudatetaan aina samaa, seuraavassa lueteltua mallia: 

\begin{itemize}
\item  Varusteet on puettu oikein. 
\item  Kaikki hypyllä tarvittavat varusteet ovat mukana oikeilla paikoillaan. Korkeusmittari on näkyvissä ja nollattu sekä koukkupuukko on saatavilla. Kahvat ovat omilla paikoillaan ja ne näkyvät. 
\item  Kaikki lukot, soljet ja hihnat ovat kiinni. Hihnojen päät on työnnetty valjaskumilenkkien alta hihnataskuihin. 
\item  Automaattilaukaisin on päällä ja se on oikein säädetty. 
\item  Hyppykaveri tarkastaa takaa, että pää- ja varavarjon läpät ovat kiinni. Pinnitarkastus suoritetaan, jos läppä tai läpät ovat auenneet I vaiheen jälkeen varusteita päälle puettaessa. 
\end{itemize}

Yleisimmät virheet varusteiden pukemisessa löytyvät hihnoista. Ne ovat joko kierteellä, löysinä tai niiden päät eivät ole valjaskumilenkkien alla. Myös mittarin asetus on voinut muuttua ja repun läpät aueta, jos varusteet päälle puettuna on liikuttu varomattomasti. Kypärä, suojalasit tai sormikkaat unohtuvat helposti, ellei niitä varata ja pueta päälle ennen tätä tarkastusvaihetta. 

\section{ Ennen uloshyppyä }
\label{varusteiden-tarkastus-ennen-uloshyppya}


Tarkastus ennen uloshyppyä aloittaa varsinaisen hyppysuorituksen. Tarkastus aloitetaan 100–300 metriä ennen uloshyppykorkeutta. Tarkastuksessa noudatetaan samaa periaatetta kuin edellisessäkin vaiheessa. Tarkastus on silmämääräinen ja siinä tehdään vain tarvittavat muutokset ja kiristykset. Liikkuminen koneessa ja uloshyppy on aina suoritettava varoen ja varusteita suojaten. Tässä vaiheessa tarkastetaan seuraavat asiat: 

\begin{itemize}
\item  Kypärä ja sormikkaat ovat puettuna päälle ja hyppylasit kiristettyinä hyppykireyteen. 
\item  Varavarjon pakkolaukaisuhihnan lukko on kiinni ja paikoillaan. 
\item  Korkeusmittari näyttää oikein ja on luettavissa. Automaattilaukaisin FXC-12000 on JUMP-asennossa. 
\item  Irtipäästöpampula, varavarjonkahva ja päävarjon avauskahva ovat paikallaan sekä näkyvissä (pois lukien BOC eli repun pohjassa sijaitseva avauskahva, jonka pitää olla esteettömästi saatavilla, tarkista sijainti koskettamalla BOC-avauskahvaa). 
\item  Rinta- ja reisihihnat ovat kireällä ja mahdolliset lukot ovat kiinni. 
\item  Hyppykaveri tarkastaa takaa, että pää- ja varavarjon läpät ovat kiinni. Pinnitarkastus suoritetaan, jos läppä tai läpät ovat auenneet. Tarraläpän alla oleva pinni on hyvä tarkastaa aina. 
\end{itemize}

Varusteiden tarkastus ennen uloshyppyä voi huolimattomasti suunnitellulla hypyllä jäädä kokonaan tekemättä. Rutiinista poikkeaminen luo hypylle epävarman tunteen, joka aiheuttaa usein hyppysuorituksen epäonnistumisen. Varjon avautuminen koneessa aiheutuu yleensä turhasta ja huolimattomasta liikkumisesta tai huonokuntoisesta luupista. Hyppyovea ei saa koskaan avata, jos koneessa on varjo auki, vaan kaikki tulevat silloin koneella alas. Varusteiden itsenäinen tarkastus vaaditaan jokaisella hypyllä tämän koulutuksen jälkeen. Kouluttaja vastaa edelleen oppilaasta koko jatkokoulutuksen ajan. 

\section{Harjoitus}
\label{varusteiden-tarkastus-harjoitus}

\begin{enumerate}[label=\bfseries \arabic*)]
\item  Suoritetaan tarkastukset opetuksen yhteydessä. 
\item  Tutustutaan koneesta mahdollisesti löytyviin paikkoihin, joihin hyppyvarusteet voivat takertua. 
\end{enumerate}
