\section{ Tarkkuushyppy oppilassuorituksena }
\label{kuvunkasittely-oppilaana-tarkkuushyppy-oppilassuorituksena}


Oppilastarkkuuden tarkoituksena on opettaa varjon hallintaa ja kehittää korkeuden ja etäisyyden arviointikykyä. Tavoitteena olisi se, että jatkokoulutusvaiheessa oppilas osaisi laskeutua turvallisesti lähelle suunniteltua pistettä. Ohjauskuviot ja finaalin aloituspiste tulee suunnitella ja käydä mielessä läpi etukäteen. Laskeutumiskuvion voi lentää puolijarruilla, jotta tuulen vaihteluihin voidaan reagoida. Perusosa ajetaan siten, että finaaliin (tuulilinjalle) käännytään tuulesta riippuen 50–100 metriä maalin takana. Finaali aloitetaan noin 100 metrin korkeudesta. Finaalin alussa säädetään lähestymiskulma oikeaksi joko jarruttamalla tai nostamalla. S-mutkia ei saa tehdä etenkään, jos ilmassa on muitakin hyppääjiä. Valmistautuminen loppuvetoon aloitetaan 30–50 metrin korkeudessa, jolloin viimeistään tulee jarrutusta keventää eli käsiä nostaa ylemmäs. Ohjauslenkkejä ei saa nostaa jarrutustilasta täysiliitoon nopeasti, sillä tällöin varjon vajoamisnopeus kasvaa hetkellisesti ja liian matalalla se voi olla kohtalokasta. Turvallisuus on pidettävä kaiken aikaa mielessä. Varjon sakkauttamista on varottava, eikä matalalla saa tehdä jyrkkiä käännöksiä. 

\subsection{ Hypyn suunnittelu }
\label{kuvunkasittely-oppilaana-hypyn-suunnittelu}


Myös tarkkuushypyssä hypyn huolellinen suunnittelu etukäteen on tärkeää. Maassa tulee tarkkailla sääolosuhteiden, erityisesti tuulen suunnan ja voimakkuuden kehittymistä. Muiden hyppääjien suorituksia seuraamalla voi havaita mahdolliset nostavat tai laskevat ilmavirtaukset sekä turbulenssit. Ennen hypylle lähtöä selvitetään uloshyppypaikka, sektori, josta laskeutumisalueelle pääsee sekä alustavasti ohjauskuviot ja finaalin aloituspiste. 

\subsection{ Uloshyppy, avaus ja alkulähestyminen }
\label{kuvunkasittely-oppilaana-uloshyppy-avaus-ja-alkulahestyminen}


Uloshypyn ja avauksen jälkeen alkulähestymisen aikana tulee varmistaa pääsy laskeutumisalueelle. 

\subsection{ Lähestymiskuvio }
\label{kuvunkasittely-oppilaana-lahestymiskuvio}


Lähestymiskuvion aikana maali ja tuulipussi tulee pitää kaiken aikaa näkyvissä. Tuulen suunnan ja voimakkuuden kehittymistä on tarkkailtava koko ajan. Myötätuuli- ja perusosa lennetään puolijarruilla. Perusosan aikana varjoa sivuttain luistattamalla voidaan tarpeen mukaan siirtää finaalin aloituspistettä joko lähemmäksi tai kauemmaksi aiotusta pisteestä. 

\subsection{ Finaali }
\label{kuvunkasittely-oppilaana-finaali}


Finaalipisteessä varjo käännetään tarkasti tuulilinjalle (tuuli on suoraan maalipisteen suunnasta). Jarruilla säädetään lähestymiskulmaa. Varjo pidetään koko ajan vastatuuleen. Valjaissa tulee olla rentona ja suorana. Varjoa ohjataan loppuun asti maalipistettä kohti, mutta ei yritetä väkisin osua maaliin esimerkiksi liian kovalla jarrutuksella tai matalalla käännöksellä. 

\section{ Kuvunkäsittelyhypyt }
\label{kuvunkasittely-oppilaana-kuvunkasittelyhypyt}


Nopeilla varjoilla lentäminen on kuin lentäisi lentokoneella. Niissä on suorituskykyä ja ominaisuuksia, jotka tekevät liikkumisen ilmassa hauskaksi ja oikein käytettyinä entistä turvallisemmaksi, mutta niissä on samalla myös ominaisuuksia, jotka tekevät niistä väärin käytettyinä vaarallisia. Varjon ominaisuudet on tunnettava, jotta sillä voidaan lentää turvallisesti niissä olosuhteissa, jotka valmistajan ja määräysten mukaan mahdollistavat hyppäämisen. Ääriolosuhteet ja maksimiohjausliikkeet eivät sovi yhteen. 


Ääriolosuhteita ja asioita, joita tulisi välttää, ovat 

\begin{itemize}
\item  kova tuuli, erityisesti puuskainen ja turbulenttinen keli 
\item  toisen jättövirtauksessa lentäminen 
\item  pienet laskeutumisalueet 
\item  väsyneenä hyppääminen 
\item  turhien riskien ottaminen. 
\end{itemize}
\subsection{ Huomioitavia asioita }
\label{kuvunkasittely-oppilaana-huomioitavia-asioita}

\subsubsection{ Vapaapudotus }
\label{kuvunkasittely-oppilaana-vapaapudotus}

\begin{itemize}
\item  Hypätään oikeasta uloshyppypaikasta. 
\item  Tarkastetaan sijainti jo vapaan aikana. 
\item  Puretaan oikeassa korkeudessa sekä riittävän korkealla tasoon ja suoritukseen nähden. 
\item  Liu’utaan eroon muista, tarkastetaan ilmatila ja annetaan avausmerkki. 
\end{itemize}
\subsubsection{ Avauksen aikana }
\label{kuvunkasittely-oppilaana-avauksen-aikana}

\begin{itemize}
\item  Avataan riittävän korkealla, stabiilissa asennossa, katse horisonttiin ja olkapäät horisontin tasossa. Ei katsota olan yli apuvarjoa. Jo avauksen aikana kannattaa tarkkailla ilmatilaa muiden varjojen varalta. 
\item  Tehdään päätös: LENTÄÄ / EI LENNÄ (⇒ varavarjotoimenpiteet). 
\item  Laitetaan kädet takimmaisille kantohihnoille, väistetään tarvittaessa (oikealle) ja tarkastetaan ilmatila. 
\item  Ohjataan takimmaisista kantohinnoista ja etsitään muut. 
\item  Tarkastetaan oma sijainti ja korkeus. 
\item  Tarkastetaan irtipäästöpampula ja varavarjon kahvan paikka ja kiinnitys. 
\item  Käännetään takimmaisista kantohinnoista haluttuun lentosuuntaan. 
\item  Tukahdutetaan slider ja avataan puolijarrut. 
\item  Tehdään varjon ohjauskokeilu viimeistään 600 metrin korkeudessa. 
\end{itemize}
\subsubsection{ Lentäminen}
\label{kuvunkasittely-oppilaana-lentaminen}

\begin{itemize}
\item  Katsekontakti 90 \% ympärille (lentosuunta, sivut, alas, ylös ja taakse) ja 10 \% maahan.  
\item  Annetaan muillekin lentotilaa: ei mennä lähelle, jos ei ole etukäteen sovittu. 
\item  Porrastetaan liikenne mahdollisimman ylhäällä, ei vasta laskeutumiskuviossa. 
\item  Huomioidaan väistämisvelvollisuus: varavarjot, hitaammat kuvut, alemmat ja oikealta tulevat. 
\item  Ohjaillaan varjoa rauhallisesti ja keliä tunnustellen (turbulenttisuus). 
\item  Päätetään tarvittaessa ajoissa varalaskupaikka ja laskusuunta. 
\item  Tarkkaillaan korkeutta ja korkeusmenetystä liikkeiden aikana. 
\end{itemize}
\subsubsection{ Varjon varassa törmäämisestä }
\label{kuvunkasittely-oppilaana-varjon-varassa-tormaamisesta}


Uudet punosmateriaalit aiheuttavat törmäämistilanteissa nykyaikaisella varjokalustolla saavutettavilla lentonopeuksilla hyvin suuria vammoja osuessaan ja takertuessaan hyppääjään. Tällaisia punosmateriaaleja on käytössä liki kaikissa nopeissa kuvuissa. 


Ohuita punosmateriaaleja käyttävillä varjotyypeillä törmättäessä on suositeltava toimintamalli palloksi käpertyminen, jolla koetetaan välttää punoksiin takertuminen ja päästään mahdollisesti punosten välistä. 


Pyri estämään törmääminen varjon varassa väistämällä oikealle ja varoittamalla huutamalla. Jos törmäämistä varjon varassa ei voi välttää, niin menetellään seuraavasti: 

\begin{itemize}
\item  Minimoidaan törmäysvauhti jarruttamalla varjoa voimakkaasti. 
\item  Käperrytään palloksi. 
\item  Jos varjot sotkeutuvat: 
	\begin{itemize}
	\item  tarkkaillaan korkeutta 
	\item  keskustellaan toimintajärjestyksestä 
	\item  varmistetaan, että ollaan irti punoksista/kuvusta (käytetään tarvittaessa koukkupuukkoa) ennen varavarjotoimenpiteitä. 
	\end{itemize}
\item  Kahden hyppääjän ei ole turvallista laskeutua yhdellä erittäin raskaasti kuormitetulla päävarjolla. 
\end{itemize}
\subsubsection{ Laskeutuminen }
\label{kuvunkasittely-oppilaana-laskeutuminen}

\begin{itemize}
\item  Noudatetaan laskeutumissääntöjä: kuvion suunta, esteet ja rajoitukset. 
\item  Tarkkaillaan ympäristöä, tuulta ja maata. 
\item  Suunnitellaan loppukuvio ajoissa. Huomioidaan esteet, ihmiset, muu liikenne, turbulenssi ja tuuli. 
\item  Ei tehdä jyrkkiä käännöksiä. 
\item  Valitaan laskeutumisalue (lyhyt, pitkä) taitojen mukaan. 
\item  Ei yritetä väkisin tiettyyn paikkaan, vaan laskeudutaan rauhallisesti ja turvallisesti.  
\item  Tehdään oikeaoppinen loppuveto (\ref{laskeutumistekniikat-oikeaoppinen-loppuveto} s.\pageref{laskeutumistekniikat-oikeaoppinen-loppuveto}). Kaikki liikkeet tehdään rauhallisesti eikä varjoa sakkauteta. Varjo ei anna anteeksi epäsymmetrisyyttä. 
\end{itemize}

Vastatuulen sijaan voidaan tulla mihin suuntaan tahansa tilanteen niin vaatiessa. Tärkeintä on laskeutua vapaaseen suuntaan ilman matalalla tehtyjä jyrkkiä käännöksiä. 

\begin{itemize}
\item  Pidetään ohjaussuunta loppuun asti eikä oteta vastaan kädellä tai jalalla. 
\item  Ei pumpata loppuvedossa ohjauslenkeillä. 
\item  Jos loppuveto tehdään liian korkealla, pidetään ohjauslenkit alhaalla tai löysätään jarrutusta vain hieman. Jalat pidetään tiukasti yhdessä. 
\item  Tarkkaillaan korkeutta, sillä loppuveto perustuu sekä näköhavaintoon että tuntemukseen. 
\end{itemize}
\subsubsection{ Maahantulon jälkeen }
\label{kuvunkasittely-oppilaana-maahantulon-jalkeen}

\begin{itemize}
\item  Tukahdutetaan kupu vetämällä toinen ohjauslenkki alas. 
\item  Tarkastetaan takaa tulevien lentosuunnat. 
\item  Asetetaan ohjauslenkit tarroihin ja avataan slider. 
\item  Kootaan varjo ja poistutaan laskeutumisalueelta. 
\item  Huomioidaan muut hyppääjät ja liikenne koko ajan. 
\end{itemize}

Hyppyohjelman läpivienti edellyttää sopivat sääolosuhteet myös yli 600 metrin korkeudessa, jotta käännökset, sakkauskokeilut jne. voidaan suorittaa turvallisesti eikä ajauduta pois kenttäalueelta. Alle 600 metrin ei tehdä ääriohjausliikkeitä. FXC-12000 ja oppilas-Cypres voivat toimia, mikäli varjon varassa porataan tai tehdään voimakkaita käännöksiä alle 600 metrin korkeudessa. Hyppyohjelman suorittaminen jatkokoulutuksessa kuuluu koulutusohjelmaan. Kuvunkäsittelyhypyillä ei tehdä vapaapudotussuorituksia. 

\subsection{ Hyppy 1 }
\label{kuvunkasittely-oppilaana-hyppy-1}


Tarkoituksena on oppia tuntemaan oman varjon ominaisuudet ja oppia tekemään loppuveto. Hyppykorkeus on 2000 metriä, josta hypätään 8–10 sekunnin vapaa. Ennen jarrujen avausta kokeillaan ohjausta sekä fleeraamista takimmaisista kantohinnoista. Näin voidaan harjoitella nopeaa väistämistä heti avauksen jälkeen. 


Avataan puolijarrut ja etsitään sakkauspiste. Varjoa ei sakata, vaan etsitään jarrujen piste, jossa pieni lisäys aikaansaisi sakkauksen. Jos varjo sakkaa, nostetaan ohjauslenkkejä ylös hitaasti ja symmetrisesti.  


Tämän jälkeen harjoitellaan loppuvetoa täysiliidosta. Muistetaan terävä alkuveto ja lisätään sen jälkeen jarrutusta vähitellen lähelle sakkauspistettä. Huomioidaan lisääntyvä G-voima ja varjon siirtyminen vaakalentoon. 


Kokeillaan 90 asteen käännöksiä täysiliidosta ja huomioidaan korkeuden menetys ja ilmanopeuden kasvu verrattuna oppilasvarjoon. Valmistaudutaan alastulokuvioon huomioiden porrastukset. Tehdään oikeaoppinen loppuveto täysiliidosta. 

\subsection{ Hyppy 2 }
\label{kuvunkasittely-oppilaana-hyppy-2}


Hyppykorkeus on 2000 metriä, josta hypätään 8–10 sekunnin vapaa. Avataan puolijarrut. Kokeillaan rauhallisia käännöksiä eri lentotiloissa (puolijarrutus, täysijarrutus) molempiin suuntiin. Kokeillaan käännöksiä takimmaisista kantohihnoista molempiin suuntiin. Ei kuitenkaan päästetä ohjauslenkkejä käsistä. 


Kokeillaan nopeaa 360 asteen käännöstä (poraamista) ja pysäytystä ennalta päätettyyn suuntaan. Huomioidaan, kuinka ilmanopeus kasvaa ja korkeus vähenee nopeasti. Huomioidaan kuinka paljon aikaisemmin käännös täytyy lopettaa, että pysäytys tapahtuu valittuun suuntaan, ja kuinka suuri vastaliike täytyy tehdä, että käännös pysähtyy. Seurataan korkeuden menetystä käännösten aikana. (HUOM. Jos varjossa on FXC-12000 tai oppilas-Cypres, ei poraamista saa tehdä alle 600 metrin korkeudessa, koska varavarjon automaattilaukaisin voi toimia). 


Valmistaudutaan alastulokuvioon ja tehdään oikeaoppinen loppuveto täysiliidosta. 

\subsection{ Hyppy 3 }
\label{kuvunkasittely-oppilaana-hyppy-3}


Uloshyppykorkeus ja vapaa ovat samat kuin edellisissä hypyissä. Harjoitellaan loppuvetoa puolijarrutustilasta. Tullaan havaitsemaan, että tarvittava veto on terävämpi kuin täydestä liidosta. Finaali ja loppuveto tehdään puolijarrutustilasta. 

