
Siipikuorma ilmaisee, miten paljon painoa kuvulla on kannatettavanaan sen pinta-alaan nähden. Siipikuorma ilmaistaan yleensä nauloina (1 lbs = 0,454 kg) neliöjalkaa (1 ft² = 0,093 m²) kohti. Oman siipikuormasi saat, kun punnitset itsesi hyppyvarusteet ja varjokokonaisuus ylläsi (exit-paino), kerrot painon 2,2:lla (lbs/kg) ja sen jälkeen jaat saamasi luvun varjosi pinta-alalla neliöjaloissa.  

\begin{description}
\item[ ] \hfill \\ 
\textit{Esim. 85 kg:n exit-paino, 150 ft² varjolla: 85 kg * 2,2 (lbs/kg) = 187 (lbs) ja 187 (lbs) / 150 (ft²) = 1,25 (lbs/ft²)} \hfill \\ 
\end{description}

Muista, että siipikuorma ei ole tarkka vertailuarvo kahden eri varjon suorituskyvyn välillä. Erityyppisillä ja -kokoisilla varjoilla saadaan samalla siipikuormalla erilainen suorituskyky. Varjot on suunniteltu toimimaan jollakin siipikuormavälillä, ja yli- tai alikuormaaminen huonontaa aina joitain ominaisuuksia. Joillakin varjoilla voidaan lentää hyvinkin suurella siipikuormavälillä, toisilla järkevä toiminta-alue on pienempi. 


Samantyyppisen varjon aerodynaaminen tehokkuus samalla siipikuormalla huononee varjon koon pienentyessä (esim. saumojen osuus pinta-alasta kasvaa). Samoin varjon herkkyys ja aggressiivisuus kasvavat mm. lyhyempien punosten aiheuttaman nopeamman reagoinnin vuoksi. Tämän vuoksi samaa suorituskykyä ja ohjaustuntuma varten kevyen hyppääjän kannattaa lentää hieman pienemmällä siipikuormalla kuin painavan. Esimerkiksi eri painoiset muuten samoista lähtökohdista ensimmäistä varjoaan valitsevat hyppääjät voivat päätyä eri siipikuormiin, jotta varjojen lento-ominaisuudet olisivat samantyyppiset. 


Seuraavassa on varjojen suorituskykyikkunoita. Optimaalinen käyttöalue on välin keskivaiheilla. Lista on vain suuntaa antava. 

\begin{itemize}
\item  oppilasvarjot (Navigator, Skymaster)  0,5 – 1,0 
\item  tarkkuusvarjot (Parafoil, Classic, Opus) 0,5 – 0,8 
\item  F-111-varjot (PD9-cell, Silhouette, Fury) 0,7 – 1,1  
\item  yleisvarjot (Sabre, Sabre2, Hornet, Triathlon, Spectre, Pilot, Safire) 0,9 – 1,5 
\item  I sukupolven elliptiset (Stiletto, Contrail, Springo) 1,3 – 1,7 
\item  II sukupolven elliptiset (Crossfire, Katana, Vengeance)  1,5 – 2,0 
\item  cross-brace-rakenteiset (FX, VX, Velocity, Xaos, Sensei) 1,8 – 2,5 
\end{itemize}

Seuraavassa luokittelussa varjot ja siipikuormat on jaettu kuuteen ryhmään, jotka kuvaavat varjojen suorituskykyeroja ja niiden lento-ominaisuuksia eri rakenne- ja siipikuormaluokissa. Mieti varjoa valitessasi minkä luokan varjolla haluat hypätä ja miksi. 


\textbf{Luokka 1, oppilasvarjot ja tarkkuushyppääminen 0,5 – 0,7 lbs/ft²} 


Tyypillisiä varjoja ovat oppilasvarjot, tarkkuusvarjot ja muut 7- tai 9-tunneliset korkeasiipiprofiiliset F-111-kankaasta valmistetut varjot kokoluokassa > 200 ft². Näillä siipikuormilla hypätään pääasiassa oppilasvarjoilla. Myös jos olet iäkäs tai epävarma kyvyistäsi, voit valita varjon ja siipikuorman tästä luokasta. Varjo kääntyy hitaasti ja menettää käännöksissä vähän korkeutta. Lähestymiset ja laskeutumiset voi tehdä tarkasti ja jopa voimakkaassa jarrutustilassa turvallisesti. Näillä varjoilla ja siipikuormilla ei kannata hypätä yli 8 m/s tuulilla, sillä turbulenssi voi olla vaarallinen, eikä varjon ilmanopeus riitä kumoamaan tuulta. 


\textbf{Luokka 2, rauhallinen 0,8 – 1,0 lbs/ft²} 


Tyypillisiä varjoja ovat luokan 1 varjot hieman raskaammin kuormattuina tai nollakankaasta valmistetut 7- tai 9-tunneliset suorakaiteen muotoiset tai \textit{lievästi elliptiset} varjot kevyesti kuormattuina. Kyseessä on edelleen varsin rauhallisesti käyttäytyvä laskuvarjo. Tämän luokan varjot sopivat edistyneille oppilaille, tai kelpoisuushyppääjille, jotka haluavat pelata varman päälle, sekä esim. näytöshypyille vaikeisiin paikkoihin. Kovat tuulet ja turbulenssi vaikuttavat näihin varjoihin edelleen selvästi enemmän kuin suuremmilla siipikuormilla hypättäviin. Varjo ohjautuu rauhallisesti ja laskeutumiset ovat varsin turvallisia huonollakin laskeutumistekniikalla. Kuitenkin matalalla suoritetut jyrkät käännökset kannattaa jättää tekemättä, koska vauhti ja vajoamisnopeus kasvavat käännöksissä selvästi nopeammin kuin luokan 1 varjoilla. 


\textbf{Luokka 3, keskialue 1,1 – 1,3 lbs/ft²} 


Tyypillisiä varjoja ovat nollakankaiset 7- tai 9-tunneliset suorakaiteen muotoiset tai lievästi elliptiset yleisvarjot kuormattuina niiden optimitoiminta-alueelle. Tätä aluetta voi hyvin käyttää vertailupohjana. Siipikuormaa on tarpeeksi, jotta vauhtia riittää hauskanpitoon, mutta toisaalta tuore kelpoisuushyppääjä pystyy yleensä hallitsemaan varjoa ensimmäisenä varjonaan, kun asennoituu varjon suorituskykyyn oikein. Silti suorituskykyä riittää niin paljon, ettei vaihtotarvetta tule ennen kuin useiden satojen hyppyjen jälkeen, jos silloinkaan. Tällä siipikuorma-alueella on kuitenkin jo oltava tarkkana vaikka korjausvaraa onkin vielä jonkin verran. Suurin osa maailmassa myytävistä varjoista kuuluu tähän käyttöluokkaan. Tuuli-olosuhteet eivät ole enää ongelma, mikäli toimitaan SIL:n toiminnallisten ohjeiden rajoissa (maksimi 11 m/s). Suomen Ilmailuliitto ry on asettanut A- ja B-lisenssihyppääjille siipikuorman ylärajaksi 1,34 lbs/ft² sekä listannut myös kyseisten hyppääjien käyttöön hyväksytyt päävarjotyypit. 


\textbf{Luokka 4, nopeat varjot 1,4 – 1,7 lbs/ft²} 


Tyypillisiä varjoja ovat nollakankaiset 9-tunneliset suorakaiteen muotoiset tai lievästi elliptiset varjot kuormattuina niiden toiminta-alueen ylärajoille tai \textit{elliptiset varjot} suorituskykynsä optimialueella. Tämän luokan varjot ovat erittäin suorituskykyisiä. Samalla siirrytään selkeästi vaarallisemmalle siipikuorma-alueelle. Niin kääntymisnopeus, korkeuden menetys käännöksessä kuin ilmanopeuskin ovat huomattavasti suuremmat kuin luokassa 3. Varjoa on aktiivisesti lennettävä koko laskeutumisen ajan. Tarvittavat ohjausliikkeet ovat selkeästi pienempiä kuin aiemmissa luokissa ja sakkaaminen saattaa tapahtua rajusti ja yllättäen. Tällä siipikuormalla lennettävän laskuvarjon hallintaan tarvitaan jo selkeästi enemmän kokemusta ja taitoa. Hyppääminen tämän ja seuraavien luokkien varjoilla vaatii hyvää rutiinitasoa varjon lentämisessä ja paljon (>100) hyppyjä vuodessa ollakseen turvallista. Tuntuman varjoon on oltava hyvä, ja jokainen liike on suunniteltava hyvissä ajoin etukäteen, koska asiat tapahtuvat nopeasti. Tuulisissa olosuhteissa turbulenssin vaikutukset ovat vähäisempiä, mutta jos jotain kuitenkin tapahtuu, ovat vajaatoiminnot rajuja. Tämä pätee kaikkeen muuhunkin; mihin tahansa törmäät, vauhtia on enemmän. Muista, että liike-energiasi kasvaa nopeuden neliössä. Tässä luokassa epätasainen maasto ja tiukat laskeutumisalueet ovat jo ongelmallisia, koska vauhtia ei saa pysäytettyä nopeasti. 


\textbf{Luokka 5, erittäin nopeat varjot 1,8 – 2,0 lbs/ft²} 


Tyypillisiä varjoja ovat uudet suorituskykyisimmät 9-tunneliset tehokasprofiiliset elliptiset varjot, joissa on sovellettu ilmalukkotekniikkaa tai kuvun etureuna on osittain suljettu sekä ristituetut varjot. Jos kuulut niihin, jotka haluavat lentää tämän luokan varjolla, harkitse tarkkaan mitä olet tekemässä. Varjo lentää todella lujaa ja reagoi erittäin nopeasti ohjausliikkeisiin ja painopisteen muutoksiin valjaissa. Sakkaukset ovat rajuja ja voivat tapahtua yllättäen. Samalla on usein menetetty suorituskykyä hidaslento- ja avausominaisuuksista. Erehtymisiin ei juuri ole varaa, ja jos virhe sattuu, lentää varjo nopeuksilla, joilla se merkitsee todennäköisesti loukkaantumista. Monet tällaisilla varjoilla hyppäävät eivät käytä niiden suorituskyvystä kuin murto-osan. Lähes sama suorituskyky ja fleerit laskeutumisissa olisivat saavutettavissa luokan 4 varjoilla hyödyntämällä niiden tarjoamat lento-ominaisuudet tehokkaammin. Silti tuolloin olisi huomattavasti enemmän varaa virheille, ja muita ominaisuuksia olisi uhrattu vähemmän vauhdin kustannuksella. Älä siirry luokan 4 varjosta tähän luokkaan ennen satojen hyppyjen kokemusta edellisen luokan varjolla. 


\textbf{Luokka 6, kilpailuluokan varjot >2,0 lbs/ft²} 


Tyypillisiä varjoja ovat ristituetut ja muut erittäin ohuen siipiprofiilin kuvut raskaasti kuormattuina. Yleisvarjoksi on harvoin syytä valita tämän luokan varjoa. Yleisesti ottaen vauhti saadaan muiden varjolle tärkeiden ominaisuuksien (avaukset, sakkausominaisuudet, hidaslento) kustannuksella. Jos kuitenkin harrastat pääasiassa nopeita laskeutumisia, hyppäät paljon, olet valmis keskittymään varjolla lentämiseen ja sen määrätietoiseen harjoitteluun ja sinulla on riittävä kokemus edellisten luokkien varjoista, on näillä varjoilla hyppääminen järkevää. Varaa virheisiin ei ole, ja laskeutuminen pienelle alueelle on yleensä erittäin vaativaa. Kisahyppääjät käyttävät yleensä kahta erikokoista varjoa eri lajeihin. Nopeuskilpailuissa lennetään pienellä varjolla ja suurella siipikuormalla, jolloin ilmanvastus saadaan pieneksi. Pituus- ja tarkkuuskilpailuissa käytetään selvästi isompia varjoja ja kompensoidaan siipikuorman pienenemistä lisäpainoilla. Tämä siksi, että samalla siipikuormalla lennettäessä kahdesta samanlaisesta varjosta isompi on aerodynaamisesti tehokkaampi, tuottaa enemmän nostetta hitaassa vauhdissa ja mahdollistaa siten pidemmät fleerit. 

