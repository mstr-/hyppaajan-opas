
Laskuvarjohyppääminen harrastuksena ei ole sitä hurjapäistä touhua, jollaiseksi se yleisesti saatetaan käsittää. Toki laskuvarjourheilussa on omat riskinsä kuten kaikissa vauhtilajeissa, mutta laskuvarjohyppääminen on paljon turvallisempaa kuin yleensä luullaan. Hyppytoimintaa ohjaavat määräykset, lait sekä Suomen Ilmailuliiton ohjeet, jotka lajin harrastajat ovat kehittäneet vuosikymmenten saatossa. Laskuvarjokalusto on oikein käytettynä todella toimintavarmaa. Tämän päivän urheilulaskuvarjohyppääminen onkin riskien hallintaa parhaimmillaan. Yleensä ongelmat johtuvat hyppääjän tekemistä virheistä. 


Laskuvarjohyppääjän koulutusohjelma on osa Suomen Ilmailuliitto ry:n (SIL) valvomaa ja hyväksymää koulutustoimintaa, joka on samanlaisena käytössä koko maassa. Kaikki kouluttajat ovat hyppymestareita, vapaapudotuskouluttajia tai muuten kokeneita hyppääjiä. Suurin osa laskuvarjoharrastustoiminnasta, mukaan lukien koulutus, tapahtuu kerhoissa ympäri maata. Suomessa on myös kaupallisia yrityksiä, jotka toimivat lajin parissa.  


Kurssin aikana käydään läpi kaikki hypyn vaiheet, jotta pystyisit suoriutumaan niistä itsenäisesti ja ennen kaikkea turvallisesti, unohtamatta hauskuutta. Kaikki kurssilla käsiteltävät asiat on osattava, jotta hyppy onnistuisi. Jokainen oppitunti ja harjoitus on tärkeä oman turvallisuutesi kannalta, joten jos sinulle tulee kurssin aikana pakollinen este, sovi kouluttajien kanssa lisäopetuksesta. Voit tehdä tiettyjä harjoituksia myös kotona, tällöin voit ehkä vielä paremmin keskittyä ja miettiä mahdollisia epäselväksi jääneitä asioita. Myös hyppytapahtuman eri vaiheiden läpikäynti mielikuvaharjoitteluna on hyödyllistä.  


Toivomme, että kysyt heti, jos jotain jää epäselväksi; kaikki pitää ymmärtää. Kurssilla opittavat asiat ovat kaikille uusia eikä tyhmiä kysymyksiä ole. Laskuvarjohyppäämisessä on myös paljon slangisanoja, joiden käyttöön kouluttajat joskus sortuvat. Kysy, jos jää epäselvyyksiä! 


Olet voinut jo pitkään haaveilla laskuvarjohyppäämisestä tai saada idean vasta vähän aikaa sitten. Olet joka tapauksessa juuri päättänyt aloittaa harrastuksen, joka toivottavasti tuo mukanaan monia antoisia hetkiä taivaalla ja maassa. Kokeiletpa lajia vain parilla hypyllä tai jäät seuraamme vuosiksi, takaamme sinulle, että ensimmäinen hyppy on kokemus, jota et unohda koskaan. 

\section{ Suomen Ilmailuliitto }
\label{tervetuloa-taivaalle-suomen-ilmailuliitto}


Vuonna 1919 perustettu Suomen Ilmailuliitto ry (SIL) on eri harrasteilmailulajeja edustavien ilmailukerhojen keskusjärjestö. Ilmailuliitossa on yli 200 (2014) jäsenkerhoa, joista pääasiassa laskuvarjohyppäämistä harrastavia kerhoja on 15. Organisaatio, joka antaa koulusta, on sitoutunut noudattamaan Ilmailuliiton hyväksymiä ohjeita.  


Suomen Ilmailuliitossa laskuvarjourheilua koskevien asioiden asiantuntijaelin on luottamushenkilöistä koostuva Laskuvarjotoimikunta, jonka alaisuudessa toimii Kalustokomitea,  Kilpailukomitea sekä Koulutus- ja turvallisuuskomitea. Nämä hoitavat nimiensä mukaisesti kalustoon, kilpailuasioihin, hyppyturvallisuuteen ja -koulutukseen liittyviä asioita.  


Laskuvarjotoimikunta komiteoineen koostuu rivihyppääjistä, jotka haluavat edistää harrastusta ja sen tarpeita Suomessa. Käytännön toimintaa koordinoi liiton toimisto Helsinki-Malmin lentoasemalla.  


Ilmailu on ilmailuviranomaisen valvomaa toimintaa. Suomen Ilmailuliitto antaa jäsenilleen omia ohjeistuksia ja toimii etujärjestönä viranomaisiin päin muun muassa 

\begin{itemize}
\item  Avustamalla asiantuntijana Trafia laskuvarjohyppytoiminnassa 
\item  Laatimalla laskuvarjourheilua koskevat koulutusohjelmat ja tiedottamalla niistä jäsenistölleen 
\item  Valvomalla jäsenorganisaatioiden koulutustoimintaa 
\item  Laatimalla laskuvarjohyppytoimintatilastoja  
\item  Kokoamalla ja analysoimalla onnettomuus- ja vaaratilanneselvityksiä 
\item  Myöntämällä SIL-laskuvarjohyppylisenssit 
\item  Osallistumalla laskuvarjourheilua koskevien määräysten ja ohjeiden valmisteluun 
\end{itemize}

Ilmailu-lehteä Ilmailuliitto on julkaissut jo vuodesta 1938 alkaen. Lehti ilmestyy 10 kertaa vuodessa ja käsittelee Ilmailuliiton omien harrastealojen lisäksi liikenne- ja sotilasilmailua sekä lentoturvallisuuteen liittyviä aiheita. 


SIL:n kotisivuilta osoitteesta \url{http://www.ilmailu.fi} saat halutessasi lisätietoja. 

\section{ Vakuutusturva }
\label{tervetuloa-taivaalle-vakuutusturva}


Vuosittain laskuvarjohyppäämisessä sattuu jonkin verran loukkaantumisia. Tyypillinen vamma on nilkan nyrjähtäminen tai murtuminen huonosti suoritetussa alastulossa. Suomessa kuolemaan johtaneita onnettomuuksia on vuodesta 1962 lähtien sattunut 25 (2013). Kokonaishyppymäärä vuodesta 1962 vuoteen 2013 on noin 1 650 000 hyppyä. Vuosittain Suomessa hypätään n. 50 000 hyppyä. 


Laskuvarjohyppääjä hyppää aina omalla vastuullaan. Kerholla on kolmannelle osapuolelle aiheutetun vahingon korvaava vastuuvakuutus, joka korvaa vain, jos kerho on syyllistynyt virheeseen. Jos oppilas toimii virheellisesti ja aiheuttaa vahinkoa itselleen tai jollekin muulle, hänen on itse vastattava vahingoista. Liittymällä Suomen Ilmailuliittoon saat vakuutuksen, joka sisältää kolmannen osapuolen vastuuvakuutuksen sekä tapaturmavakuutuksen. 


Normaalit vapaa-ajan tapaturmavakuutukset eivät yleensä korvaa laskuvarjohypyllä tapahtunutta vahinkoa, koska hyppääminen kuuluu vakuutusyhtiöiden vaaralliseksi luokittelemiin lajeihin. Mikäli vakuutuksen halutaan korvaavan myös laskuvarjohypyillä tapahtuneet onnettomuudet, on vakuutukseen otettava lisäsuoja.  Lisävakuutuksen voi ottaa omalta vakuutusyhtiöltä tai hyödyntää Suomen Ilmailuliitto ry:n vakuutuksia koskevia etuja. Katso lisätietoja vakuutuksien ehdoista ja korvaussummista Hyvä hyppykurssilainen -esitteestä. Lisätietoja saa myös kouluttajilta ja Ilmailuliiton puhelinnumerosta (09) 350 9340. Suosittelemme laskuvarjotoiminnan kattavan vakuutuksen hankkimista. 

\section{ Terveydentilavaatimukset }
\label{tervetuloa-taivaalle-terveydentilavaatimukset}


Hyppääminen on urheilua ja vaatii fyysistä ja henkistä kuntoa sekä normaalia terveyttä. Sairaudet, jotka aiheuttavat edes hetkellistä tajuttomuutta tai toimintakyvyn menetystä, tai vaativat jatkuvaa lääkitystä, estävät hyppäämisen. Kurssilla täytetään laskuvarjohyppääjän terveydentilavakuutus. 60 vuotta täyttäneeltä henkilöltä vaaditaan myös lääkärintodistus. Koulutusorganisaatio voi tarvittaessa vaatia jäseneltään lääkärintodistuksen osoituksena hyppäämiseen riittävästä terveydentilasta. 

