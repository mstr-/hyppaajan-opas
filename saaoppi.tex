
SIL:n ohjeiden mukaan hyppypaikalla on oltava tuulen suuntaa ja voimakkuutta osoittavat välineet. Lisäksi hyppypaikan puiden, pensaiden, järvien, pilvien, tuulipussin, heitettyjen streamerien ja hyppykoneen sortaman tarkkailu antavat tietoa vallitsevasta säästä. Kokeneet hyppääjät kertovat mielellään, mistä he tietävät tuulen olevan liian voimakas, pilvien olevan liian matalalla tai jos jonnekin paikkaan ei kannata laskeutua pyörteiden takia. Sään tutkiminen ja oppiminen jatkuu koko koulutusajan. Tavoitteena on paikallisten olosuhteiden tunteminen niin, että hyppääjällä on edellytykset hyppypäätöksen tekoon. 

\section{Tuuli}
\label{saaoppi-tuuli}


Tuulet rajoittavat hyppäämistä, sillä tuulen voimistuessa voimistuvat myös pyörteet. Tuulen voimistuessa myös uloshyppypaikan määrittäminen käy vaativammaksi ja ohjaaminen sekä laskeutuminen vaikeutuvat. Laskuvarjohyppääjien tuulirajat määrittävät toiminnalliset rajat, mutta joskus pyörteet voivat olla vaarallisen voimakkaita, vaikka tuulen voimakkuus ei olisikaan yli sallitun. Tuulen suunta eli se suunta mistä päin tuulee ja voimakkuus on tunnettava sekä pinta- että ylätuulten osalta. Tuulen nopeus metreinä sekunnissa saadaan solmuista kertomalla 0,51:llä. Laskuissa kertoimena on helpointa käyttää arvoa 0,5. On syytä muistaa, että jos tuulen nopeus on 16 solmua, se on yli 8 m/s (8,23 m/s). (Samoin 22 kt on 11,32 m/s.) Pintatuuli on tuuli, joka vaikuttaa maanpinnassa (maatuuli). Ylätuulet ovat tuulia, jotka vaikuttavat hyppykorkeuksissa. Pinta- ja ylätuulten suunnan ero on pohjoisella pallonpuoliskolla tyypillisesti oikealle noin 30° maan pyörimisliikkeen takia. Tuulen suunta voi vaihdella kerroksittain. Käsitteellä tuulen yläpuolella tarkoitetaan aluetta laskeutumisalueelta uloshyppypaikalle päin. Vastaavasti käsite tuulen alapuolella tarkoittaa aluetta maalipisteestä poispäin uloshyppypaikkaan nähden. Tuulen nopeus kasvaa ylöspäin mentäessä, maan kitkan pienetessä. Vaikka maassa olisi tyyntä, heti maanpinnan yläpuolella voi tuulla rajusti. Säärintamat (ukkonen) voimistavat tuulia ja lämpötilaerot alueittain (maastoerot) aiheuttavat äkillisiä tuulen suunnan ja voimakkuuden muutoksia. Meren läheisyydessä tuuli voi muuttua kesällä aamupäivällä ja illansuussa täysin päinvastaiseksi (merituuli-ilmiö). Tyynellä hypättäessä loppujarrutuksen oikea ajoittaminen voi olla vaikeaa. 

\subsection{Tuulirajat}
\label{saaoppi-tuulirajat}


Maatuulen nopeuden asettamat rajoitukset laskuvarjohypyille ovat seuraavat: 

\begin{itemize}
\item Itsenäisellä hyppääjällä maatuulen nopeus ei saa ylittää: 
	\begin{itemize}
	\item  A-lisenssihyppääjällä 8 m/s 
	\item  B-lisenssihyppääjällä 11 m/s 
	\item  C-lisenssihyppääjällä 11 m/s 
	\item  D-lisenssihyppääjällä 11 m/s 
	\end{itemize}
\item  Oppilas ei saa hypätä, mikäli maatuulen nopeus ylittää 8 m/s. Tandemoppilas ei saa hypätä, mikäli maatuulen nopeus ylittää 11 m/s. 
\item  Pallovarjoa pää- tai varavarjona käytettäessä maatuulen nopeus ei saa ylittää 8 m/s. 
\end{itemize}
\section{Pilvet}
\label{saaoppi-pilvet}


Laskuvarjohyppäämistä ei saa suorittaa pilvessä tai pilven läpi. Oppilastoiminnan raja on puolipilvinen sää hyppykorkeuden alapuolella. Uloshyppyhetkellä on nähtävä joko maalialue tai uloshyppypaikka. Ilmailumääräyksen OPS M6-1 mukaan tästä voidaan tietyin erityisehdoin poiketa. Tämä asettaa kuitenkin lisävaatimuksia hyppääjille, lentäjälle ja koneen varustukselle. Pilvet jaetaan alapilviin (0–2500 m), keskipilviin (2500–5000 m) ja yläpilviin (yli 5000 m). Pilvet voivat olla esimerkiksi kumpu\mbox{-,} kuuro\mbox{-,} ukkos- tai sumupilviä. Pilvien tyyppejä ei välttämättä tarvitse osata, mutta alasimen mallinen ukkospilvi (Cumulonimbus, Cb) on tunnettava ja hyppytoiminta on keskeytettävä, jos sellainen syntyy kentän läheisyyteen. Pilvet muodostuvat kosteudesta ja ilman epäpuhtauksista, joten vapaapudotuksessa tai varjon varassa pilveen ajautuva hyppääjä kastuu tai saa epämiellyttävän sade- tai raekuuron kasvoilleen. Kostealla säällä, sateen jälkeen ja syksyllä on huomioitava, että kostea ja lämmin ilma kantaa huonommin kuin kuiva ja kylmä ilma. Kosteus vaikuttaa varjon ominaisuuksiin heikentäväksi. 


Jos joudut avaamaan varjon pilvessä: 

\begin{itemize}
\item  Lennä puolijarruilla 
\item  Tee loivaa oikeaa kaarrosta 
\item  Tarkkaile koko ajan etu- ja sivusektoreita 
\item  Huuda ja kuuntele, jos epäilet, ettet ole ainoa pilvessä varjon avannut. 
\end{itemize}
\section{Termiikki}
\label{saaoppi-termiikki}


Termiikki eli nouseva ilmavirtaus muodostuu auringon lämmittämän paikan tai alueen päälle. Tuuli kaataa termiikin, joten se voi olla joko paikan päällä tai alatuulen puolella. Kylmä paikka, esimerkiksi järvi, aiheuttaa puolestaan laskevan ilmavirtauksen. Pilvessä ilma voi virrata sekä ylös- että alaspäin. Nostavat ja laskevat ilmavirtaukset ovat normaalisti 1–8 m/s. Ukkospilvessä virtaukset ovat jopa 30 m/s ylös ja alas. Nyrkkisääntönä voidaan sanoa, että siellä missä on nostava, on myös laskeva. Tämä on huomioitava kuumana kesäpäivänä suunniteltaessa laskeutumiskuvioita. Hyppykentällä virtauksia aiheuttavat mm. maalialueen hiekka, kiitotiet ja kynnetyt pellot. Virtaukset tuntuvat varjossa epämiellyttävänä tärinänä, heilumisena ja kääntymisenä. Lisäksi varjo voi tukahtua, nousta, vajota tai jopa sakata. 

\section{Turbulenssi}
\label{saaoppi-turbulenssi}


Turbulenssi muodostuu tuulen ja esteen yhteisvaikutuksesta ja sitä ilmenee myös tuulikerrostumien rajapinnoissa. Se voi olla voimakas, terävä, nostava tai laskeva, riippuen esteen muodosta ja koosta sekä tuulen voimakkuudesta. Rinnetuuli on aina turbulenttista. Turbulenssi aiheuttaa varjolle samat oireet kuin nostava ja laskeva ilmavirtaus, joten paikallisten turbulenssien tunteminen on tärkeää laskeutumiskuvioita suunniteltaessa. Voimakas turbulenssi voi tyhjentää kuvun osittain tai jopa kokonaan. Tyhjentynyt kupu täyttyy uudestaan, kun kupu saavuttaa oikean kohtauskulman ilmavirtaan nähden. Turbulenssin alla on usein tyyni kohta, joten loppuvedon tekeminen oikea-aikaisesti voi olla vaikeaa. Turbulenttisella säällä lentäminen ja hyppääminen on aina riskialtista ja sitä pitää välttää, vaikkei tuulen voimakkuus olisikaan yli tuulirajojen! Varjoja ei ole suunniteltu toimimaan turbulenttisessa kelissä! Jos kuitenkin syystä tai toisesta joudut varjollasi turbulenttiseen keliin, lennä täydessä liidossa. Liian suurilla jarruilla lentäminen saattaa aiheuttaa kuvun sakkaamisen pyörteen vaikutuksesta. Pyörteitä voi välttää laskeutumalla tarpeeksi kauas reunaesteistä ja esimerkiksi hiekan reunasta, jossa virtausten suunnat muuttuvat. 

\section{Lämpötila}
\label{saaoppi-lampotila}


Ilman lämpötila laskee ylöspäin mennessä n. 6,5 °C / 1000 m. Arvo on epätarkka ja siihen vaikuttavat inversiot, säärintamat sekä ilmanpaineen paikalliset vaihtelut. Laskuvarjohyppääjän kannattaa käyttää hanskoja, kun lämpötila hypyn aikana laskee alle nollan. Oppilailla käsineet ovat pakolliset. Vaikka maassa olisikin lämmintä, hyppykorkeudella voi olla jo pakkasta. Jos maassa on +10 °C lämmintä, voi kahden kilometrin hyppykorkeudella olla jo pakkasta. Jos maassa on +10 °C, lämpötila on kahden kilometrin korkeudessa tyypillisesti noin –3 °C. On huomioitava, että kolmen kilometrin korkeudessa on usein pakkasta. 

\section{Käytännössä}
\label{saaoppi-kaytannossa}


Parhaat hyppykelit Suomessa ovat alku- ja loppukesällä sekä aamulla ja illalla. Tällöin säätekijöiden aiheuttamat muutokset ilmatilassa ovat pienimmillään. Tuuli on heikko, aurinko ei aiheuta nostavia eikä laskevia ja ilmanpaine sekä kosteus ovat ihanteelliset. Säätiedot on aina varmistettava ennen hyppypäätöksen tekoa. Lentosääaseman tiedot, teksti-TV:n tiedot tai mittareilla mitatut arvot ovat aina ehdottomia epävarmoissa tilanteissa. Ne eivät sulje toisiaan pois vaan varmentavat niitä. On muistettava, että taivaalle pääsy ei takaa onnistunutta hyppyä, vaan turvallinen alastulo. 


Lentosääsanomia saadaan mm. seuraavasti: 

\begin{itemize}
\item  Internet (mm. \url{http://www.ilmailusaa.fi/} ) 
\item  Lentosääaseman automaattitiedotus radiolla ja puhelimella 
\item  Sääsanomat YLE Teksti-TV s. 428 ja 429 (myös \url{http://www.yle.fi/tekstitv/html/P428_01.html} ) 
\end{itemize}
\section{ METAR-sanomat }
\label{saaoppi-metar-sanomat}


METAR-sanoma on ilmaliikenteen käyttöön tarkoitettu sääsanoma, joka laaditaan lentopaikan sääasemalla. Metar kertoo lentoasemalla vallitsevan säätilan. Metareita laaditaan pääsääntöisesti kaikilla lentopaikoilla, joille on säännöllistä liikennettä. Suomessa Metar luodaan näiltä lentoasemilta 30 minuutin välein, 20 minuuttia ja 50 minuuttia yli jokaisen tunnin. 


EFTU 231250Z 27006KT 9999 SCT055 FEW080 08/04 Q1015 


Tämä viesti voidaan jakaa osioihin ja tulkita sen perusteella. Metar-viestin pääosiot ovat: 

\begin{itemize}
\item  EFTU Sääasema lentopaikan ICAO-koodilla; Eurooppa, Finland, Turku 
\item  231250Z Aika (UTC) Päivä kuluvaa kuukautta sekä kellonaika UTC-ajassa ilmaistuna  
\item  27006KT Tuulen suunta asteina ja nopeus solmuina, 270 astetta, 3 m/s  
\item  9999 Näkyvyys metreinä, yli 10 km 
\item  SCT055 Pilvien määrä sekä korkeus 
\item  FEW080 Pilvien äärä sekä korkeus  
\item  08/04 Lämpötila ja kastepiste celsiusasteina  
\item  Q1015 Ilmanpaine millibareina / hehtopascaleina  
\item  Lisätiedot ja mahdolliset muut huomioitavat asiat kerrotaan Metarin lopussa  
\end{itemize}

Metar-sanomassa tuulen suunta ilmoitetaan kolmella numerolla, lähimpään 10 asteeseen pyöristettynä. Tuulen nopeuden yksikkönä käytetään solmua (KT). Tuulen ollessa tyyni, ilmoitetaan se metar-sanomassa 00000KT. 

\begin{itemize}
\item  VRB02KT: Suunnaltaan vaihtelevaa tuulta, tuulen nopeus 1 m/s. 
\item  18010KT: Tuulen suunta on etelästä (180 astetta) ja tuulen nopeus on 5 m/s. 
\item  22015G28KT: Tuulen suunta on 220 astetta, keskituulen nopeus 7,5 m/s ja puuskissa 14 m/s. 
\item  35014KT 310V030: Tuulen suunta vaihtelee lentopaikalla siten, ettei keskituulen suuntaa voi määrätä ja keskituulen nopeus on 7 m/s. Tuulen suunnan vaihteluväli on 310 astetta ja 030 astetta. 
\end{itemize}

Metar-sanomassa ilmoitetaan sääilmiöt kuten sumu, sade, lumisade, ukkonen ja muut vastaavat säähän vaikuttavat tekijät.  


Esimerkkejä vallitsevan sään ilmoittamisesta: 

\begin{itemize}
\item  TS: ukkosta ilman sadetta 
\item  TSRA: ukkosta ja vesisadetta 
\item  -RABR: heikkoa vesisadetta ja utua 
\item  FG: sumua 
\item  MIFG: matalaa sumua 
\item  -DZBR: heikkoa tihkusadetta ja utua 
\end{itemize}

Metar-sanomassa pilven alaraja ilmoitetaan lyhennetysti jättämällä korkeuden kaksi viimeistä nollaa pois. Esimerkiksi pilven alaraja 300 ft kirjoitetaan METARissa 003, 3000 ft kirjoitetaan 030 ja 30000 ft kirjoitetaan 300. Korkeus ilmoitetaan maanpinnasta. 


Näkyvä taivas jaetaan pilvisyyttä ilmoitettaessa kahdeksaan osaan. Pilvisyys ilmoitetaan kolmen kirjaimen tunnuksella, joka kertoo kuinka monta kahdeksasosaa taivaasta on pilven peitossa kyseisessä korkeudessa: 

\begin{itemize}
\item  0/8 SKC = sky clear (pilvetöntä) 
\item  1/8-2/8 FEW = few (muutamia pilviä)  
\item  3/8-4/8 SCT = scattered (hajanaisia pilviä)  
\item  5/8-7/8 BKN = broken (melkein pilvistä)  
\item  8/8 OVC = overcast (täysin pilvistä) 
\end{itemize}

Pilvestä voidaan ilmoittaa myös lisämääreet CB ja TCU: 

\begin{itemize}
\item  CB - Cumulonimbus, ukkospilvi 
\item  TCU - Towering Cumulus, nouseva cumuluspilvi 
\end{itemize}

Esimerkkejä pilvien ilmoittamisesta: 

\begin{itemize}
\item  FEW007 - vähän pilviä 700 ft korkeudella maanpinnasta. 
\item  SCT020 - 3/8 taivaasta pilvessä 2000 ft korkeudella 
\item  BKN080 - 5/8 taivaasta pilvessä 8000 ft korkeudella. 
\item  OVC003 - täyskatto (8/8) pilviä 300 ft korkeudella. 
\end{itemize}

Koodisana CAVOK on lyhenne englannin kielen sanoista ''Ceiling And Visibility OK''. Sitä käytetään, mikäli kaikki seuraavat ehdot ovat voimassa samanaikaisesti:  

\begin{itemize}
\item  Näkyvyys on 10 km tai enemmän. 
\item  Alueella ei ole pilviä 1500 m (5000 ft) alapuolella eikä CB- tai TCU-pilviä havaita.  
\item  Ei esiinny merkittäviä sääilmiöitä kuten ukkosta, sadetta tai sumua. 
\end{itemize}
\section{ GAFOR-sanomat }
\label{saaoppi-gafor-sanomat}


Suomi on GAFOR-alue-ennusteiden osalta jaettu kolmeen osaan. GAFOR-sanomassa käytetään samoja lyhenteitä kuin METAR-sanomassa. Hyppääjien kannalta kiinnostavinta GAFOR-sanomassa on tuuliennusteet sekä 2000 ft että 5000 ft korkeudelle. Näitä tuulitietoja voidaan käyttää uloshyppypaikan määrityksessä. 


FBFI42 EFRO 121000 


GA-FCST FOR AREAS 21/25 VALID 0312 UTC 


WX AURINKOISTA JA PILVETÖNTÄ. 


PÄIVÄN MITTAAN ODOTETTAVISSA 


HELLELÄMPÖTILOJA KOKO ALUEELLA. 


WINDS 


SFC 21/23 270-310 / 05-10 KT 24/25 010-030 / 03-06 KT 


2000 FT 300-320 / 15-20 KT 


5000 FT 330 / 25 KT 


0-C LEVEL FL120 


ICE NIL TURB NIL 


GAFOR EFRO 1218 BBBB 21/25 O 


Esimerkkipauksessa kyse on Itä-Suomen alue-ennusteesta. 


Sanoma on voimassa alueilla 21-25 (Utin lentokenttä sijaitsee alueella 22), ja aikavälillä 3.00-12.00 UTC-aikaa. GAFORit laaditaan joka vuorokausi samoille yhdeksän tunnin aikaväleille, joten 0312-sanoma on siis aamu/päiväennuste, ja 1221-sanoma on iltapäivä/iltaennuste. 


WX (Weather eXplanation, sään kuvaus)- tunnuksesta eteenpäin kuvataan parilla-kolmella rivillä selkeästi sanoman voimassaoloaikana vallitseva tai lähitunteina odotettavissa oleva sää. Joskus WX-osassa kerrotaan myös seuraavan päivän ennuste. 


Tuulitietojen alkaminen ilmaistaan sanomassa tunnuksella WINDS (tuulet). Mikäli kerrostuulien yhteydessä ei ole erillistä alueisiin jakoa, lukee WINDS-tunnuksen perässä monesti 21/25, jolloin kaikki tiedot koskevat koko Itä-Suomen aluetta. Tuulitiedot on aina jaettu kolmeen osaan, eli on ilmoitettu tuulet: 

\begin{itemize}
\item  Maanpinnan läheisyydessä (SFC, surface)  
\item  2000 jalan (600 metrin) korkeudella   
\item  5000 jalan (1500 metrin) korkeudella.  
\end{itemize}

Eri kerroksien tuulitiedot on toisinaan jaettu alueisiin, kuten tässä maatuuliosa. Maatuuliennusteen (SFC) ensimmäinen osa koskee tässä tapauksessa alueita 21/23 eli alueita 21, 22 ja 23. Aluemäärityksen jälkeen ilmoitetaan tuulen suunnan vaihteluväli asteina. 270-310 tarkoittaa siis, että tuulen suunta vaihtelee välillä 270 ja 310 astetta. Kautta-viivan jälkeen annetaan tuulen nopeus tai nopeuden vaihteluväli solmuina (KT, knots). 05-10 KT tarkoittaa siis, että tuulen keskimääräinen voimakkuus on 5 - 10 solmua (2,5 - 5 metriä sekunnissa). Nämä lukemat eivät kerro maatuulen huippuarvoista, jotka saadaan selville kerhon tuulimittarilla tai sääasemalta. 


Maatuulikoodin toinen osa 24/25 010-030 / 03-06 KT kertoo vastaavat tiedot alueille 24 ja 25. 


2000 FT:n (600 metriä) tuulitiedoissa ei ole tällä kertaa erillistä alueisiin jakoa, jolloin ne koskevat koko Itä-Suomen aluetta. 2000 FT 300-320 / 15-20 KT tarkoittaa siis, että 600 metrin korkeudessa tuulen suunta vaihtelee välillä 300-320 astetta, ja voimakkuus välillä 15-20 solmua (7,5 - 10 metriä sekunnissa). 


5000 FT 330 / 25 KT tarkoittaa, että 1500 metrin korkeudessa tuulee suunnasta 330 astetta 25 solmun voimakkuudella (12,5 metriä sekunnissa). 


SFC\mbox{-,} 2000 ft- ja 5000 ft-tuulitiedoilla voi yleensä riittävällä tarkkuudella määritellä jopa 4000 m:n UH-paikan. Erityisesti 5000 ft:n tuulitietoja voidaan käyttää ajautuman laskemiseen korkeusvälillä noin 1000-4000 m, sillä noilla korkeuksilla ei juurikaan ole tuuliolosuhteisiin vaikuttavia häiriötekijöitä. 


Satunnaisesti GAFORin tuulitiedoissa mukana oleva BECMG-lyhenne (Becoming, tulossa) kellonaika- ja tuulitietoineen tarkoittaa, että sanoman voimassaoloaikana on odotettavissa merkittäviä tuulen suunnan tai nopeuden muutoksia. Tämä saattaa merkitä esimerkiksi säärintaman lähestymistä, ja tällöin GAFORista luettaviin tuulitietoihin tulee suhtautua kriittisesti. 


0-C LEVEL-tunnuksella alkava rivi kertoo, missä korkeudessa lämpötila vaihtuu pakkasen puolelle ylöspäin mentäessä. Korkeus on ilmaistu kolmella numerolla lentopintana (FL, Flight Level), joka on lentoliikenteessä käytetty ilmanpaineesta riippuva suhteellinen korkeus. Korkeus voidaan kuitenkin tässä tapauksessa karkeasti muuttaa metreiksi ottamatta huomioon ilmanpaineen vaikutusta. Tällä kertaa lämpötilan nollaraja on siis suunnilleen korkeudella 120 sataa eli 12000 jalkaa (3660 metriä). Yksi jalka on 0,3048 metriä. Talvella nollarajatiedolla ei yleensä ole merkitystä, sillä lämpötila on monesti pakkasen puolella heti maanpinnasta lähtien. Tällöin 0-C LEVELin jälkeen saattaa lukea esim. NIL (No information, ei tietoa), NEAR SFC (lähellä maanpintaa) tai GND (Ground, maanpinnan tasalla). 


GAFOR-sanoman parilla viimeisellä rivillä kerrotaan esimerkiksi jäätämisestä ja yläkerrosten turbulenssista, sekä esitetään lyhyt koodi lentosääluokista. 

\section{Harjoitus}
\label{saaoppi-harjoitus}

\begin{enumerate}[label=\bfseries \arabic*)]
\item  Tutustutaan internetin ilmailusääsivuihin: 
	\begin{itemize}
	\item  METAR = vallitseva sää lentopaikalla 
	\item  TAF = sääennuste lentopaikalla 
	\item  GAFOR = alue-ennuste. 
	\end{itemize}
\item  Etsitään maastokartalta kenttäalueen ympäriltä pyörteitä ja nostavia ilmavirtauksia aiheuttavia kohteita. 
\item  Tutustutaan kentän nyrkkisääntöihin tuulen ja pilvien lukemiseksi ilman mittareita.  
\item  Selvitetään laskuvarjohyppääjien tuulirajat. 
\end{enumerate}
