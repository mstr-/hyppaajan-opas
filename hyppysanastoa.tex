\begin{description}
\item[Blokki, block (FS) ] \hfill \\ 
FS-hypyssä suorituksen osa, joka koostuu kahdesta kuvasta ja välisuorituksesta. \hfill \\ 
\end{description}
\begin{description}
\item[Boksi, box ] \hfill \\ 
Vapaapudotuksen perusasento. \hfill \\ 
\end{description}
\begin{description}
\item[Bootit, booties ] \hfill \\ 
FS-haalareissa jalkojen päälle vedettävät osat, joilla saadaan lisää pinta-alaa liikkeiden tehostamiseksi. \hfill \\ 
\end{description}
\begin{description}
\item[Dive pool ] \hfill \\ 
International Parachuting Commissionin (IPC) julkaisemat kilpailumuodostelmat, sisältää randomeja ja blokkeja. \hfill \\ 
\end{description}
\begin{description}
\item[Flutteri, floater ] \hfill \\ 
4-way:ssä etu- ja takaflutteri. Suurissa muodostelmissa lähtevät koneesta ennen muodostelman pohjaa. \hfill \\ 
\end{description}
\begin{description}
\item[FS] \hfill \\ 
Formation Skydiving, muodostelmahyppy, myös Relative Work. Muodostelman koko ilmoitetaan esim. 2-way, 4-way. Kilpailulajit ovat 4\mbox{-,} 8- ja 16-way. \hfill \\ 
\end{description}
\begin{description}
\item[Grippi, grip ] \hfill \\ 
FS-haalareiden hihoissa ja lahkeissa olevat pötkylät, jotka mahdollistavat paremmat otteet. \hfill \\ 
\end{description}
\begin{description}
\item[Integraali ] \hfill \\ 
Koko pään peittävä kypärä, jossa on leukasuojus. Visiirillä tai ilman. \hfill \\ 
\end{description}
\begin{description}
\item[Intermediate ] \hfill \\ 
Eurooppalainen kilpailusarja, joka on tarkoitettu aloitteleville joukkueille. Suppeampi Dive pool. \hfill \\ 
\end{description}
\begin{description}
\item[Jälkikuivat, debrief   ] \hfill \\ 
Hypyn jälkeen tapahtuva suorituksen läpikäynti, joko keskustellen ja/tai videon kanssa. \hfill \\ 
\end{description}
\begin{description}
\item[Kuivat ] \hfill \\ 
Maassa tapahtuva harjoittelu. \hfill \\ 
\end{description}
\begin{description}
\item[Laudat ] \hfill \\ 
Rullalaudat, joilla harjoitellaan siirtymiä ja muodostelmia maassa. \hfill \\ 
\end{description}
\begin{description}
\item[Liuku ] \hfill \\ 
Purun jälkeen tapahtuva liikkuminen, jolla varmistetaan riittävä ja turvallinen ilmatila avausta varten. \hfill \\ 
\end{description}
\begin{description}
\item[Mantis ] \hfill \\ 
Kehittyneempi vapaapudotusasento. \hfill \\ 
\end{description}
\begin{description}
\item[Muodostelma ] \hfill \\ 
Hyppääjiä toisiinsa ottein liittyneinä. \hfill \\ 
\end{description}
\begin{description}
\item[Open ] \hfill \\ 
Kilpailusarja, kilpailuhypyt arvotaan koko Dive poolista. \hfill \\ 
\end{description}
\begin{description}
\item[Ote ] \hfill \\ 
Joko käsivarresta tai jalasta. Minimi käsiote on liikkumaton kosketus kädestä tai jalasta. \hfill \\ 
\end{description}
\begin{description}
\item[Painot ] \hfill \\ 
Painovyö tai painoliivit. \hfill \\ 
\end{description}
\begin{description}
\item[Pulja ] \hfill \\ 
Muodostelman osa, jolla suoritetaan blokkien siirtymä. \hfill \\ 
\end{description}
\begin{description}
\item[Purku ] \hfill \\ 
Otteiden irrottaminen siirryttäessä muodostelmasta toiseen. \hfill \\ 
\end{description}
\begin{description}
\item[Pystykuivat ] \hfill \\ 
Hypyn läpikäyminen kävellen. \hfill \\ 
\end{description}
\begin{description}
\item[Random ] \hfill \\ 
Ennalta määritelty yksittäinen muodostelma. \hfill \\ 
\end{description}
\begin{description}
\item[Sekvenssi ] \hfill \\ 
Muodostelmien, välisuoritusten ja vapaiden siirtymien sarja. \hfill \\ 
\end{description}
\begin{description}
\item[Uloshyppy, exit ] \hfill \\ 
Muodostelman yhtäaikainen irtautuminen koneesta. \hfill \\ 
\end{description}
\begin{description}
\item[Äänikorkeusmittari ] \hfill \\ 
Elektroninen korkeusmittari, joka antaa äänimerkin ennalta asetetuissa korkeuksissa, esim. purku ja avaus. \hfill \\ 
\end{description}
