\section{ 15'' }
\label{pl-peruskoulutuksen-suoritukset-15}


Avaustoimenpiteet aloitetaan 1200 metrin korkeudessa mittarin mukaan. Vapaapudotusaika on noin 15 sekuntia. Lisää tästä lähin avaustoimenpiteisiisi avausmerkki, jolla ilmoitat muille avaavasi varjosi, eli heilauta käsiäsi ristiin pääsi edessä.  

\subsection{ Oppimistavoitteet }
\label{pl-peruskoulutuksen-suoritukset-oppimistavoitteet}

\begin{enumerate}[label=\bfseries \arabic*)]
\item  Stabiili asento  
\item  HD-apuvarjo heitetty oikein ilmavirtaan (jos hypyllä tehdään HD-totuttelu) 
\end{enumerate}
\subsection{ Hyppylennolla }
\label{pl-peruskoulutuksen-suoritukset-hyppylennolla}

\begin{itemize}
\item Kertaa suoritus mielessäsi 
\item Keskity suoritukseesi 
\item 3X3 -tarkastus ennen hyppyä (\ref{laskuvarjokalusto-ja-hyppyvarusteet-3x3-tarkastus} s.\pageref{laskuvarjokalusto-ja-hyppyvarusteet-3x3-tarkastus}) 
\end{itemize}
\subsection{ Hypyn kulku }
\label{pl-peruskoulutuksen-suoritukset-hypyn-kulku}

\begin{description}
\item[TAIVUTA] \hfill \\ 
Tee uloshyppy koneesta ottaen välittömästi ilmavirtaan päästyäsi hyvä taivutus. \hfill \\ 
Rentouta asento uloshypyn jälkeen \hfill \\ 
Tarkkaile korkeutta \hfill \\ 
\item[1200 metriä] \hfill \\ 
Aloita avaustoimenpiteet: Näytä avausmerkki. \hfill \\ 
\item[TAIVUTA ] \hfill \\ 
Varmistetaan taivutus ja asento. \hfill \\ 
\item[TARTU] \hfill \\ 
Vasen käsi kypärän päälle eteen vartalon normaaliksi jatkeeksi ja oikea käsi kahvalle, tiukka ote \hfill \\ 
\item[VEDÄ] \hfill \\ 
	\begin{itemize}
	\item  Veto kahvasta putken tai taskun suuntaan 
	\item  Palauta perusasento. 
	\end{itemize}
\item[101 ] \hfill \\ 
Aloitetaan laskeminen alusta vedon jälkeen. \hfill \\ 
\item[102..104 ] \hfill \\ 
Varjo avautuu ja tehdään sen lopullinen avaaminen. \hfill \\ 
\item[105 ] \hfill \\ 
Vilkaistaan olkapään yli tarvittaessa (turbulenssin poisto). \hfill \\ 
\end{description}

\end{multicols}\pagebreak\begin{multicols}{2} 

\section{ Suora uloshyppy }
\label{pl-peruskoulutuksen-suoritukset-suora-uloshyppy}


Tavoitteena on oppia uloshyppy suorana, eli suoraan koneen sisältä ilmavirtaan. 

\subsubsection{Streeva-kone}
\label{pl-peruskoulutuksen-suoritukset-streeva-kone}

\begin{enumerate}[label=\bfseries \arabic*)]
\item Vasen jalka astinlaudalle. 
\item Kädet oviaukon reunalle. 
\item Kohottaudutaan koneen ulkopuolelle varoen repun osumista koneeseen. 
\item Käännetään rinta koneen etenemissuuntaan. 
\item Ponnistetaan ilmavirran suuntaan taaksepäin. 
\item Taivutus, katse ylös, jalat lähes suorina ja kädet taakse delta-asentoon. 
\item Sitä mukaa kun asento kääntyy vaakatasoon, otetaan perusasento. 
\end{enumerate}
\subsubsection{Muut koneet}
\label{pl-peruskoulutuksen-suoritukset-muut-koneet}

\begin{enumerate}[label=\bfseries \arabic*)]
\item Siirrytään oviaukolle kyykkyyn tai seisaalleen, varoen repun osumista koneeseen. 
\item Ponnistetaan ilmavirran suuntaan eteenpäin, rinta koneen etenemissuuntaan. 
\item Taivutus, jalat lähes suorina ja kädet taakse delta-asentoon. 
\item Sitä mukaa kun asento kääntyy vaakatasoon, otetaan perusasento. 
\end{enumerate}
\subsection{ Oppimistavoitteet }
\label{pl-peruskoulutuksen-suoritukset-oppimistavoitteet}

\begin{enumerate}[label=\bfseries \arabic*)]
\item  Stabiili asento 5 sekunnin kuluessa 
\item  Suunnan hallinta, ei tahatonta yli 360° käännöstä. 
\end{enumerate}
\subsection{ Hyppylennolla }
\label{pl-peruskoulutuksen-suoritukset-hyppylennolla}

\begin{itemize}
\item Kertaa suoritus mielessäsi 
\item Keskity suoritukseesi 
\item 3X3 -tarkastus ennen hyppyä (\ref{laskuvarjokalusto-ja-hyppyvarusteet-3x3-tarkastus} s.\pageref{laskuvarjokalusto-ja-hyppyvarusteet-3x3-tarkastus}) 
\end{itemize}
\subsection{ Hypyn kulku }
\label{pl-peruskoulutuksen-suoritukset-hypyn-kulku}

\begin{description}
\item[TAIVUTA] \hfill \\ 
Tee uloshyppy koneesta ottaen välittömästi ilmavirtaan päästyäsi hyvä delta-asento. \hfill \\ 
Asennon käännyttyä vaakatasoon (tyypillisesti noin 5 sekunnin jälkeen) ota perusasento \hfill \\ 
Tarkkaile korkeutta \hfill \\ 
\item[1200 metriä] \hfill \\ 
Aloita avaustoimenpiteet. \hfill \\ 
\end{description}

\end{multicols}\pagebreak\begin{multicols}{2} 

\section{ Sukellusuloshyppy }
\label{pl-peruskoulutuksen-suoritukset-sukellusuloshyppy}


Koneesta sukelletaan ulos ja pidetään suunta. 

\subsubsection{Streeva-kone}
\label{pl-peruskoulutuksen-suoritukset-streeva-kone}

\begin{enumerate}[label=\bfseries \arabic*)]
\item Oikea jalka astimelle, varpaat peräsimen suuntaan. 
\item Kädet oviaukon reunalle. 
\item Kohottaudutaan koneen ulkopuolelle varoen repun osumista koneeseen. 
\item Käännytään koneen peräsintä kohti. 
\item Sukelletaan ilmavirran suuntaan taaksepäin, rinta ilmavirtaan. 
\item Taivutus, jalat koukkuun ja kädet eteen leveälle. 
\item Sitä mukaa kun asento kääntyy vaakatasoon, otetaan perusasento. 
\end{enumerate}
\subsubsection{Muut koneet}
\label{pl-peruskoulutuksen-suoritukset-muut-koneet}

\begin{enumerate}[label=\bfseries \arabic*)]
\item Siirrytään oviaukolle kyykkyyn tai seisaalleen, varoen repun osumista koneeseen. 
\item Sukelletaan ilmavirran suuntaan taaksepäin, rinta ilmavirtaan. 
\item Taivutus, jalat koukkuun ja kädet eteen leveälle. 
\item Sitä mukaa kun asento kääntyy vaakatasoon, otetaan perusasento. 
\end{enumerate}
\subsection{ Oppimistavoitteet }
\label{pl-peruskoulutuksen-suoritukset-oppimistavoitteet}

\begin{enumerate}[label=\bfseries \arabic*)]
\item  Stabiili asento 5 sekunnin kuluessa 
\item  Suunnan hallinta, ei tahatonta yli 360° käännöstä. 
\end{enumerate}
\subsection{ Hyppylennolla }
\label{pl-peruskoulutuksen-suoritukset-hyppylennolla}

\begin{itemize}
\item Kertaa suoritus mielessäsi 
\item Keskity suoritukseesi 
\item 3X3 -tarkastus ennen hyppyä (\ref{laskuvarjokalusto-ja-hyppyvarusteet-3x3-tarkastus} s.\pageref{laskuvarjokalusto-ja-hyppyvarusteet-3x3-tarkastus}) 
\end{itemize}
\subsection{ Hypyn kulku }
\label{pl-peruskoulutuksen-suoritukset-hypyn-kulku}

\begin{description}
\item[TAIVUTA] \hfill \\ 
Tee sukellus-uloshyppy \hfill \\ 
Pidä asentosi symmetrisenä, jalat koukussa \hfill \\ 
Asennon käännyttyä vaakatasoon (tyypillisesti noin 5 sekunnin jälkeen) ota perusasento \hfill \\ 
Tarkkaile korkeutta \hfill \\ 
\item[1200 metriä] \hfill \\ 
Aloita avaustoimenpiteet. \hfill \\ 
\end{description}

\end{multicols}\pagebreak\begin{multicols}{2} 

\section{ 360° käännökset }
\label{pl-peruskoulutuksen-suoritukset-360deg-kaannokset}


Uloshypyn jälkeen tehdään hallittu 360 asteen käännös. 

\begin{enumerate}[label=\bfseries \arabic*)]
\item Otetaan kiintopiste horisontista. 
\item Painetaan hartialinja alas halutun käännöksen suuntaan. 
\item Otetaan perusasento ennen kiintopistettä. 
\item Vähän ennen kiintopistettä tehdään pieni vastaliike, jotta käännös pysähtyy täsmällisesti. 
\end{enumerate}

Pidä taivutus koko käännöksen ajan. Tarkkaile korkeutta. 

\subsection{ Oppimistavoitteet }
\label{pl-peruskoulutuksen-suoritukset-oppimistavoitteet}

\begin{enumerate}[label=\bfseries \arabic*)]
\item  360° käännös sovittuun suuntaan ja hallittu pysäytys. 
\item  Asento säilyy stabiilina. 
\end{enumerate}
\subsection{ Hyppylennolla }
\label{pl-peruskoulutuksen-suoritukset-hyppylennolla}

\begin{itemize}
\item Kertaa suoritus mielessäsi 
\item Keskity suoritukseesi 
\item 3X3 -tarkastus ennen hyppyä 
\end{itemize}
\subsection{ Hypyn kulku }
\label{pl-peruskoulutuksen-suoritukset-hypyn-kulku}

\begin{description}
\item[TAIVUTA] \hfill \\ 
Tee uloshyppy koneesta \hfill \\ 
Asennon käännyttyä vaakatasoon (tyypillisesti noin 5 sekunnin jälkeen) ota perusasento \hfill \\ 
Ota kiintopiste maasta/horisontista ja tee 360° käännös sovittuun suuntaan \hfill \\ 
Tarkkaile korkeutta, toista harjoitus jos korkeutta on riittävästi. \hfill \\ 
\item[1600 metriä] \hfill \\ 
Lopeta työskentely ja valmistaudu päävarjon avaamiseen. \hfill \\ 
\item[1200 metriä] \hfill \\ 
Aloita avaustoimenpiteet. \hfill \\ 
\end{description}

\end{multicols}\pagebreak\begin{multicols}{2} 

\section{ Tynnyri ja takavoltti }
\label{pl-peruskoulutuksen-suoritukset-tynnyri-ja-takavoltti}

\subsubsection{ Tynnyri }
\label{pl-peruskoulutuksen-suoritukset-tynnyri}


Tynnyri on freeasentojen perusliike, sillä isommilla freekuvilla purun jälkeen liu’uttaessa voidaan tekemällä liu’usta tynnyri tarkastaa vapaa ilmatila ennen avausta. Tynnyrissä on tarkoitus kääntyä vatsaltaan kyljen kautta täysi vaakakierre. 


Tynnyri tehdään seuraavasti: 

\begin{enumerate}[label=\bfseries \arabic*)]
\item Otetaan perusasento. 
\item Viedään toinen käsi vartalon lähelle tai alle. 
\item Kierretään vartaloa siten, että lähelle tuodun käden puolta painetaan alaspäin. 
\item Asennon käännyttyä selälleen jatketaan liikettä vielä toisen kyljen ympäri. 
\item Palautetaan perusasento. 
\end{enumerate}
\subsubsection{ Takavoltti }
\label{pl-peruskoulutuksen-suoritukset-takavoltti}


Takavoltissa tehdään voltti vaaka-akselin ympäri taaksepäin. 


Takavoltti tehdään seuraavasti: 

\begin{enumerate}[label=\bfseries \arabic*)]
\item Otetaan perusasento. 
\item Vedetään jalat nopeasti yhteen koukkuun. 
\item Painetaan käsillä alaspäin ilmavirtaa vasten. 
\item Pään ollessa alaspäin palautetaan perusasento. 
\end{enumerate}

Kädet pidetään perusasennossa suorina, hieman sivuille käännettyinä. Jalat laitetaan yhteen ja koukkuun vartalon alle. Painetaan käsillä alaspäin ilmavirtaa vasten. Pysäytys tapahtuu palauttamalla perusasento. Jalkojen nopea ja yhtaikainen tuonti vartalon alle takaa voltin onnistumisen. Käsillä autetaan ympäri menoa ja estetään asennon kallistuminen sivulle. 

\subsection{ Oppimistavoitteet }
\label{pl-peruskoulutuksen-suoritukset-oppimistavoitteet}

\begin{itemize}
\item  Asento saadaan hallintaan liikkeiden jälkeen. 
\end{itemize}
\subsection{ Hyppylennolla }
\label{pl-peruskoulutuksen-suoritukset-hyppylennolla}

\begin{itemize}
\item Kertaa suoritus mielessäsi. 
\item Keskity suoritukseesi. 
\item 3X3 -tarkastus ennen hyppyä. 
\end{itemize}
\subsection{ Hypyn kulku }
\label{pl-peruskoulutuksen-suoritukset-hypyn-kulku}

\begin{description}
\item[TAIVUTA] \hfill \\ 
Tee uloshyppy koneesta \hfill \\ 
Asennon käännyttyä vaakatasoon (tyypillisesti noin 5 sekunnin jälkeen) ota perusasento \hfill \\ 
Tee tynnyri \hfill \\ 
Tarkkaile korkeutta \hfill \\ 
Tee takavoltti \hfill \\ 
Tarkkaile korkeutta \hfill \\ 
Mikäli korkeutta on vielä yli 1800 metriä, toista harjoitus. \hfill \\ 
\item[1800 metriä] \hfill \\ 
Lopeta työskentely ja valmistaudu päävarjon avaamiseen. \hfill \\ 
\item[1200 metriä] \hfill \\ 
Aloita avaustoimenpiteet. \hfill \\ 
\end{description}

\end{multicols}\pagebreak\begin{multicols}{2} 

