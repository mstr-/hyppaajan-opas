
Laskuvarjot ovat viime vuosikymmeninä kehittyneet erittäin nopeasti. Nopea kehitys johti 2000-luvun alussa siihen, että hyppääjien koulutus ja asenteet eivät pysyneet kehityksen vauhdissa. Tämä aiheutti selvän turvallisuusongelman. Sekä Suomessa että muualla maailmassa tähän asiaan reagoitiin panostamalla koulutukseen ja saatavilla olevaan tietoon. Näin on saanut alkunsa tämäkin opas. 


Vielä kolme vuosikymmentä sitten loukkaantumiset ja kuolemantapaukset johtuivat usein siitä, että laskuvarjokaluston (varavarjo mukaan lukien) toimintavarmuus ei ollut nykyistä tasoa ja hyppyvarusteet olivat huonokuntoisia. Lisäksi varjojen lento-ominaisuudet, kuten tuulen kumoaminen ja loppuvedon teho olivat huomattavasti heikommat kuin nykyään. 1960- ja 1970-luvuilla hypättiin pallovarjoilla. Kokeneet hyppääjät saattoivat käyttää aukollisia tehovarjoja. 1980-luvulle tultaessa kokeneet hyppääjät siirtyivät patjavarjoihin, mutta oppilaat hyppäsivät yhä pallovarjoilla.  


Oppilashyppytoiminnassa isoja F-111-kankaisia patjavarjoja (Manta, Raider) alettiin käyttää 1980-luvulla. Lisenssihyppääjät hyppäsivät tuolloin pienemmillä 150–220-neliöjalkaisilla F-111-kankaisilla varjoilla, kuten PD-9 cell, Fury, Maverick ja Falcon. Vuosikymmenen lopulla otettiin suuri kehitysaskel eteenpäin, kun Parachutes de France esitteli Blue Trackin 1988 ja Performance Designs Sabren 1989. Nämä ensimmäiset ilmaa läpäisemättömästä (zero-porosity, ZP) kankaasta valmistetut varjot lensivät paremmin ja säilyttivät suorituskykynsä huomattavasti kauemmin kuin edeltäjänsä, mutta vaativat lentäjältään huomattavasti suurempaa omistautumista pakkaustekniikkaan sekä turvalliseen lentämiseen. Samalla hyppääjien siipikuormat kasvoivat, eikä varjoja enää osattu lentää niiden suorituskyvyn vaatimalla tavalla. Puutteellisen koulutuksen ja virheellisten asenteiden vuoksi 1990-luvulla laskeutumisissa tapahtuneet onnettomuudet ja jopa kuolemantapaukset yleistyivät nopeasti. 


Kehitys on jatkunut samansuuntaisena valmistajien esitellessä yhä nopeampia ja suorituskykyisempiä varjoja kokeneimpien hyppääjien käyttöön (PD Stiletto 1993, Icarus Extreme 1995 ja Extreme VX 1999, Precision Aerodynamics Xaos27 2003, PD Velocity), eikä kehityksen pysähtyminen näytä todennäköiseltä. Tällä hetkellä nopeimmat varjot lentävät laskeutumisen aikana jopa yhtä suurella vaakanopeudella kuin hyppykoneet. Samalla myös esimerkiksi oppilaille ja kokemattomammille lisenssihyppääjille on kehitetty yhä suorituskykyisempiä kupuja. Varjoja myös lennetään yhä suuremmilla siipikuormilla. Valitettavan usein keskittyminen varjon lentämisen harjoitteluun ja turvallisiin asenteisiin jää liian vähäiseksi. Varjojen kasvava suorituskyky on sinänsä hyvä ja turvallisuutta lisäävä asia, mutta väärinkäytettynä se aiheuttaa ongelmia.  


2000-luvulla herättiin siihen, että maailmalla suurin osa loukkaantumisista ja lähes puolet kuolemaan johtaneista hyppyonnettomuuksista tapahtuu sen jälkeen, kun hyppääjä on avannut päävarjonsa ja todennut sen täysin toimivaksi. Onnettomuudet tapahtuvat törmätessä varjon varassa, matalissa käännöksissä, tarkoituksellisissa nopeissa laskuissa jne. Tämän vuoksi on alettu ymmärtää, että laskuvarjon lentämiseen tarvitaan koulutusta.  


Suomessa hyppykausi on lyhyt ja hyppääjien vuosittaiset hyppymäärät varsin alhaisia verrattuna suuriin laskuvarjomaihin. Suomessa kulttuuri kehittyi nopeasti siihen suuntaan että Suomessa hypättiin kokemukseen nähden varsin suurilla siipikuormilla kun taas esimerkiksi Ruotsissa ja USA:ssa hyppääjät käyttivät huomattavasti konservatiivisempia varjoja. Aktiivisen turvallisuustyön ansiosta Suomessa on tapahtunut muutos konservatiivisempaan suuntaan ja nykyään siipikuormat ovat pääosin kansainvälisellä tasolla. 


2000-luvulla Suomessa nousi ongelmaksi se, että loukkaantumisia, sattui laskeutumisissa varsin usein kokemattomille hyppääjille. Tämän vuoksi jokaisen hyppääjän tulisikin miettiä, millä varjolla sekä siipikuormalla haluaa lentää ja miksi. Tämä opas on tarkoitettu kaikille lisenssihyppääjille hyppymäärään katsomatta. Jokaisen laskuvarjolla lentävän tulisi perehtyä niihin lainalaisuuksiin, joiden mukaan laskuvarjot toimivat, sekä niihin tapoihin, joilla laskuvarjoa lennetään turvallisesti.  


Opetellessasi varjolla lentämistä tulet huomaamaan, miten hienoa ja palkitsevaa on, kun laskeutumiset alkavat sujua paremmin ja varmemmin erilaisissa olosuhteissa. Samalla alat nauttia varjon suorituskyvyn ja ominaisuuksien käyttämisestä niin paljon, että mielesi tekee lähteä taivaalle vain varjolla lentämisen vuoksi. 


Tämä opas ei ole yksiselitteinen tai ainoa totuus turvallisesta kuvunkäsittelystä. Tarkoituksena on kertoa, mihin aerodynaamisiin lakeihin ja tekniikkaan laskuvarjon lentäminen perustuu. Niiden avulla voit opetella tuntemaan varjosi ominaisuudet paremmin ja hyödyntää niitä. Opit hallitsemaan varjoa myös yllättävissä tilanteissa.  


Tämä opas on hyvä alku oppimiselle, mutta hyvä kuvunkäsittelijä sinusta tulee vasta, kun maltat harjoitella ilmassa kärsivällisesti ja määrätietoisesti. 

