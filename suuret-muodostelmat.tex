
Suurien muodostelmien (yli 8-way) hyppääminen on luonteeltaan hidastempoisempaa kuin pienempien. Uloshyppy, muodostelman lähestyminen, lentäminen muodostelmassa sekä muodostelman purku poikkeavat pienempien muodostelmien hyppäämisestä. 


Uloshypyt tapahtuvat ilman otteita, otteita on vain muutamia (pohjassa) tai uloshyppymuodostelmasta \textit{kiivetään} varsinaiseen, suunniteltuun muodostelmaan. Suurien muodostelmien uloshyppyjen täytyy olla tiiviitä ja nopeita, jotta vältetään suuret etäisyyserot hyppääjien välillä heti uloshypyn jälkeen. Suuren muodostelman uloshyppyä pitää harjoitella riittävästi, jotta uloshypystä saadaan sujuva. Uloshyppy tulee suunnitella niin, että muodostelmaa lähestyttäessä vältytään ristiin lentämisiltä. 


Suuri muodostelma voidaan jakaa sektoreihin. Muodostelmaa lähestytään omasta sektorista ja muodostelman tasolta. Suuren muodostelman rakentuminen lähtee pohjasta, josta muodostelma laajenee. Lähestymällä muodostelmaa sektoreittain vältytään ristiin lentämisiltä muiden hyppääjien kanssa. Jos muodostelmaa ei lähestytä tasolta, on vaarana muodostelman alle tai päälle joutuminen, jolla puolestaan saattaa olla arvaamattomat seuraukset. 


Suuressa muodostelmassa lentäminen ja liikkuminen on rauhallisempaa verrattuna pieneen. Kun on päästy omalle paikalle muodostelmassa ja otettu varmat otteet, jatketaan aktiivista lentämistä pitämällä painetta kuvan keskustaan aiheuttamatta vetoa otteisiin. 


Suuren muodostelman purku tapahtuu yleensä vaiheittain. Muodostelman uloimmat hyppääjät lähtevät liukumaan aikaisemmin kuin muodostelman keskusta. Tällä taataan kaikille riittävästi aikaa liukua erilleen ja vapaaseen ilmatilaan varjon avausta varten. Varsinkin suurissa muodostelmissa liu'un merkitys korostuu. Ainoa keino varmistaa itselleen riittävästi turvallista tilaa avaukseen on osata liukua kantavasti ja pitkälle. 

