\section{ 8'' }
\label{nova-peruskoulutuksen-suoritukset-8}

\subsection{ Oppilaan toiminta }
\label{nova-peruskoulutuksen-suoritukset-oppilaan-toiminta}


Käytä tasokoulutuksessa opittuja asioita: suora uloshyppy, asennon stabilointi uloshypyn jälkeen, korkeuden tarkkailu ja stabiili asento. 

\subsection{ Oppimistavoitteet }
\label{nova-peruskoulutuksen-suoritukset-oppimistavoitteet}

\begin{enumerate}[label=\bfseries \arabic*)]
\item  Stabiili asento 5 sekunnissa 
\item  Avaustoimenpiteet oikeaan aikaan (7-10 s uloshypystä) 
\end{enumerate}
\subsection{ Hyppylennolla }
\label{nova-peruskoulutuksen-suoritukset-hyppylennolla}


Tarkista omat varusteesi, ilmatila, pilvet ja huomioi mahdollinen exit-väli. 

\subsection{ Hypyn kulku }
\label{nova-peruskoulutuksen-suoritukset-hypyn-kulku}

\begin{description}
\item[TAIVUTA] \hfill \\ 
Tee uloshyppy koneesta ottaen välittömästi ilmavirtaan päästyäsi hyvä taivutus.  \hfill \\ 
Rentouta asento uloshypyn jälkeen  \hfill \\ 
Tarkkaile korkeutta  \hfill \\ 
\end{description}
\begin{description}
\item[8-SEKUNTIA ] \hfill \\ 
Aloita avaustoimenpiteet: Näytä avausmerkki.  \hfill \\ 
\end{description}
\begin{description}
\item[TAIVUTA] \hfill \\ 
Varmistetaan taivutus ja asento.  \hfill \\ 
\item[TARTU] \hfill \\ 
Vasen käsi vartalon jatkeeksi ja oikea käsi apuvarjolle, tiukka ote. \hfill \\ 
\end{description}
\begin{description}
\item[VEDÄ] \hfill \\ 
Heitä apuvarjo ilmavirtaan. \hfill \\ 
Palauta perusasento.  \hfill \\ 
\end{description}
\begin{description}
\item[101] \hfill \\ 
Aloitetaan laskeminen alusta vedon jälkeen.  \hfill \\ 
\item[102..104] \hfill \\ 
Varjo avautuu ja tehdään sen lopullinen avaaminen.  \hfill \\ 
\item[105] \hfill \\ 
Vilkaistaan olkapään yli tarvittaessa (turbulenssin poisto). \hfill \\ 
\end{description}

\end{multicols}\pagebreak\begin{multicols}{2} 

\section{ 5'' }
\label{nova-peruskoulutuksen-suoritukset-5}

\subsection{ Oppilaan toiminta }
\label{nova-peruskoulutuksen-suoritukset-oppilaan-toiminta}


Käytä tasokoulutuksessa opittuja asioita: suora uloshyppy, asennon stabilointi uloshypyn jälkeen, korkeuden tarkkailu ja stabiili asento. 

\subsection{ Oppimistavoitteet }
\label{nova-peruskoulutuksen-suoritukset-oppimistavoitteet}

\begin{enumerate}[label=\bfseries \arabic*)]
\item  Avaustoimenpiteet stabiilista asennosta oikeaan aikaan. (4-7 s uloshypystä) 
\end{enumerate}
\subsection{ Hyppylennolla }
\label{nova-peruskoulutuksen-suoritukset-hyppylennolla}


Tarkista omat varusteesi, ilmatila, pilvet ja huomioi mahdollinen exit-väli. 

\subsection{ Hypyn kulku }
\label{nova-peruskoulutuksen-suoritukset-hypyn-kulku}

\begin{description}
\item[TAIVUTA] \hfill \\ 
Tee uloshyppy koneesta ottaen välittömästi ilmavirtaan päästyäsi hyvä taivutus.  \hfill \\ 
Rentouta asento uloshypyn jälkeen  \hfill \\ 
Tarkkaile korkeutta  \hfill \\ 
\end{description}
\begin{description}
\item[5-SEKUNTIA ] \hfill \\ 
Aloita avaustoimenpiteet: Näytä avausmerkki.  \hfill \\ 
\end{description}
\begin{description}
\item[TAIVUTA] \hfill \\ 
Varmistetaan taivutus ja asento.  \hfill \\ 
\item[TARTU] \hfill \\ 
Vasen käsi vartalon jatkeeksi ja oikea käsi apuvarjolle, tiukka ote \hfill \\ 
\end{description}
\begin{description}
\item[VEDÄ] \hfill \\ 
Heitä apuvarjo ilmavirtaan \hfill \\ 
Palauta perusasento.  \hfill \\ 
\end{description}
\begin{description}
\item[101] \hfill \\ 
Aloitetaan laskeminen alusta vedon jälkeen.  \hfill \\ 
\item[102..104] \hfill \\ 
Varjo avautuu ja tehdään sen lopullinen avaaminen.  \hfill \\ 
\item[105] \hfill \\ 
Vilkaistaan olkapään yli tarvittaessa (turbulenssin poisto). \hfill \\ 
\end{description}
