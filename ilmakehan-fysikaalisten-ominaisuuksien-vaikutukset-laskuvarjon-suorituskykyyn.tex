
Jotta hyppääjä voisi ymmärtää, miten laskuvarjo toimii erilaisissa olosuhteissa, tulee hänen myös perehtyä siihen väliaineeseen, jossa varjo lentää, eli ilmaan. Koska varjon ympärillä virtaavan ilman ominaisuudet vaihtelevat olosuhteista riippuen, vaihtelee varjon suorituskyky erilaisissa ympäristöissä. Nämä muuttujat voi karkeasti jakaa kahteen ryhmään: säätila ja ilman fysikaaliset ominaisuudet. 

\section{ Säätilan vaikutukset }
\label{ilmakehan-fysikaalisten-ominaisuuksien-vaikutukset-laskuvarjon-suorituskykyyn-saatilan-vaikutukset}


Tuulen vaikutuksesta varjolla lentämiseen ja uloshyppypaikan määrittämiseen puhutaan enemmän myöhemmin. Huomattavin ilmakehän fysikaalinen ilmiö laskuvarjon lentämisen kannalta on turbulenssi. 


Turbulenssia eli ilman pyörteitä syntyy seuraavilla tavoilla: 

\begin{enumerate}[label=\bfseries \arabic*)]
\item  Ilmavirran kohdatessa mekaanisen esteen (esimerkiksi toisen kuvun jättämä turbulenssi, rakennus, mäki tai metsän reuna) syntyy turbulenssia. Maanpinnan epätasaisuuksien tai rakennusten aiheuttama turbulenssi voi ulottua voimakkaalla tuulella sen aiheuttaneesta esteestä kymmenistä aina satojen metrien päähän. Hyvänä nyrkkisääntönä voidaan pitää sitä, että pyörteet voivat ulottua ainakin 10 kertaa esteen korkeuden etäisyydelle siitä. 
\item  Varjon lentäessä ilmassa kuvun jättämät pyörteet ulottuvat kymmeniä metrejä kuvun taakse. Kannattaa joskus suunnitellusti lentää korkealla toisen varjon perässä ja seurata, kuinka se vaikuttaa varjon käyttäytymiseen. Älä kuitenkaan laskeudu suoraan toisen hyppääjän perässä. Toisaalta, jos aiot tehdä swooppilaskun, huomioi jälkeesi tulevat äläkä ota vauhtia juuri toisten varjojen edessä.   
\item  Eri ominaisuuksia omaavien ilmamassojen (termiikki) tai -virtausten (kerrostuulet) rajapinnoilla syntyy turbulenssia. Lämmin ilma pyrkii nousemaan ylöspäin ja paikalle virtaa viileämpää ilmaa sitä korvaamaan. Paikallisesti ilmavirtausten suunnat voivat vaihdella suuresti, etenkin eri lämpiämisominaisuuksia omaavien alustojen raja-alueilla. Erityisesti keväisin maaston ja ilman lämpötilaerot voivat olla erittäin suuria: asfalttiplatta, metsä ja lumihanki lämmittävät aurinkoisella säällä ilmaa erittäin epätasaisesti. Myös kesällä viileän veden ja lämpimän hiekkarannan lämpötilaero aiheuttaa ilman epästabiiliutta. Ukkospilvissä ja niiden lähistöllä ilmavirtaukset voivat olla erittäin voimakkaita ja tuuli erittäin puuskaista. Älä siis koskaan hyppää ukkospilveen tai sellaisen ollessa lähellä hyppypaikkaa. 
\end{enumerate}

Turbulenssin koko voi vaihdella senttimetreistä kymmeniin metreihin ja voimakkuus heikoista ja huomaamattomista aina koko varjon tukahduttaviin asti. Turbulenssi vaikuttaa varjoon niin, että lentäminen tuntuu ”pomppuisalta”. Reunatunnelit voivat tukahtua, tai ääritilanteessa koko kupu voi tyhjentyä ilmasta. Kupu tyhjenee, kun ilmavirta kohtaa kuvun sellaisesta suunnasta, että ilmavirran kuvun sisään ohjaava patopiste siirtyy kuvun suuaukkojen ylä- tai alapuolelle ja ilmavirta alkaakin virrata kuvusta ulos. Varjo alkaa lentää uudelleen vasta kun kupuun kohdistuva ilmavirta tulee oikeasta suunnasta. Nykyisillä varjoilla kannattaa turbulenttisessa kelissä lentää täydessä liidossa, jolloin tunneleissa on mahdollisimman suuri paine. Nopeita ja voimakkaita ohjausliikkeitä tulee välttää. Älä ota etummaisista vauhtia laskeutumiseen turbulenttisella kelillä. Vaikka näin saavutettu vauhti nostaakin kuvun sisäistä painetta, saattaa vauhdinottohetkellä kupuun kohdistuva ilmavirta turbulenssin vuoksi tulla väärästä suunnasta ja varjo voi tukahtua matalalla. 

\section{ Ilmanpaine }
\label{ilmakehan-fysikaalisten-ominaisuuksien-vaikutukset-laskuvarjon-suorituskykyyn-ilmanpaine}


Merenpinnan tasolla ilmanpaine on keskimäärin 1013 mbar. Ylöspäin mentäessä ilmanpaine laskee siten, että 600 metrissä se on noin 93\%, 4000 metrissä noin 60\% ja 6000 metrissä noin puolet alkuperäisestä. Sääilmiöiden ääripäissä ilmanpaine voi matalimmillaan meren pinnan tasolla olla noin 950 mbar ja korkeimmillaan 1040 mbar, mikä vastaa n. 450 metrin korkeuseroa. Tämän voit havaita maanpinnalla myös korkeusmittarissa eri päivien välillä tapahtuvina muutoksina. Mitä korkeampi ilmanpaine on, sitä tiheämpää ilma on ja sitä paremmin se kantaa. 

\section{ Ilman tiheys }
\label{ilmakehan-fysikaalisten-ominaisuuksien-vaikutukset-laskuvarjon-suorituskykyyn-ilman-tiheys}


Ilman tiheys, tai siitä laskettava tiheyskorkeus, on käyttökelpoisin suure mietittäessä erilaisten olosuhteiden vaikutusta varjon suorituskykyyn. Tiheyskorkeutta verrataan standardiolosuhteisiin, joissa ilma ei sisällä kosteutta, vallitseva ilmanpaine on 1013 mbar, lämpötila +15 °C ja ilman tiheys 1,225 kg/m³. Ilman tiheys pienenee korkeuden lisääntyessä niin, että 600 metrissä se on noin 94 \%, 4000 metrissä 66 \% ja 6000 metrissä noin 58 \% verrattuna ilman tiheyteen meren pinnan tasolla. Mitä harvempaa ilma on, sitä huonommin se kantaa. 

\section{ Ilman lämpötila }
\label{ilmakehan-fysikaalisten-ominaisuuksien-vaikutukset-laskuvarjon-suorituskykyyn-ilman-lampotila}


Lämpenevä ilma pyrkii laajenemaan, jolloin ilmanpaine nousee. Lämpenevän ilman päästessä vapaasti laajenemaan sen tiheys samalla laskee. Kylmä ilma on siis tiheämpää kuin lämmin, ja se kantaa paremmin. Ilman lämmetessä se pyrkii nousemaan, mikä ilmenee termiikkeinä ja pyörteinä. 

\section{ Ilman kosteus }
\label{ilmakehan-fysikaalisten-ominaisuuksien-vaikutukset-laskuvarjon-suorituskykyyn-ilman-kosteus}


Koska höyrystynyt vesi on kevyempää kuin kuiva ilma, pienenee ilman tiheys kosteuden lisääntyessä. Kuiva ilma on tiheämpää kuin kostea, ja se kantaa paremmin. 

\section{ Tiheyskorkeuden vaikutus varjon käyttäytymiseen }
\label{ilmakehan-fysikaalisten-ominaisuuksien-vaikutukset-laskuvarjon-suorituskykyyn-tiheyskorkeuden-vaikutus-varjon-kayttaytymiseen}


Mitä pienempi tiheyskorkeus on (eli mitä suurempi on ilman tiheys), sitä tehokkaammin varjo lentää, sitä paremmin se fleeraa, sitä nopeammin se oikaisee käännöksen ja palaa itsestään vakaaseen lentotilaan. Tiheyskorkeus vaihtelee edellä mainittujen asioiden vaikutuksesta ja vaikuttaa varjon lentämiseen; äärioloissa jopa varjon valintaan (laskeutumisalue korkealla). Tiheyskorkeuden vaikutusta voidaan valottaa seuraavalla esimerkillä. Verrataan kahta erilaista päivää samassa paikassa sijaitsevalla laskeutumisalueella: 

\begin{enumerate}[label=\bfseries \arabic*)]
\item  Kuuma (+35 °C), kostea (lähes 100 \% suhteellinen ilmankosteus, ukkoskuuroja) kesäpäivä ja matala (980 mbar) ilmanpaine. Laskemalla ilman tiheydeksi saadaan 1,11 kg/m³. Tämä vastaa ilman tiheyttä n. 1050 metrissä. 
\item  Viileä (+2 °C), kuiva (ilmankosteus 25 \%) kevätpäivä ja korkea (1040 mbar) ilmanpaine. Johtuen ilman viileydestä ja kuivuudesta ilman tiheys (1,31 kg/m³) vastaa ilman tiheyttä noin 700 metrissä. 
\end{enumerate}

Kahden päivän välinen ero tiheyskorkeudessa voi ääritapauksissa vastata jopa 2000 metrin eroa laskeutumisalueen korkeudessa. Tätä voidaan verrata varjon ominaisuuksien (vajoamisnopeus, fleeri, sakkaaminen) kannalta laskeutumiseen 120 neliöjalan varjolla 135 neliöjalkaisen varjon sijasta. Jos tarkoituksena on ottaa vauhtia laskeutumiseen, voi vauhdinoton aloituspiste olla lämpimällä ja kostealla ilmalla kymmeniä metrejä ja fleerin aloituskorkeus metrejä korkeammalla kuin viileällä ja kuivalla ilmalla. Tämän asian huomioimatta jättäminen voi erityisesti nopeita laskeutumisia tehtäessä johtaa vakavaan vaaratilanteeseen tai loukkaantumiseen. Erityisesti mentäessä uudelle hyppypaikalle, joka sijaitsee eri korkeudessa ja eri ilmastossa totuttuun verrattuna, kannattaa asiaa miettiä etukäteen. 

