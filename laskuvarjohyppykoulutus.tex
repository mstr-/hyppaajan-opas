\section{ Alkeiskurssi - ensimmäinen askel }
\label{laskuvarjohyppykoulutus-alkeiskurssi-ensimmainen-askel}


Alkeiskurssi antaa valmiudet suorittaa turvallisesti ensimmäisen laskuvarjohypyn sekä pohjan perus- ja jatkokoulutukselle. Tiivis koulutusohjelma vaatii sinulta aktiivisuutta sekä opetettujen asioiden kertaamista. Kuuntele, kysy ja selvitä. Ensimmäisen hyppysi suoritat joko pakkolaukaisuhyppynä 1000 metristä tai novakoulutuksessa itseaukaisuhyppynä 2700-4000 metrin korkeudesta. Sitä ennen sinun tulee suorittaa kirjallinen koe sekä käytännön kokeet uloshypystä, maahantulokierähdyksestä ja varavarjon käytöstä. Tämän jälkeen olet valmis suorittamaan laskuvarjohyppyjä hyppymestarin valvonnassa. Hyppymestari valvoo ja arvioi kaikki suorituksesi sekä tekee niistä merkinnän hyppypäiväkirjaasi joko hyväksyen tai hyläten suorituksesi. Jälkimmäinen vaihtoehto merkitsee uutta yritystä ennen eteenpäin pääsyä. Aina ennen uutta suoritusta sinun tulee saada koulutus hyppymestarilta. Tässä oppaassa on materiaali, jota tulet käyttämään kaikissa kolmessa koulutusvaiheessa. 

\subsection{Alkeiskoulutus}
\label{laskuvarjohyppykoulutus-alkeiskoulutus}


Alkeiskoulutuksen tavoitteena on oppia toimimaan lentokoneessa, suorittaa uloshyppy, hallittu itsenäinen vapaapudotus, varjon avaaminen ja ohjaaminen itsenäisesti laskeutumisalueelle, sekä turvallinen laskeutuminen. Alkeiskoulutuksen päätteeksi suoritetaan teoriakoe. 

\subsection{Peruskoulutus}
\label{laskuvarjohyppykoulutus-peruskoulutus}


Peruskoulutuksen tavoitteena on oppia stabiili vapaapudotus yhdessä erilaisten liikkeiden ja liikesarjojen kanssa. Ohjelmassa on käännöksiä, voltteja ja selällään lentämistä. Tämän jälkeen teet teoria- sekä varusteiden tarkastuskokeen. Lisäksi kertaat vaaratilanteet ja varavarjotoimenpiteet, jonka jälkeen siirryt jatkokoulutukseen. 

\subsection{Jatkokoulutus}
\label{laskuvarjohyppykoulutus-jatkokoulutus}


Tässä vaiheessa olet jo itsenäinen oppilas etkä enää välttämättä tarvitse kouluttajaa koneeseen. Opettelet erilaisia hyppylajeja, kuten ryhmä- ja freehyppyjä. Keskityt myös tarkkuuslaskeutumisiin sekä saat mahdollisuuden käyttää muita kuin koulutuskäyttöön tarkoitettuja hyppyvarusteita. Tärkein tavoitteesi on oppia turvallisen ryhmähyppäämisen perusteet. Lopuksi suoritat vielä teoriakokeen ja käytännön kokeen oppilaspäävarjon pakkaamisesta ja tarkastuksesta sekä vaaratilanne- ja varavarjotoimenpiteiden kertauksen. Koko oppilasaika päättyy kokeiden lisäksi koehyppyyn, jossa osoitat osaavasi käytännön hyppäämiseen liittyvät perusasiat. Näiden suoritusten jälkeen koulutus\mbox{-,} apulaiskoulutuspäällikkö tai hyppymestari myöntää sinulle itsenäisen hyppääjän kelpoisuuden. 

\subsection{A-lisenssi}
\label{laskuvarjohyppykoulutus-a-lisenssi}


Itsenäisen hyppääjän kelpoisuuden saamisen jälkeen voit hakea A-lisenssiä Suomen Ilmailuliitosta. A-lisenssihyppääjällä on oikeus Ilmailulain, Ilmailumääräysten, SIL:n ohjeiden ja kerhon määräyksien rajoissa itsenäiseen hyppytoimintaan. Lisäksi SIL:n myöntämä lisenssi helpottaa hyppäämistä ulkomailla. 

