\section{ Hypyn hahmottaminen }
\label{mielikuva-eli-mentaaliharjoittelu-hypyn-hahmottaminen}


Koska laskuvarjohyppäämistä on vaikea harjoitella maassa ja vapaapudotusaika on lyhyt, korostuu mielikuvaharjoittelun merkitys. Mielikuvien avulla hyppy pyritään suorittamaan ajatuksissa samalla tavalla kuin se halutaan taivaalla tehdä. Mentaalien suorittamiseen ei ole yhtä ainoaa oikeaa tapaa, vaan jokaisen on löydettävä oma tapansa hahmottaa ja visualisoida hyppy. Yleisimpiä tapoja hypyn hahmottamiseen mielessä ovat hypyn näkeminen omasta näkökulmasta eli muodostelman tasolta katsottuna ja/tai ylhäältä eli kuvaajan näkökulmasta. 


Mielikuvaharjoittelun voi aloittaa, kun hypyn muodostelmat on päätetty. Uloshyppyharjoittelun ja lautailun aikana pyritään muodostamaan mielikuvia siitä, mitä ilmassa tulee nähdä sekä miltä liikkeet tuntuvat ja muodostaa näistä mielikuvista kokonainen hyppy. Hyppyä kannattaa käydä läpi mielessä niin kauan, että se toimii. Jos hyppy tuntuu mielikuvissa jotenkin vaikealta, kannattaa palata harjoittelemaan laudoille tai konemallille. Mielikuvan hypystä tulee olla selkeä ja toimiva ennen koneeseen menoa, sillä jos hypyn mieltämisessä tai muistamisessa on ongelmia, niin ongelmat todennäköisesti tulevat esiin myös itse hypyllä. Koneessa hyppy kannattaa käydä muutaman kerran läpi mielessä, lähinnä muistutukseksi - ei niinkään enää oppimismielessä. Hyppyä ei kuitenkaan kannata stressata liikaa, jotta ei aiheuta itselleen ylivirittynyttä oloa. 

\section{ Unohtelu eli brainlocking }
\label{mielikuva-eli-mentaaliharjoittelu-unohtelu-eli-brainlocking}


Hyppäämisessä puhutaan usein brainlockeista, joilla tarkoitetaan hypyn aikana tulevaa ajatuskatkoa, jolloin ei muista mitä seuraavaksi pitää tehdä. Nämä hetket voivat olla hyvinkin lyhyitä. Yleisimpiä syitä brainlockin syntyyn ovat mm. hapenpuute, väsymys, kiire, alhainen verensokeri, vähäinen harjoittelu ja hypyn vääränlainen tai puuttuva hahmottaminen. Näiden osatekijöiden huomioiminen ja välttäminen ehkäisee suurelta osin brainlockien syntymistä 

