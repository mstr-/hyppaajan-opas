
Kuviohypyissä (Formation Skydiving, FS) hyppääjät tekevät kuvioita vapaapudotuksessa ottamalla otteita toisista hyppääjistä. Yleisin FS-hyppäämisen kilpailumuoto on 4-way, jossa joukkue muodostuu neljästä jäsenestä, kuvaajasta sekä varamiehestä. Kilpailuissa 4-way hypätään 3000 metristä ja työskentelyaika on 35 sekuntia. Kilpailuhypyt arvotaan IPC:n julkaisemasta Dive poolista. 

\section{ Lyhyt historia }
\label{yleista-fs-hyppaamisesta-lyhyt-historia}

\subsubsection{ 1970-luku }
\label{yleista-fs-hyppaamisesta-1970-luku}

\begin{itemize}
\item  RW-hyppääminen alkaa (Relative Work)  
\item  1974 ensimmäinen World Cup Etelä-Afrikassa  
\item  1975 ensimmäiset SM-kilpailut järjestetään  
\item  1975 ensimmäiset MM-kilpailut Länsi-Saksassa  
\end{itemize}
\subsubsection{ 1980-luku }
\label{yleista-fs-hyppaamisesta-1980-luku}

\begin{itemize}
\item  1984 Mike Zahar kehitti 4- ja 8-way blokit ja randomit kilpailuihin  
\item  1985 suomalainen Skorpions sijoittui 5. sijalle MM-kisoissa Brasiliassa  
\item  1986 ensimmäinen virallinen 100-way, Muskogee USA  
\end{itemize}
\subsubsection{ 1990-luku }
\label{yleista-fs-hyppaamisesta-1990-luku}

\begin{itemize}
\item  1991 Relative Workistä tulee Formation Skydiving  
\item  1993 suomalainen Madway sijoittui 4. sijalle MM-kisoissa USA:ssa  
\item  1993 Suomen ennätys 57-way  
\item  1997 Max 5 teki Suomen ennätyksen 24 pistettä kilpailuhypyllä  
\item  1999 Suomen naisten ennätys 30-way  
\end{itemize}
\subsubsection{ 2000-luku }
\label{yleista-fs-hyppaamisesta-2000-luku}

\begin{itemize}
\item  2002 ranskalainen Maubeuge teki maailman ennätyksen 42 pistettä kilpailuhypyllä  
\item  2003 suomalainen naisjoukkue Zooey sijoittui 8. sijalle naisten sarjassa MM-kisoissa Ranskassa  
\item  2004 maailman ennätys 357-way, Thaimaa 
\item  2006 maailman ennätys 400-way, Thaimaa 
\end{itemize}
