\section{ Selkälento }
\label{peruskoulutuksen-muut-suoritukset-selkalento}


Selkälennossa rintahihnassa oleva korkeusmittari ei välttämättä näytä oikein. 

\subsection{ Oppilaan toiminta }
\label{peruskoulutuksen-muut-suoritukset-oppilaan-toiminta}

\begin{itemize}
\item Harjoittele suorituksen vaiheet: 
	\begin{enumerate}[label=\bfseries \arabic*)]
	\item  Käännetään asento kyljelleen käyttämällä kättä vartalossa 
	\item  Taivutetaan samalla asento väärinpäin, istuma-asentoon, jalat koukistettuina 
	\item  Lennetään muutama sekunti selällään (huomioidaan korkeus) 
	\item  Käännetään asento perusasentoon taivuttamalla 
	\end{enumerate}
\end{itemize}
\subsection{ Oppimistavoitteet }
\label{peruskoulutuksen-muut-suoritukset-oppimistavoitteet}

\begin{enumerate}[label=\bfseries \arabic*)]
\item  Pääsy perusasennosta selkälentoon 
\item  Asennon hallinta selällään (pieni kääntyminen ok, lattakierre ei) 
\end{enumerate}
\subsection{ Hyppylennolla }
\label{peruskoulutuksen-muut-suoritukset-hyppylennolla}

\begin{itemize}
\item Kertaa suoritus mielessäsi 
\item Keskity suoritukseesi 
\item 3X3 -tarkastus ennen hyppyä 
\end{itemize}
\subsection{ Hypyn kulku }
\label{peruskoulutuksen-muut-suoritukset-hypyn-kulku}

\begin{description}
\item[TAIVUTA] \hfill \\ 
Tee uloshyppy koneesta ottaen välittömästi ilmavirtaan päästyäsi hyvä taivutus. \hfill \\ 
Rentouta asento uloshypyn jälkeen \hfill \\ 
Ota ensin perusasento ja käänny tämän jälkeen selälleen \hfill \\ 
Tarkkaile korkeutta \hfill \\ 
Muutaman sekunnin jälkeen käännä asento takaisin mahalleen perusasentoon \hfill \\ 
Mikäli korkeutta on vielä yli 1800 metriä, toista harjoitus. \hfill \\ 
\item[1800 metriä] \hfill \\ 
Lopeta työskentely ja valmistaudu päävarjon avaamiseen.  \hfill \\ 
\item[1200 metriä] \hfill \\ 
Aloita avaustoimenpiteet. \hfill \\ 
\end{description}

\end{multicols}\pagebreak\begin{multicols}{2} 

\section{ Liuku }
\label{peruskoulutuksen-muut-suoritukset-liuku}


Liu’un tavoitteena on liikkua vaakatasossa mahdollisimman kauaksi muihin hyppääjiin nähden samalla, kun pudotaan alaspäin. Liuku ei ole siis syöksyä tai tikkaamista. Putoamisnopeus voi kasvaa liu’un aikana. Korkeuden tarkkailu on tärkeää, sillä liu’un pysäytys vaatii myös aikansa. Varjon avaamista suoraan liu’usta ei suositella. Hyvä liuku on ensiarvoisen tärkeä taito laskuvarjohyppääjällä, sillä ainoastaan hyvällä liu’ulla pystyy varmistamaan riittävän välimatkan muihin hyppääjiin purun jälkeen. 


Liuku tehdään seuraavasti: 

\begin{enumerate}[label=\bfseries \arabic*)]
\item Otetaan kiintopiste edestä, maasta. 
\item Oikaistaan jalat nilkkoja myöten. 
\item Viedään kädet sivuille taakse, lähelle vartaloa. 
\item Painetaan olkapäät alas eteen. 
\item Ohjataan liukusuuntaa kämmenillä. 
\item Tarkkaillaan korkeutta, muita hyppääjiä sekä liu'utaan kohti kiintopistettä. 
\item Palautetaan perusasento rauhallisesti ottamalla ensin taivutus ja sen jälkeen kädet ja jalat perusasentoon 
\end{enumerate}
\subsection{ Oppilaan toiminta }
\label{peruskoulutuksen-muut-suoritukset-oppilaan-toiminta}

\begin{itemize}
\item  Ota selvää hyppylinjan suunnasta, jotta osaat valita oikean liukusuunnan (poispäin linjasta). 
\end{itemize}
\subsection{ Oppimistavoitteet }
\label{peruskoulutuksen-muut-suoritukset-oppimistavoitteet}

\begin{enumerate}[label=\bfseries \arabic*)]
\item  Liu'u sovittuun suuntaan ja pysy suunnassa 
\item  Liuku liikkuu eteenpäin 
\item  Liuku ei nyöki 
\end{enumerate}
\subsection{ Hyppylennolla }
\label{peruskoulutuksen-muut-suoritukset-hyppylennolla}

\begin{itemize}
\item Kertaa suoritus mielessäsi 
\item Keskity suoritukseesi 
\item 3X3 -tarkastus ennen hyppyä 
\end{itemize}
\subsection{ Hypyn kulku }
\label{peruskoulutuksen-muut-suoritukset-hypyn-kulku}

\begin{description}
\item[TAIVUTA] \hfill \\ 
Tee uloshyppy koneesta ottaen välittömästi ilmavirtaan päästyäsi hyvä taivutus. \hfill \\ 
Rentouta asento uloshypyn jälkeen \hfill \\ 
Käänny liukusuuntaan, ota liukuasento ja liu'u noin 5 sekuntia \hfill \\ 
Pysäytä liuku \hfill \\ 
Tarkkaile korkeutta \hfill \\ 
Mikäli korkeutta on vielä jäljellä, käänny 180° ja toista liuku. \hfill \\ 
\item[1600 metriä] \hfill \\ 
Lopeta työskentely ja valmistaudu päävarjon avaamiseen.  \hfill \\ 
\item[1200 metriä] \hfill \\ 
Aloita avaustoimenpiteet. \hfill \\ 
\end{description}

\end{multicols}\pagebreak\begin{multicols}{2} 

\section{ FS-liuku }
\label{peruskoulutuksen-muut-suoritukset-fs-liuku}


FS-liuku on liikesarja, joka suoritetaan jokaisen FS-hypyn lopuksi. Liikesarjassa annettavilla merkeillä viestitään kanssahyppääjille niitä toimenpiteitä, mitä on aikomus seuraavaksi tehdä. 

\subsection{ Oppilaan toiminta }
\label{peruskoulutuksen-muut-suoritukset-oppilaan-toiminta}


FS-liuku sisältää seuraavan liikesarjan: 

\begin{enumerate}[label=\bfseries \arabic*)]
\item Purkumerkki 
\item Käännös 180° 
\item Muutaman sekunnin liuku 
\item Ilmatilan tarkastus 
\item Avausmerkki 
\item Harjoitusveto 
\end{enumerate}
\subsection{ Oppimistavoitteet }
\label{peruskoulutuksen-muut-suoritukset-oppimistavoitteet}

\begin{enumerate}[label=\bfseries \arabic*)]
\item  Tee kaikki FS-liukuun liittyvät merkit ja tarkastukset 
\item  Tee liuku, joka liikkuu eteenpäin 
\item  Käännös on 180° ja liuku pysyy suunnassa 
\end{enumerate}
\subsection{ Hyppylennolla }
\label{peruskoulutuksen-muut-suoritukset-hyppylennolla}

\begin{itemize}
\item Kertaa suoritus mielessäsi 
\item Keskity suoritukseesi 
\item 3X3 -tarkastus ennen hyppyä 
\end{itemize}
\subsection{ Hypyn kulku }
\label{peruskoulutuksen-muut-suoritukset-hypyn-kulku}

\begin{description}
\item[TAIVUTA] \hfill \\ 
Tee uloshyppy koneesta ottaen välittömästi ilmavirtaan päästyäsi hyvä taivutus. \hfill \\ 
Rentouta asento uloshypyn jälkeen \hfill \\ 
Ota maasta/horisontista kiintopiste ja aloita FS-liuku \hfill \\ 
Tee liikesarja ja päätä liike harjoitusvetoon \hfill \\ 
Tarkista korkeus \hfill \\ 
Mikäli korkeutta on vielä jäljellä, toista FS-liuku. \hfill \\ 
\item[1600 metriä] \hfill \\ 
Lopeta työskentely ja valmistaudu päävarjon avaamiseen.  \hfill \\ 
\item[1200 metriä] \hfill \\ 
Aloita avaustoimenpiteet. \hfill \\ 
\end{description}
